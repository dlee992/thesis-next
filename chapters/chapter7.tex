\chapter{总结与展望}
本章对本文的工作进行总结,并展望与电子表格缺陷检测相关的研究中需要解决的问题和可能存在的解决方案。


\section{工作总结}
电子表格在终端用户开发工具中变得越来越流行,能够帮助用户迅速高效地完成数值存储和公式计算等任务,在各个领域都获得广泛应用。
然而,电子表格容易产生错误的特性也给许多终端用户带来了麻烦,给众多公司带来了经济损失,以及随着表格复杂化带来的较高的维护成本。

针对电子表格中的公式缺陷问题,近二十年来,研究者们提出了许多不同思路的缺陷检测技术,致力于提升电子表格的可靠性。
根据检测方法的设计思路可以分成三类,分别是基于类型推导的技术,基于规则和模式匹配的技术,以及基于学习/聚类的机器学习技术。
其中,后两类方法逐渐占据主流,前者的特点在于能够相对精准地识别特定类型的公式缺陷,但往往整体的识别召回率不高;而后者的特点恰恰相反,整体的识别召回率相对较高,但由于聚类或学习技术的自动化特征抽取不够精准,往往导致精度不高,使得终端用户在进行人工检查时耗费较多精力。

本文提出的技术方案 \wa 立足于电子表格中的机器学习技术\cu,在聚类的过程中,通过设计全方位的有效性属性检测方法,对所得的单元格类进行检验,将仅仅因为从取得特征相似但计算语义上无关的单元格从类中移除出去。
这些有效性属性关注三个层面的类有效性,即每个单元格自身的层面、多个单元格之间关系的层面和整个类的层面,实现聚类精度提升和缺陷检测效果的提升。
在基准测试集和更大的电子表格数据集上,本文的实验和案例研究结果均表明 \wa 技术在提升聚类准确性和检测电子表格公式缺陷方面的有效性。


\section{研究展望}
就本文提出的基于有效性属性的检验技术而言,从整体上看,\wa 技术仍有提升空间:
\begin{itemize}
    \item 充实有效性属性框架:本文提供了一个由下而上相对完整的基于有效性属性的检验框架,立足于每个单元格自身,到单元格之间的关系,再到单元格类的有效性属性,在整个框架中仍可以添加新的更加有效或者改进当前的有效性属性,进而更好地提升单元格聚类和缺陷检测的有效性;
    \item 吸纳没有被聚类但与计算语义相关的单元格:本文主要关注如何提升聚类和缺陷检测的精度方面。仅从单元格聚类角度来看,当前的算法依然会丢失一些缺乏共性特征,但的确有用相似计算语义的单元格,如何将这些单元格吸纳到对应的单元格类中,是值得进一步思考和探究的角度;
    \item 优化缺陷检测阶段:尽管本文主要关注于在聚类阶段进行优化,但在缺陷检测阶段,目前的技术仍有优化空间,比如借鉴程序合成相关技术为每个单元格类合成若干个能够表达类中每个单元格的公式表达式,以及给终端用户提供针对检测出的公式缺陷的高效修复建议。
\end{itemize}

最后,从终端用户使用电子表格的立场来看,如何能够将电子表格缺陷检测技术集成到电子表格开发环境中,类似传统编程的集成开发环境,即时给用户反馈当前表格中当前正在编辑的公式是否存在缺陷的提示,能够更加有效地帮助终端用户在开发过程中稳步提升电子表格的可靠性。
