\begin{abstract}

终端用户编程获得了广泛关注和应用,其中,电子表格无疑是终端用户编程中最流行的编程范式之一。
电子表格得到用户的广泛接受,并在世界范围内广泛使用。
电子表格软件,例如 Microsoft Excel 和 Google Sheet,通常被用来存储、计算、和分析用户数据,给用户提供数据计算和商业决策的便利。
不过,在电子表格编程范式广泛应用的背后,每一个电子表格中也潜藏着各种类型的错误和缺陷。
近二十年来,多家公司因为使用电子表格时犯错,已经造成了数亿美元的经济损失。

因此,利用软件工程技术帮助终端用户检测相关的电子表格潜在缺陷,已经得到了广泛关注。
过去二十年间,多种不同技术思路的技术和工具被提出,用来检测电子表格中的各类错误和缺陷。
具有代表性的已有工作,大体可以依据技术思路分成两类:基于规则校验和基于聚类/学习算法的缺陷检测技术。
前者通常能够依据设计的精准规则,以较高的准确率检测到特定类型的单元格缺陷,但相对而言检测的召回率不高;
而后者,例如\cu,能够根据公式特征和单元格的布局特征对单元格进行聚类,大幅提高检测的召回率,但相对而言,由于电子表格本身的复杂性,导致对公式的结构和特征提取不够准备,导致最终的检测精度不高。

考虑到上述不足,文本提出了一个新颖的技术\wa ,基于\cu 的先天的适应性的跨表格、跨布局风格的学习能力,但同时改善它在将相关和不相关的单元格吸收进同类中的不足。综上所属,本文的主要贡献有如下三点:
\begin{enumerate}
    \item 提出了电子表格缺陷检测技术\wa ,聚焦于单元格自身、单元格之间关系和整个单元格类的有效性属性来提升 \cu 方法的单元格聚类效果,整体上提升了电子表格缺陷检测的有效性。
    \item 使用了一个现有的电子表格基准测试集和一个大规模的电子表格语料库,对 \wa 进行了充分测试和评估,并对比了现在最流行的其他电子表格缺陷检测技术。
    \item 实现了两个可视化工具:一个基于Java的可视化工具 \sg,可对比多种主流技术的检测结果并集中展示;另一个基于JavaScript的Excel插件 EGuard,可即时、分步地看到 \wa 对当前终端用户打开的电子表格的检测结果,方便终端用户快速确认和修复报告出的缺陷。
\end{enumerate}

\keywords{电子表格测试;单元格聚类;缺陷检测;有效性检验;软件工程}

\end{abstract}