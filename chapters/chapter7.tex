\chapter{总结与展望}

本章将对本文的工作进行总结,然后展望与电子表格错误检测相关的研究中需要解决的问题和可能存在的解决方案。

\section{工作总结}
电子表格在终端用户开发工具中变得越来越流行,帮助用户迅速高效完成数值存储和公式计算等即时任务,在各个领域都获得广泛应用。
然而电子表格容易产生错误的特性也给许多终端用户带来了麻烦,给众多公司带来了经济损失,以及随着表格复杂化带来的较高的维护成本。

针对电子表格中的公式缺陷问题,近二十年来,研究者们提出了许多不同思路的缺陷检测技术,致力于提升电子表格的可靠性。
根据检测方法的设计思路可以分成三类,分别是利用类型推导技术,利用基于规则的模式匹配技术,以及基于学习/聚类的机器学习技术。
其中,后两类方法逐渐占据主流,前者的特点在于能够相对精准地识别特定子类型的公式缺陷,但往往整体的识别召回率不高;而后者的特点恰恰相反,整体的识别召回率相对较高,但由于聚类或学习技术的自动化特征抽取不够精准,往往导致精度不高,使得终端用户在进行人工检查时耗费较多精力。

本文提出的技术方案立足于电子表格中的聚类技术,在聚类的过程中,通过设计和规则类似的有效性属性检测方法,来对所得的单元格类进行精化,将仅仅因为特征相似但语义上无关的单元格从类中剔除出去,实现整体检测效果的提升。
这些有效性属性关注三个层面的类有效性,单个单元格层面,多个单元格层面,以及整个类的层面,能够极大地提升检测精度。
在基准测试集和以更大语料库为测试数据的案例研究中证实\wa 在检测电子表格缺陷方面的有效性。

\section{研究展望}

就本文采用的优化技术而言,\wa 从整体分析来看依然有提升改进的空间。
\begin{itemize}
    \item 有效性属性框架:本文提供了一个从下到上相对完整的有效性属性的思考框架,立足于每个单元格自身,到单元格之间的关系,再到单元格类(即单元格集合)的有效性,在整个框架中仍可以添加新的或者改进当前的有效性属性,进而更好地提升检测精度;
    \item 捞回丢失但相关的单元格:本文只关注于如何提升检测的精度,但就其聚类而言,依然会丢掉一些缺乏共性特征,但也的确属于相同语义的公式,如何将这些公式单元格捞回到对应的类中,也是一个值得进一步思考和探究的角度(已经有相关工作\cite{huang2020warder}发表);
    \item 缺陷检测阶段:另外,本文主要关注于在聚类阶段进行优化,不过,在缺陷检测阶段,也有优化和提升的空间,比如对检测出的缺陷提供额外的修复建议。
\end{itemize}

另外,从终端用户使用电子表格的立场来看,如何能够将电子表格缺陷检测技术集成到电子表格开发环境中,类似编程的集成开发环境一样,即时给用户反馈当前表格,乃至当前正在编辑的公式是否存在缺陷的提示,能够更加有效地辅助终端用户在开发过程中就同时进行提升可靠性的工作。
