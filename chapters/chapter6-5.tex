\section{威胁性分析和讨论}

一个主要的内在威胁性来源是单元格聚类指标的计算方式,即\prc ,\rec 和 \fmc 。
它们是基于真阳性,假阳性和假阴性概念计算而来,而后三者又是根据\cu 的结对相似度比较计算而来。
我们注意到这样的比较会统计是否两个单元格属于同一个类或者属于不同的类的单元格对的数量。
这样的计算方式不同于衡量缺陷检测效果的方式。
因此,研究\wa 的单元格聚类和它的缺陷检测相关性结论可能在一定程度上受到影响。
然而,我们依然观察到 90\% 的工作表在我们的分析下是正相关的。
这表明\wa 对单元格聚类的提升的确有助于最终的缺陷检测。

另外,我们也注意到\wa 啊仍然有提升的空间,考虑到它不能检测到某些特定的电子表格缺陷,正如我们之前分析的那样。
\begin{enumerate}
    \item \wa 关注于识别不相关的单元格(将其移除)和无效的单元格类(取消该类的分析)。它并没有挥手哪些相关的单元格,但被\cu 遗漏掉的部分;
    \item 即便所有相关的单元格都能被正确的聚类,\cu 本身仍然不能检测出某些缺陷,由于其自身在缺陷检测上有限的分析能力。因为\wa 仅仅关注于改善单元格聚类过程,并没有涉及到缺陷检测部分的优化。因此,两种技术可能都无法检测到这样的电子表格缺陷。
\end{enumerate}
不过,我们在实验中观察到:\wa 已经极大地提升了 \cu 的适应性。这表明\wa 关注到了对优化起主导作用的因素。不过上述分析也的确指出了进一步优化的新方向。

最后,一个主要的外部威胁性来源是我们尝试了但没能和另外两个基于机器学习的电子表格缺陷检测技术,Melford\cite{singh2017melford} 和 ExceLint\cite{Barowy2018excelint} 进行对比实验和案例研究。
前者,我们没有找到可获得的工具。后者,我们找到了对应的工具但在实验评估时遇到了问题。
首先,ExceLint 的检测范围和其他六类我们研究的电子表格缺陷检测技术很不一样,它仅仅关注与检测由于错误引用导致的公式不一致性缺陷。
第二,ExceLint 认为公式丢失的缺陷不太重要,因为它们不会立刻触发错误。
然而,其他所有技术都认为这样的缺陷是有害的,并检测这些缺陷,因为这类缺陷在未来的维护过程中,可能导致意料之外的错误。
事实上,公式丢失缺陷在实际的电子表格中很常见(例如,不同技术在 VEnron2 数据集上检测出的此类缺陷占比 64-78\%)。
因此,直接对比\wa 和 ExceLint 可能不太公平,并严重低估 ExceLint 的有效性。另外,我们在实际运行 ExceLint 的过程中遇到了别的问题,比如它缺少人工标记的基准测试集。
因此,我们把对它的比较留给其他研究人员,在将来进行综合的比较和实验。