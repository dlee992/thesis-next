% !TEX program = xelatex
% !Mode:: "TeX:UTF-8"
%# -*- coding:utf-8 -*-

%% 南京大学学位论文的示例文档
%% 作者:njuhan: https://github.com/njuHan
%% 源模版repo: https://github.com/njuHan/njuthesis-nju-thesis-template

\documentclass[master, macfonts, twoside]{njuthesis}
%% njuthesis 文档类的可选参数有:
%%   winfonts, linuxfonts, macfonts, adobefonts winfonts 选项使得文档使用Windows 系统提供的字体;linuxfonts 选项使得文档使用Linux 系统提供的字体;macfonts 选项使得文档使用Mac 系统提供的字体;adobefonts 选项使得文档使用Adobe提供的OTF中文字体(需自行下载安转)
%%   phd/master/bachelor 选择博士/硕士/学士论文
%%   twoside 或 oneside 指定排版的文档为双面打印或单面打印格式(twoside会使得chapter 章节从奇数页开始,即纸张的正面开始,因此会出现一些空白的页面)
%%   nobackinfo 取消封二页导师签名信息。注意,按照南大的规定,是需要签名页的。
\newcommand*{\CJKunderlinecolor}{\color{black}}
\newcommand*{\CJKunderline}[1]{\uline{#1}}


%%%%%%%%%%%%%%%%%%%%%%%%%%%%%%%%%%%%%%%%%%%%%%%%%%%%%%%%%%%%%%%%%%%%%%%%%%%%%%%
% set up labelformat and labelsep for subfigure 详见: http://www.latexstudio.net/archives/8652.html
\captionsetup[subfigure]{labelformat=simple, labelsep=space}

%%%%%%%%%%%%%%%%%%%%%%%%%%%%%%%%%%%%%%%%%%%%%%%%%%%%%%%%%%%%%%%%%%%%%%%%%%%%%%%
% 设置《国家图书馆封面》的内容,仅博士论文才需要填写

% 设置论文按照《中国图书资料分类法》的分类编号
\classification{0175.2}
% 设置论文按照《国际十进分类法UDC》的分类编号
% 该编号可在下述网址查询:http://www.udcc.org/udcsummary/php/index.php?lang=chi
\udc{004.72}
% 国家图书馆封面上的论文标题第一行,不可换行。此属性可选,默认值为通过\title设置的标题。
\nlctitlea{论文标题第一行}
% 国家图书馆封面上的论文标题第二行,不可换行。此属性可选,默认值为空白。
\nlctitleb{论文标题第二行}
% 国家图书馆封面上的论文标题第三行,不可换行。此属性可选,默认值为空白。
\nlctitlec{}
% 导师的单位名称及地址
\supervisorinfo{南京大学计算机科学与技术系~~南京市汉口路22号~~210093}
% 答辩委员会主席
\chairman{张三丰~~教授}
% 第一位评阅人
\reviewera{阳顶天~~教授}
% 第二位评阅人
\reviewerb{张无忌~~副教授}
% 第三位评阅人
\reviewerc{黄裳~~教授}
% 第四位评阅人
\reviewerd{郭靖~~研究员}


%%%%%%%%%%%%%%%%%%%%%%%%%%%%%%%%%%%%%%%%%%%%%%%%%%%%%%%%%%%%%%%%%%%%%%%%%%%%%%%
% 设置论文的中文封面

% 单行论文标题,不可换行
\title{南京大学毕业论文\LaTeX 模板}

% 如果论文标题过长,可以分两行,第一行用\titlea{}定义,第二行用\titleb{}定义,
% 使用以下3行:
%\title{} %用于覆盖单行标题内容为空
%\titlea{长标题第一行}  %第一行标题写这里
%\titleb{长标题第二行用于长标题换行} %第二行标题写这里
% 注意: \title 不能都注释,它用于控制标题选择双行还是单行。\title{}如果内容为空,则编译\titlea{},titleb{}双行标题,否则编译单行标题


% 论文作者姓名
\author{作者}
% 论文作者联系电话
\telphone{xxxx}
% 论文作者电子邮件地址
\email{sample@smail.nju.edu.cn}
% 论文作者学生证号
\studentnum{xxxxxxx}
% 论文作者入学年份(年级)
\grade{2012}
% 论文作者毕业年份(届), 出版授权书的学位年度
\graduateyear{20xx}
% 导师姓名职称
\supervisor{某~~教授}
% 导师的联系电话
\supervisortelphone{}
% 论文作者的学科与专业方向
\major{计算机科学与技术}
% 论文作者的研究方向
\researchfield{分布式计算}
% 论文作者所在院系的中文名称
\department{计算机科学与技术系}
% 论文作者所在学校或机构的名称。此属性可选,默认值为``南京大学''。
\institute{南京大学}
% 论文的提交日期,需设置年、月、日。
\submitdate{xxxx年 xx 月 xx 日}
% 论文的答辩日期,需设置年、月、日。
\defenddate{xxxx年 xx 月 xx 日}
% 论文的定稿日期,需设置年、月、日。
% 此属性可选,若注释\date{},则默认值为最后一次编译时的日期,精确到日。
% \date{2019年5月20日}

%%%%%%%%%%%%%%%%%%%%%%%%%%%%%%%%%%%%%%%%%%%%%%%%%%%%%%%%%%%%%%%%%%%%%%%%%%%%%%%
% 设置论文的英文封面

% 论文的英文标题,不可换行
\englishtitle{\LaTeX \;  NJU thesis template }
% 论文作者姓名的拼音
\englishauthor{Author}
% 导师姓名职称的英文
\englishsupervisor{ Professor}
% 论文作者学科与专业的英文名
\englishmajor{Computer Science and Technology}
% 论文作者所在院系的英文名称
\englishdepartment{Department of Computer Science and Technology}
% 论文作者所在学校或机构的英文名称。此属性可选,默认值为``Nanjing University''。
\englishinstitute{Nanjing University}
% 论文完成日期的英文形式,它将出现在英文封面下方。需设置年、月、日。日期格式使用美国的日期
% 格式,即``Month day, year'',其中``Month''为月份的英文名全称,首字母大写;``day''为
% 该月中日期的阿拉伯数字表示;``year''为年份的四位阿拉伯数字表示。
% 此属性可选,若注释掉\englishdate{},则默认值为最后一次编译时的日期。
% \englishdate{May 20, 2019}

%%%%%%%%%%%%%%%%%%%%%%%%%%%%%%%%%%%%%%%%%%%%%%%%%%%%%%%%%%%%%%%%%%%%%%%%%%%%%%%
% 设置论文的中文摘要

% 设置中文摘要页面的论文标题及副标题的第一行。
% 此属性可选,其默认值为使用|\title|命令所设置的论文标题
\abstracttitlea{标题第一行}
% 设置中文摘要页面的论文标题及副标题的第二行。
% 此属性可选,其默认值为空白
\abstracttitleb{标题第二行用于长标题换行}

%%%%%%%%%%%%%%%%%%%%%%%%%%%%%%%%%%%%%%%%%%%%%%%%%%%%%%%%%%%%%%%%%%%%%%%%%%%%%%%
% 设置论文的英文摘要

% 设置英文摘要页面的论文标题及副标题的第一行。
% 此属性可选,其默认值为使用|\englishtitle|命令所设置的论文标题
\englishabstracttitlea{englishabstracttitlea}
% 设置英文摘要页面的论文标题及副标题的第二行。
% 此属性可选,其默认值为空白
\englishabstracttitleb{nglishabstracttitleb}

%%%%%%%%%%%%%%%%%%%%%%%%%%%%%%%%%%%%%%%%%%%%%%%%%%%%%%%%%%%%%%%%%%%%%%%%%%%%%%
%% 盲审命令,空白字段设置请看 .cls文件 \newcommand*{\blind}
%% 此外,请按照盲审要求自行去掉个人简历、致谢等页面中的个人信息
%\blind

%%%%%%%%%%%%%%%%%%%%%%%%%%%%%%%%%%%%%%%%%%%%%%%%%%%%%%%%%%%%%%%%%%%%%%%%%%%%%%%
\begin{document}

%%%%%%%%%%%%%%%%%%%%%%%%%%%%%%%%%%%%%%%%%%%%%%%%%%%%%%%%%%%%%%%%%%%%%%%%%%%%%%%

% 制作国家图书馆封面(博士学位论文才需要)
%\makenlctitle
% 制作中文封面
\maketitle
% 制作英文封面
\makeenglishtitle


%%%%%%%%%%%%%%%%%%%%%%%%%%%%%%%%%%%%%%%%%%%%%%%%%%%%%%%%%%%%%%%%%%%%%%%%%%%%%%%
% 开始前言部分
\frontmatter

%%%%%%%%%%%%%%%%%%%%%%%%%%%%%%%%%%%%%%%%%%%%%%%%%%%%%%%%%%%%%%%%%%%%%%%%%%%%%%%
% 论文的中文摘要
\begin{abstract}
\lipsum[1-2]

%通过改变链路中子流的个数,分配不同的数据流量给不同的链路。

% 中文关键词。关键词之间用中文全角分号隔开,末尾无标点符号。
\keywords{关键词1 \quad 关键词2 }
\end{abstract}

%%%%%%%%%%%%%%%%%%%%%%%%%%%%%%%%%%%%%%%%%%%%%%%%%%%%%%%%%%%%%%%%%%%%%%%%%%%%%%%
% 论文的英文摘要
\begin{englishabstract}
\lipsum[2]

%Rate adaptation can be implemented by adjusting the number of subflows on each path.

% 英文关键词。关键词之间用英文半角逗号隔开,末尾无符号。
\englishkeywords{keyword1\quad keyword2}
\end{englishabstract}

%%%%%%%%%%%%%%%%%%%%%%%%%%%%%%%%%%%%%%%%%%%%%%%%%%%%%%%%%%%%%%%%%%%%%%%%%%%%%%%
% 论文的前言,应放在目录之前,中英文摘要之后
%
\begin{preface}
\lipsum[1]
\vspace{1cm}
\begin{flushright}
作者\\
20xx年夏于南京大学
\end{flushright}

\end{preface}

%%%%%%%%%%%%%%%%%%%%%%%%%%%%%%%%%%%%%%%%%%%%%%%%%%%%%%%%%%%%%%%%%%%%%%%%%%%%%%%
% 生成论文目录
\tableofcontents

%%%%%%%%%%%%%%%%%%%%%%%%%%%%%%%%%%%%%%%%%%%%%%%%%%%%%%%%%%%%%%%%%%%%%%%%%%%%%%%
% 生成插图清单。如无需插图清单则可注释掉下述语句。
\listoffigures

%%%%%%%%%%%%%%%%%%%%%%%%%%%%%%%%%%%%%%%%%%%%%%%%%%%%%%%%%%%%%%%%%%%%%%%%%%%%%%%
% 生成附表清单。如无需附表清单则可注释掉下述语句。
\listoftables

%%%%%%%%%%%%%%%%%%%%%%%%%%%%%%%%%%%%%%%%%%%%%%%%%%%%%%%%%%%%%%%%%%%%%%%%%%%%%%%
% 开始正文部分
\mainmatter

%%%%%%%%%%%%%%%%%%%%%%%%%%%%%%%%%%%%%%%%%%%%%%%%%%%%%%%%%%%%%%%%%%%%%%%%%%%%%%%
% 学位论文的正文应以《绪论》作为第一章
\chapter{绪论}\label{chapter_introduction}
\section{研究背景}
使用.bib文件管理参考文献引用,引用示例:\cite{BHR12}.\par
\lipsum[1]

\section{Section}\label{subsec:mptcp_conges}
\lipsum[1]

\section{Section}
\lipsum[1]




\section{论文结构}

\lipsum[1]
\begin{itemize}
\item 一级item
 \begin{itemize}
 \item 二级item
	\begin{itemize}
	\item 三级item

	\end{itemize}

 \end{itemize}
\item 一级item

\end{itemize}



\chapter{相关工作}

\lipsum[1]

\chapter{chapter}

\lipsum[1]

\section{section}\label{sec:rate}
\lipsum[1]
\subsection{subsection}
\lipsum[1]

\begin{figure}[htbp]
  \centering
  \includegraphics[width=0.6\linewidth]{./figure/github.jpg}
  \caption{单图示例}
  \label{fig:system}
\end{figure}

\chapter{算法}

\begin{algorithm}[htbp]
  \caption{算法名字}
  \label{alg:alg1}
  \begin{algorithmic}[1]
        \REQUIRE 这是输入
        \ENSURE 这是输出
        \WHILE {flag}
		      \STATE 这是语句
        \ENDWHILE
  \end{algorithmic}
\end{algorithm}

\chapter{实验验证}

实验硬件设备如图\ref{img:1}所示。
\begin{figure}[htbp]
\begin{minipage}[t]{0.5\textwidth}
\centering
\includegraphics[width=0.8\textwidth]{./figure/github.jpg}
\caption{实验硬件设备总览}
\label{img:1}
\end{minipage}
\begin{minipage}[t]{0.5\textwidth}
\centering
\includegraphics[width=0.8\textwidth]{./figure/github.jpg}
\caption{实验测量示意图}
\label{img:2}
\end{minipage}
\end{figure}

图\ref{fig:sub}所示子图\ref{subfig:a}和子图\ref{subfig:b}。
\begin{figure}[H]
	\begin{subfigure}{.5\textwidth}
		\centering
		\includegraphics[width=0.8\textwidth]{./figure/github.jpg}
		\caption{子图}
		\label{subfig:a}
	\end{subfigure}
	\begin{subfigure}{.5\textwidth}
		\centering
		\includegraphics[width=0.8\textwidth]{./figure/github.jpg}
		\caption{子图}
		\label{subfig:b}
	\end{subfigure}
\caption{子图样例}
\label{fig:sub}
\end{figure}

\chapter{总结与展望}
\lipsum[1]



%%%%%%%%%%%%%%%%%%%%%%%%%%%%%%%%%%%%%%%%%%%%%%%%%%%%%%%%%%%%%%%%%%%%%%%%%%%%%%%
% 致谢,应放在《结论》之后
\begin{acknowledgement}
%thanks
\lipsum[1]

\end{acknowledgement}

%%%%%%%%%%%%%%%%%%%%%%%%%%%%%%%%%%%%%%%%%%%%%%%%%%%%%%%%%%%%%%%%%%%%%%%%%%%%%%%




% 参考文献。应放在\backmatter之前。
% 推荐使用BibTeX,若不使用BibTeX时注释掉下面一句。
%\nocite{*}
\bibliography{sample}


% 附录,必须放在参考文献后,backmatter前
\appendix
\chapter{附录代码}\label{app:1}
\section{main函数}
\begin{lstlisting}[language=C]
int main()
{
	return 0;
}
\end{lstlisting}

%%%%%%%%%%%%%%%%%%%%%%%%%%%%%%%%%%%%%%%%%%%%%%%%%%%%%%%%%%%%%%%%%%%%%%%%%%%%%%%
% 书籍附件
\backmatter
%%%%%%%%%%%%%%%%%%%%%%%%%%%%%%%%%%%%%%%%%%%%%%%%%%%%%%%%%%%%%%%%%%%%%%%%%%%%%%%
% 作者简历与科研成果页,应放在backmatter之后
\begin{resume}
% 论文作者身份简介,一句话即可。
\begin{authorinfo}
\noindent 韦小宝,男,汉族,1985年11月出生,江苏省扬州人。
\end{authorinfo}
% 论文作者教育经历列表,按日期从近到远排列,不包括将要申请的学位。
\begin{education}
\item[2007年9月 --- 2010年6月] 南京大学计算机科学与技术系 \hfill 硕士
\item[2003年9月 --- 2007年6月] 南京大学计算机科学与技术系 \hfill 本科
\end{education}
% 论文作者在攻读学位期间所发表的文章的列表,按发表日期从近到远排列。
\begin{publications}
\item Xiaobao Wei, Jinnan Chen, ``Voting-on-Grid Clustering for Secure
  Localization in Wireless Sensor Networks,'' in \textsl{Proc. IEEE International
    Conference on Communications (ICC) 2010}, May. 2010.
\item Xiaobao Wei, Shiba Mao, Jinnan Chen, ``Protecting Source Location Privacy
  in Wireless Sensor Networks with Data Aggregation,'' in \textsl{Proc. 6th
    International Conference on Ubiquitous Intelligence and Computing (UIC)
    2009}, Oct. 2009.
\end{publications}
% 论文作者在攻读学位期间参与的科研课题的列表,按照日期从近到远排列。
\begin{projects}
\item 国家自然科学基金面上项目``问题研究''
(课题年限~2010年1月 --- 2012年12月),负责相关问题的研究。
\end{projects}
\end{resume}

%%%%%%%%%%%%%%%%%%%%%%%%%%%%%%%%%%%%%%%%%%%%%%%%%%%%%%%%%%%%%%%%%%%%%%%%%%%%%%%
% 生成《学位论文出版授权书》页面,应放在最后一页
\makelicense

%%%%%%%%%%%%%%%%%%%%%%%%%%%%%%%%%%%%%%%%%%%%%%%%%%%%%%%%%%%%%%%%%%%%%%%%%%%%%%%
\end{document}
