\documentclass[master, macfonts]{njuthesis}

\newcommand*{\CJKunderlinecolor}{\color{black}}
\newcommand*{\CJKunderline}[1]{\uline{#1}}

%%%%%%%%%%%%%%%%%%%%%%%%%%%%%%%%%%%%%%%%%%%%%%%%%%%%%%%%%%%%%%%%%%%%%%%%%%%%%%%
% set up labelformat and labelsep for subfigure 详见: http://www.latexstudio.net/archives/8652.html
\captionsetup[subfigure]{labelformat=simple, labelsep=space}
\usepackage{subcaption}
\usepackage{multirow}

% 反复出现的概念或名称
\newcommand{\wa}{WARDER}
\newcommand{\wasc}{WARDER-sc}
\newcommand{\wamc}{WARDER-mc}
\newcommand{\wawc}{WARDER-wc}

\newcommand{\sg}{SGUARD}
\newcommand{\cu}{CUSTODES}
\newcommand{\uc}{UCheck}
\newcommand{\di}{Dimension}
\newcommand{\am}{AmCheck}
\newcommand{\ca}{CACheck}

% 三个优化方法的专有名词
\newcommand{\scvp}{单个单元格的有效性属性}
\newcommand{\mcvp}{多个单元格的有效性属性}
\newcommand{\wcvp}{整类的有效性属性}

% 实验中出现的数据指称
\newcommand{\prd}{\textit{precision_d}}
\newcommand{\fm}{$F\text{-}measure$}
\newcommand{\fmd}{$F\text{-}measure_d$}
\newcommand{\fmc}{$F\text{-}measure_c$}

%%%%%%%%%%%%%%%%%%%%%%%%%%%%%%%%%%%%%%%%%%%%%%%%%%%%%%%%%%%%%%%%%%%%%%%%%%%%%%%
% 设置论文的中文封面

% 单行论文标题,不可换行
\titlea{电子表格中基于有效性属性的}
\titleb{单元格聚类和错误检测的技术研究}
% 论文作者姓名
\author{李达}
% 论文作者联系电话
\telphone{15651667030}
% 论文作者电子邮件地址
\email{njulida@outlook.com}
% 论文作者学生证号
\studentnum{MG1633116}
% 论文作者入学年份(年级)
\grade{2021}
% 导师姓名职称
\supervisor{xxx教授}
% 导师的联系电话
\supervisortelphone{00000000000}
% 论文作者的学科与专业方向
\major{计算机科学与技术}
% 论文作者的研究方向
\researchfield{软件测试}
% 论文作者所在院系的中文名称
\department{计算机科学与技术系}
% 论文作者所在学校或机构的名称。此属性可选,默认值为``南京大学''。
\institute{南京大学}
% 论文的提交日期,需设置年、月、日。
\submitdate{2021年x月xx日}
% 论文的答辩日期,需设置年、月、日。
\defenddate{2021年x月xx日}
% 论文的定稿日期,需设置年、月、日。此属性可选,默认值为最后一次编译时的日期,精确到日。
\date{2021年x月xx日}

%%%%%%%%%%%%%%%%%%%%%%%%%%%%%%%%%%%%%%%%%%%%%%%%%%%%%%%%%%%%%%%%%%%%%%%%%%%%%%%
% 设置论文的英文封面

% 论文的英文标题,不可换行
\englishtitle{A Research on Spreadsheet Cell Clustering and Defect Detection}
% 论文作者姓名的拼音
\englishauthor{LI Da}
% 导师姓名职称的英文
\englishsupervisor{Professor xx xx}
% 论文作者学科与专业的英文名
\englishmajor{Computer Science and Technology}
% 论文作者所在院系的英文名称
\englishdepartment{Department of Computer Science and Technology}
% 论文作者所在学校或机构的英文名称。此属性可选,默认值为``Nanjing University''。
\englishinstitute{Nanjing University}
% 论文完成日期的英文形式,它将出现在英文封面下方。需设置年、月、日。日期格式使用美国的日期
% 格式,即``Month day, year'',其中``Month''为月份的英文名全称,首字母大写;``day''为
% 该月中日期的阿拉伯数字表示;``year''为年份的四位阿拉伯数字表示。此属性可选,默认值为最后
% 一次编译时的日期。
\englishdate{xx xx, 2021}

%%%%%%%%%%%%%%%%%%%%%%%%%%%%%%%%%%%%%%%%%%%%%%%%%%%%%%%%%%%%%%%%%%%%%%%%%%%%%%%
% 设置论文的中文摘要

% 设置中文摘要页面的论文标题及副标题的第一行。
% 此属性可选,其默认值为使用|\title|命令所设置的论文标题
\abstracttitlea{标题第一行}
% 设置中文摘要页面的论文标题及副标题的第二行。
% 此属性可选,其默认值为空白
\abstracttitleb{标题第二行用于长标题换行}

%%%%%%%%%%%%%%%%%%%%%%%%%%%%%%%%%%%%%%%%%%%%%%%%%%%%%%%%%%%%%%%%%%%%%%%%%%%%%%%
% 设置论文的英文摘要

% 设置英文摘要页面的论文标题及副标题的第一行。
% 此属性可选,其默认值为使用|\englishtitle|命令所设置的论文标题
\englishabstracttitlea{englishabstracttitlea}
% 设置英文摘要页面的论文标题及副标题的第二行。
% 此属性可选,其默认值为空白
\englishabstracttitleb{nglishabstracttitleb}

%%%%%%%%%%%%%%%%%%%%%%%%%%%%%%%%%%%%%%%%%%%%%%%%%%%%%%%%%%%%%%%%%%%%%%%%%%%%%%
%% 盲审命令,空白字段设置请看 .cls文件 \newcommand*{\blind}
%% 此外,请按照盲审要求自行去掉个人简历、致谢等页面中的个人信息
%\blind

%%%%%%%%%%%%%%%%%%%%%%%%%%%%%%%%%%%%%%%%%%%%%%%%%%%%%%%%%%%%%%%%%%%%%%%%%%%%%%%
\begin{document}

\maketitle
\makeenglishtitle

\abstracttitlea{电子表格的单元格聚类与缺陷检测优化技术研究}

\begin{abstract}

终端用户编程获得了广泛关注,电子表格无疑是终端用户编程中最流行的编程范式之一。
电子表格通常被用来存储、计算和分析用户数据,帮助用户进行数据计算和决策计划。
由于电子表格中可能存在大量的数据,其组织方式也多种多样,因此其中可能潜藏着各种类型的缺陷。
近年来,多家金融投资公司因为电子表格中潜藏的错误,造成了巨大的经济损失。

利用自动化技术帮助终端用户检测电子表格是否存在错误,尤为重要。
其中,如何自动化检测与公式相关的电子表格缺陷得到了最多的关注。
近二十年来,研究者们提出了各种各样的方法来检测公式缺陷。
目前检测效果较好的主流方法可以依据它们的技术思路分成两类:基于规则匹配和基于聚类/学习算法的缺陷检测技术。
前者通常能够依据预先设计的精准规则,以较高的准确率检测到特定类型的单元格缺陷,但相对而言检测的召回率不高;
而后者能够根据公式特征和单元格的布局特征对单元格进行聚类,大幅提高检测的召回率,但由于电子表格本身的复杂性,后者对公式的结构和特征提取不够准确,导致最终的检测精度不高。

文本基于已有工作 \cu 提出了一个新技术 \wa 。
\cu 具有自适应性的学习能力,可以进行跨表格、跨布局风格的学习,但同时它存在将不相关的单元格吸纳进单元格类中的不足之处。
通过对单元格类的构建过程进行自下而上的有效性检验和优化,\wa 能够足够精准地过滤掉和单元格类无关的单元格或不符合要求的整个单元格类,最终提升单元格聚类和缺陷检测的效果。

综上所述,本文的主要贡献有如下三点:
\begin{enumerate}
    \item 提出了\wa ,一种新的电子表格的单元格聚类和缺陷检测优化技术。
    \wa 聚焦于单元格自身、单元格之间关系和整个单元格类的有效性检验,通过提升 \cu 的单元格聚类效果,进而提升缺陷检测的效果;

    \item 使用一个被相关工作广泛应用的电子表格基准测试集(采样自 EUSES 数据集)和一个大规模的电子表格语料库 VEnron2 ,对 \wa 进行充分的实验评估和案例研究,也对比了当下最流行的其它电子表格缺陷检测技术,验证了 \wa 在提升聚类和检测精度方面的优势;

    \item 实现了两个工具:一个 Excel 插件 EGuard,允许终端用户在 Excel 软件中查看 \wa 对当前打开的工作表的检测结果和执行信息,帮助终端用户迅速检查和修复工具报告出的缺陷;另一个可视化集成工具 \sg ,可对比多种主流技术的检测结果并统一展示。
\end{enumerate}

\keywords{电子表格检测;单元格聚类;缺陷检测;有效性检验;软件工程}

\end{abstract}
\begin{englishabstract}

End-user programming has gained widespread attention and applications, among which, spreadsheets are undoubtedly one of the most popular programming paradigms in end-user programming.
Spreadsheets are widely accepted by users and are widely used worldwide. 
They are usually used to store, calculate, and analyze user data. 
For example, software such as MS Excel and Google Sheet provides users with the convenience of storing data and calculation formulas. 
However, behind the widespread application of spreadsheet programming paradigms, there are indeed various types of errors and defects hidden in every spreadsheet. 
In the past ten years, major commercial companies have caused millions of dollars in economic losses because they made mistakes when using spreadsheets.

Therefore, the use of software engineering technology to help end users detect potential defects in spreadsheets has received widespread attention. 
In the past two decades, a variety of technologies has been proposed to detect various types of explicit errors and implicit defects in spreadsheets. 
With the iterative update of spreadsheet software, explicit errors have been effectively tested, and how to detect more complex implicit defects has been more extensively studied. 
Representative existing work can be roughly divided into two categories based on technical ideas: rule-based verification and defect detection technology based on clustering/learning algorithms. The former (representative work CACheck) can usually detect specific types of cell defects with high accuracy according to the precise rules of the design, but the recall rate of the detection is not high relative to the latter; while the latter (representative work CUSTODES) clustering cells according to the syntactic structures and semantic features of cell formulas, thereby greatly improving the detection recall rate. However, compared with the former, the complexity of the spreadsheet itself leads to insufficient preparation for the structure and feature extraction of cell formulas. As a result, the final detection accuracy is not high. 

Taking into account the above shortcomings, in this work we propose a novel technology WARDER, based on CUSTODES's innate adaptive cross-table, cross-layout style learning ability, but at the same time improve it in the relevant and irrelevant cells mixing into the same cluster. The main contributions of this article are as follows:

\begin{enumerate}
    \item Proposing the spreadsheet defect detection technology WARDER, focusing on the validity properties of the cells and the clusters formed to refine the cell clustering process of CUSTODES, and effectively improve the effectiveness of the spreadsheet defect detection as a whole.
    \item Using an existing spreadsheet benchmark and a large-scale spreadsheet corpus, WARDER was fully tested and evaluated, and compared with other most popular spreadsheet defect detection technologies. 
    \item Two visualization tools are implemented: a Java-based visualization tool SGuard, which can compare the detection results of a variety of mainstream technologies; another JavaScript-based Excel plug-in EGuard, using that, you can see instantly and step by step see the detection results of the current spreadsheet, which is convenient for users to confirm and repair formula defects immediately.
\end{enumerate}

\englishkeywords{Spreadsheet Testing, Cell Clustering, Defect Detection, Validity Checking, Software Engineering}

\end{englishabstract}

\tableofcontents
\listoffigures
\listoftables

\mainmatter

\chapter{绪论}\label{introduction}
本章介绍与电子表格缺陷检测相关的研究背景,然后从电子表格缺陷检测的研究现状、研究思路和研究贡献三方面介绍本文工作,最后说明余下章节的组织结构。


\section{研究背景}
\subsection{电子表格的流行}
电子表格已经成为终端用户编程中最流行的范式,在世界范围内被广泛使用。
电子表格的易用性(所见即所得的编程范式)是它如此流行的主要原因之一。
几乎每个使用计算机办公的人都会接触到电子表格应用,例如 Microsoft Excel,Google Sheets 和 Apple Numbers。
众多行业的工作人员,例如精算师,销售人员和行政人员,都是重度依赖电子表格的终端用户\cite{scaffidi2005estimating},他们经常通过精心编制的电子表格进行数据分析和商业决策。

\subsection{电子表格的使用风险}
然而,终端用户对电子表格的数据可靠性也一直缺乏必要的重视。
电子表格的易用性使得用户不再需要通过专业训练来掌握它的使用方法,而是让用户通过边用边学的方式来掌握它。
因此,很多终端用户并不清楚伴随电子表格而来的使用风险。
这类风险使得电子表格中容易产生数值或公式错误,严重时会导致巨额经济亏损\footnote{http://www.eusprig.org/horror-stories.htm},例如:

\begin{itemize}
    \item 2012 年,金融投资公司 JP 摩根使用电子表格计算价值-风险模型,其中隐含的电子表格公式方面的使用错误导致决策失效,损失约 40 亿美元;
    \item 2016 年,加拿大电力公司 TransAlta 在电子表格中的复制粘贴不慎引入错误,在与美国签订电力传输保障的合同时,额外支付了约 2,400 万美元。
\end{itemize}

多个实证调研 \cite{panko2016we,powell2009impact} 表明:即使在商业化管控的电子表格环境中,依然缺乏有效的数据可靠性保障技术。
因为终端用户大多不是受过专业训练的程序员,缺乏良好的编程规范,他们更加关心如何把工作里的具体计算或分析任务快速完成。
因此,终端用户的整个编程过程缺少对预期功能和系统的建模、封装和测试等传统的软件质量保障思路和相应的支撑技术。
在现实场景下,多达 90\% 的电子表格都包含一些数值或公式计算上的错误\cite{rajalingham2008classification},难免造成相关公司的经济损失。


\section{本文工作}
\subsection{研究问题和现状}
电子表格中的错误集中发生在公式计算环节,并暴露在数值单元格和公式单元格中,现有的实证研究\cite{panko2010revising}表明公式单元格中的潜在问题常常是电子表格错误的根源。
在本文中,我们将公式单元格中的错误称为电子表格中的缺陷\footnote{在本文的剩余部分如无特殊说明,电子表格的错误(error)和缺陷(defect)是相同概念,都泛指电子表格中和公式有关的单元格错误,在某些研究工作中也用潜在错误(smell)来指代此类错误},并且关注于针对这类缺陷进行检测的有效技术。

自动化检测电子表格的缺陷并非易事。
第一,电子表格通常由终端用户维护,维护过程中包含非专业的操作,例如出于各种现实目的,用纯数值改写整个公式或者子公式,在电子表格中埋下有缺陷的单元格。
第二,追踪电子表格的修改历史在电子表格的使用场景下难以实现,因为终端用户并不习惯使用类似 Git 和 SVN 的版本控制软件,这使得我们难以直接诊断出电子表格缺陷是在何时何处引入的。
第三,电子表格单元格之间的计算依赖关系有时是隐晦的,这使得检测电子表格的缺陷变得更为困难。

为了应对这些挑战,研究者们已经提出了各式各样的电子表格缺陷检测技术:

\begin{itemize}
    \item 早期技术依赖于表头信息来推导公式引用中的单位或类型不一致(如\uc \cite{abraham2007ucheck} 和 \di \cite{chambers2009automatic});
    \item 目前一类主流技术利用矩形布局的特性来识别丢失公式或者不一致的公式缺陷(如\am \cite{dou2014spreadsheet} 和 \ca \cite{dou2017cacheck});
    \item 目前另一类主流技术使用自适应学习的方法来检测跟公式相关的单元格缺陷(如\cu \cite{cheung2016custodes}、Melford\cite{singh2017melford} 和 ExceLint \cite{Barowy2018excelint})。
\end{itemize}

然而,这些电子表格缺陷检测技术仍有各自的不足之处:

\begin{itemize}
    \item 对第一类来说(基于类型推导的技术),它们的推导相对粗糙,并且关注于非常有限的缺陷种类,导致缺陷检测的精度和召回率都比较低\cite{zhang2017effectively};
    \item 对第二类来说(基于特定模式或规则匹配的技术),它们倚仗的模式或规则通常是严格且精确的,专注于电子表格中具有某种特性的缺陷类型,因而往往能够获得较高的检测精度(比如\am 的 71.9\% 和 \ca 的 86.8\% \cite{dou2017cacheck}),但无法适应不同电子表格中多变的布局和排版,导致相对有限的召回率(比如\am 的 60.3\% 和 \ca 的 71.0\% \cite{dou2017cacheck});
    \item 对第三类来说(基于学习的技术),由于它们先天的自适应学习能力,此类技术得到较为广泛的应用。以 \cu \cite{cheung2016custodes} 为例,因为它被认为是当前“最好的自动化缺陷检测工具”\cite{Barowy2018excelint}。\cu 提取电子表格中的公式特征(强特征)对单元格进行分类,同时通过在电子表格中提取多种单元格的布局特征和隐式计算特征(弱特征),将缺少强特征但含有相似弱特征的单元格吸纳到单元格类中,最后根据公式语法树的差异,识别出类中有缺陷的单元格。该方法极大地提升了检测的召回率(达到 80\% \cite{cheung2016custodes}),但同时把一些无关的单元格也吸纳到类中,进而牺牲了一些检测精度(仅 65\% \cite{cheung2016custodes})。 
\end{itemize}

\subsection{研究思路}
考虑到上述技术的不足之处,本文提出了一个新颖的技术 \wa \footnote{\wa 的命名方式遵循了 \cu 的风格},基于 \cu 的自适应的跨表格和跨布局风格的学习能力,同时改善它把不相关的单元格吸纳进单元格类中的不足。
本文的核心观察是:在对单元格进行聚类时,如果偶然地将不相关的单元格混入到类中,必会极大地影响缺陷检测的有效性(例如牺牲检测的精度)。
因此,在 \wa 中本文的主要突破点是关注如何提升电子表格的聚类准确性,通过尽可能地排除那些不相关的单元格使得聚类结果可靠性更高,同时也保留住 \cu 技术的长处。

自下而上地分析,我们认为提升聚类准确性的方法有三层设计思路:
\begin{itemize}
    \item 针对单元格自身的有效性检验:当把数值单元格吸纳进类中时,我们检验它自身是否是有效的。当用一个合适的公式来替换当前数值单元格里的纯数值时,该单元格应该是可有效计算的。否则,如果枚举所有合适的公式后,每一次该单元格都无法有效计算(例如引用一个错误的单元格类型或者使用无效的单元格引用),这个单元格就无法被有效计算得到,应当被阻止吸纳进类中;
    \item 针对单元格之间关系的有效性检验:当把数值单元格吸纳进类中时,它们不应该破坏类中已有单元格之间的共同属性。例如类中已有单元格的引用单元格集合没有交集,如果被吸纳的单元格当它的值被类中某个合适的公式替换后,存在某个单元格和它的引用集合却出现了非空交集,这个单元格就是无效的,应当被阻止吸纳进类中;
    \item 针对整个类的有效性检验:关注整个类层面而不是单个单元格层面的检验。原则上,每个类是用来囊括具有相似计算语义的单元格,那么每个类应该存在一个统一的公式能够符合类中多数单元格的计算语义。如果该类无法满足该属性,即该类具有多种计算语义,它就应该被删除或拆分。
\end{itemize}

% 这里描述比较混乱?
有了这三组自下而上的有效性检验方法,\wa 相较于它的前身 \cu 以及其它电子表格缺陷检测技术,展现出了明显的优势。
在\cu 使用的采样自 EUSES \cite{fisher2005euses} 数据集的 291 个工作表的基准测试集中,\wa 的聚类结果与 \cu 对比,取得了显著的提升,提升了 79.8\% 的工作表的单元格聚类效果,尤其在聚类精度方面取得了 0.3-94.6\% 的提升(平均 20.7\%),在召回率方面仅有 2.4\% 的牺牲。
\wa 通过更加有效的单元格聚类,对比于\cu,使得缺陷检测的精度提升了 23.1\%。
结合召回率来看,最终将 $F\text{-}measure$ 值从 0.71 提升至 0.79。
\wa 也比其它电子表格缺陷检测技术在缺陷检测方面有较大的提升,取得了平均 87.8\% 的精度表现和 71.9\% 的召回率表现,对比于其他技术的 0.5-72.4\% 和 0.1-68.4\%。
另外,我们在一个更大的电子表格数据集 VEnron2 \cite{xu2017spreadcluster} 上进行案例研究,虽然每个缺陷检测技术的检测效果都有大幅下降,但相对而言,\wa 依旧展现出了自身优势,在检测精度方面达到 41.3\%,对比于其它技术的16.3-34.3\%。

\subsection{研究贡献}
结合上述分析,本文的主要贡献有如下三点:
\begin{enumerate}
    \item 我们提出了电子表格缺陷检测技术 \wa ,利用单元格自身、单元格之间关系和整个单元格类的有效性属性来提升 \cu 技术的单元格聚类效果,最终也提升了电子表格缺陷检测的有效性;
    \item 我们使用采样自 EUSES 数据库的电子表格基准测试集和大规模的电子表格语料库 VEnron2 这两个测试对象,与目前主流的其它电子表格缺陷检测技术进行对比,实验结果表明 \wa 技术能够有效地提升聚类和检测的有效性;
    \item 我们实现了两个可视化工具:一个 Excel 插件 EGuard,可在操作 Excel 工作表的同时,看到 \wa 对当前打开的工作表的检测结果和执行信息,使得终端用户能够迅速检查和修复工具报告出的缺陷;另一个集成检测工具 \sg ,可对比多种主流技术的检测结果并统一展示。
\end{enumerate}


\section{组织结构}
本文的剩余章节组织如下:
第2章为相关工作综述,将对和本文相关的工作按照各自类别总结并介绍。
第3章为研究问题的形式化规约,将对本文所研究的问题进行形式化描述。
第4章为系统设计,将对解决本文研究问题所用的方法进行详细介绍。
第5章为系统实现和工具展示,将介绍本文的 \wa 检测方法的实现细节,并对两个可视化工具的使用进行简要说明。
第6章为实验评估,将阐述实验的设计思路和执行配置,并对实验结果进行分析比对,验证该方法的有效性。
第7章总结全文,并从本文的视角指出该方向的研究展望。
\chapter{相关工作综述}

绝大多数的电子表格的研究工作主要出现在两个研究领域,信息系统(IS)和计算机科学(CS),在管理审计、风险控制领域也偶有涉及\cite{creeth1985microcomputer,galletta1993empirical}。

在信息系统领域下的电子表格研究工作通常关注与终端用户密切相关、更加基础的部分(例如电子表格的错误类型、错误率),人工错误(比如用户交互、认知能力),和支撑工具的用户接受程度,以及有系统的实证评估的方法论问题相关\cite{reinhart2010growth,galletta1996spreadsheet,powell2008critical,howe2006factors,olson1987analysis,panko1998we}。
这些方面的考量应当作为设计自动化电子表格质量保障工具的研究基石,即如何让开发工具更易于使用,让终端用户更易于接受,在这方面此类工作具有显著意义。

在本章剩余部分中,我们从计算机科学(包含软件工程和程序语言社区)视角出发,关注电子表格开发过程的支撑工具,并对自动化电子表格的质量保障方法进行分类讨论。总得来说,根据它们在开发流程中的角色和用途,我们把各种电子表格质量保障技术分成两个大类:

\begin{itemize}
    \item \textit{避免错误}的技术用来帮助开发者从使用的起始阶段就能创建出不含错误的电子表格;
    \item \textit{发现和修复错误}的技术用来帮助用户检测错误并理解错误发生的原因。这类工具通常由开发者或者审计、复核人员在电子表格基本构建完成之后使用。
\end{itemize}

%% 这里可以对综述和实证研究进行一番讨论,然后在给出我们如此分类相关工作的理由



接下来,我们根据完成上述两个用途的具体技术进行更细粒度的分类,将和本文相关的工作分成如下角度来讨论,即模型驱动的电子表格开发方法、电子表格的测试方法、电子表格的错误定位、检测和修复和电子表格的辅助开发工具。
这里提醒读者,很多电子表格工作本身就隶属于多个类别,比如既属于缺陷检测,也属于缺陷修复,同时也包含可视化工具,这里我们以该技术的主要贡献点来进行单一分类,而不是多次在不同分类中提及。

\section{模型驱动的电子表格开发方法}

对比与之前提到的检测和测试方法,基于模型驱动的方法并不是设计用来帮助终端用户发现潜在的错误,而是把提升电子表格的质量和结构清晰度,并预先防止错误注入放在首位。
类似于一般软件工程领域的模型驱动方法,这类技术的核心想法是在电子表格的开发过程中引入额外的抽象层。
这个额外的抽象层引入了关于问题的更多抽象概念,并在开发者的内在意图和电子表格的真实实现中间充当了沟通的桥梁。
原本在商业化的电子表格系统中日益扩大的这层语义鸿沟\cite{luckey2012systematic}得到了有效缓解。

这类抽象的电子表格模型通常出现在开发过程的两个阶段:
\begin{itemize}
    \item 它们被用作“代码生成器”的形式化描述。在这个场景下,电子表格的一部分从模型中自动生成,因此减少了人工操作错误的风险;
    \item 它们也被用于从已有的电子表格中恢复出底层的概念结构,这类似于一般软件工程中逆向工程方法。
\end{itemize}

\textbf{声明式和面向对象的模型:}
Isakowitz 等人\cite{isakowitz1995toward}是最早提出从建模角度考虑电子表格程序的。
他们的核心假设是电子表格程序可以看做物理和逻辑的两个部分,物理部分就是单元格的公式和数值,逻辑部分就是描述电子表格功能的一系列函数关系。
如图\ref{..}所示,而逻辑这部分可以从给定的电子表格中自动提取出来,并用领域特定语言表示出来。
同时,该系统也能够从这样的逻辑规约中合成电子表格。

Paine 等人\cite{ireson1997model,paine2008ensuring}也提出了类似的电子表格程序的面向对象概念。
在 Model Master 方法中,电子表格以声明式的方式表达为文本程序。
这些程序通过编译器处理,随即从规约中生成电子表格。
电子表格的逻辑以类的形式表达,其中包括属性和计算逻辑。
该建模语言中也韩版许多特征(如继承性,多维数组)来支持表格化的计算。
通过该方法也可以逆向使用,从给定的电子表格中提取出对应的逻辑模型,进而用于发现电子表格中的特定计算结构\cite{paine2008spreadsheet}。

Paine 等人\cite{paine2005bringing,paine2008rapid}后续提出了另一种声明式建模语言。
Excelsior 是一个电子表格开发系统,构建在 Prolog 上的编程语言,同时针对 Excel 设计了模块化和可重用的规约表达方式。


\textbf{可视化的电子表格模板:}
对比与 Paine 等人的工作,Erwig 等人\cite{erwig2004gencel,erwig2005automatic,abraham2005goal}提出依赖于可视化的基于模板的方法来刻画电子表格底层模型的特定方面。
在他们的 Gencel 方法中一个“模板”被用来特指电子表格中重复的区域。
图\ref{.}展示了一个模板规约的例子。
模板的设计可以用类似于 MS Excel 的可视化方式完成。
在图\ref{..}中,$B$、$C$和$D$列下面的内容被标记为可重复的。
这类可重复的区域用列上和行上的两个省略号"..."和两条分隔带来表达。

类似于 Paine 等人的工作,电子表格也可以从模型中自动生成。相反地,基于模板的方法也支持逆向工程过程,能够使用一些启发式方法从给定的电子表格中自动提取出存在的模板\cite{abraham2006inferring}。

后来,Engels 等人\cite{engels2005classsheets,cunha2010automatically}结合基于模板的方法和面向对象的概念模型提出了 ClassSheet 的概念,突破了基于模板方法只能刻画电子表格的“词法特征”的局限性,使用 ClassSheet 概念能够更完整地刻画整个表格的“语义特征”。
许多基于基础 ClassSheet 方法的改进方法被陆续提出\cite{luckey2012systematic,cunha2011type,cunha2011embedding,cunha2012bidirectional}。比如 Luckey 等人\cite{luckey2012systematic}处理了模型演化和如何使得这类更新能够自动转换到已经生成的电子表格中的问题,以便更好地支撑整个工程过程。

Hermans等人\cite{hermans2010automatically}提出了另一个不同的可视化方法来重构底层面向对象模型。
该方法基于典型模式库,通过二维的解析和模式匹配算法来尝试定位电子表格中的各类模式。
最终的模式被转换成 UML 类图,被用于更好地理解和提升当前的电子表格。

\textbf{关系型模型:}
电子表格的主要原则之一:数据以表格的形式组织。
一个明显的获得抽象表格结构模型的方法就是借鉴关系型数据库的设计和方法。
以构建高质量和零错误的表格为目的,Cunha 等人\cite{cunha2009spreadsheets}提出了从电子表格中提取关系型数据库范式的方法。
该方法的主要优点是获得更加模块化,没有数据冗余,并预防错误的数据输入的电子表格模型\cite{cunha2009discovery,cunha2012relational}。



\section{电子表格的测试方法}

在专业的软件开发流程中,系统测试对于保障软件制品的高质量至关重要。
通常这类测试活动既有开发者,也有专门的测试者介入。
但因为非专业的电子表格用户通常没有软件工程的思维和实践经验,对应的测试过程通常是不系统且散乱的。

考虑到电子表格即时反馈的特质,测试过程通常仅通过输入一组测试用例,然后检查对应的中间单元格和最终的汇总单元格是否产生了预期输出。
同时,商业电子表格工具,如 MS Excel,并不提供任何特定的机制帮助用户存储这些测试用例或进行回归测试。
而且,这类工具通常也不会帮助用户评估是否已经进行了足够的测试。
下面,我们回顾一下那些旨在将标准软件测试的概念,想法和工具移植到电子表格开发过程的相关工作。

\textbf{测试完备性和测试用例管理:}
1997 年,Rothermel 等人\cite{rothermel1997testing,rothermel1998you,rothermel2001methodology}提出了针对电子表格的测试方法,被称为“所见即所测”(What You See Is What You Test,简记为 WYSIWYT)。
在电子表格构建期间,用户交互性地对当前给定输入下的一些派生出的单元格的值标记为“正确的”。
基于这些测试,系统自动判定该电子表格的被测试程度。
这个判定过程依赖于一个测试完备性准则,该准则基于电子表格的抽象模型,一种特定的“定义-使用”关系和动态执行轨迹。
后来,相继提出了一些优化方法,例如扩展到更大的同质性电子表格,增加对递归的支持,或是处理测试用例重用的问题\cite{burnett1999scaling,burnett2001visually,burnett2002testing,fisher2002automated,fisher2006scaling,randolph2002generalised}。

\textbf{自动化测试用例生成:}
在使用 WYSIWYT 时,电子表格用户会受到关于表格被测试程度的反馈,但用户人必须要手动给出测试用例。
为了在这个用例生成过程中给予用户帮助,Fisher 等人\cite{fisher2002automated,fisher2006integrating}提出了测试用例自动化生成技术。
主要通过两种方法来生成新的测试用例。
一种是随机方法,随机地生成值并检查是否整个执行使用到了目前没有被测试过的“定义-使用”对的路径,有点像传统软工中的测试路径覆盖。
另一种是目标导向的方法,以尚未测试过的“定义-使用”对为目标,尝试修改输入值来覆盖它,该过程可以迭代进行。

在 Abraham 等人的工作\cite{abraham2006autotest}中,AutoTest 工具实现了自动化测试用例生成的不同策略,采用约束求解的方式来搜索导致预期的“定义-使用”对能够执行的单元格值。该方法能够为所有可行的“定义-使用”对生成测试用例,相比于 Fisher 等人的方法\cite{fisher2006integrating},AutoTest 工具更加有效且高效。

\textbf{基于断言的测试:}
另一类对于用户来说非常不同的测试方法,基于断言的测试技术\cite{burnett2003end,wilson2003harnessing,beckwith2002reasoning},同样可以用来保障电子表格的质量。
以 Burnett 等人的工作\cite{burnett2003end}为例,电子表格领域的断言对应于以布尔表达式的形式限定允许使用的单元格值的前置条件和后置条件语句。
这些断言由终端用户通过一个相应的面向用户的工具来提供,该工具能够自动化检查各个断言并通过电子表格中的数据流进行受限的传播。
当断言和某个单元格值发生冲突时,用户会受到相应的问题提醒和信息反馈。

\textbf{测试驱动的电子表格开发:}
除了测试用例管理和生成的单个技术,McDaid 等人\cite{mcdaid2008test}研究了将软件工程领域获得广泛关注的测试驱动开发原则(test-driven development)是否适合应用到电子表格开发过程。
依照这一原则,用户预先编写符合预期的电子表格功能的测试用例,之后再逐步完成功能的实现,直到完全通过测试为止。整个编写测试用例,在实现相应功能的开发过程可以迭代多轮。
这种持续性的系统化测试方法应当有助于在最终完成阶段最小化错误的数量。


\section{电子表格的缺陷定位、检测和修复}

电子表格的编程范式使得终端用户容易犯错。
随着电子表格包含的数据量越来越大,依赖终端用户人工检查每个数值和公式的计算过程和结果是否正确,显得效率低下。
那么,相应的自动化电子表格错误(缺陷)的定位、检测和修复工作得到了学界广泛关注和研究。
这些方法通常可以分成两大类,取决于各个技术是否依赖于用户给定测试输入,或提供额外的标注信息。
在下面的技术讨论中,会单独指出每个技术在是否需要额外的用户协助。

\textbf{错误定位:}
早期的电子表格错误定位工作\cite{reichwein1999slicing,ruthruff2005interactive},基于计算轨迹的候选者排序策略,类似于传统程序分析中的基于频谱的错误定位方法。
他们首先提出将程序切片的概念应用到电子表格中,以消除不可能的错误候选者。
该类技术使用用户给定的辅助信息关于正确和不正确的单元格值,并把理论上对一个错误的单元格值有贡献的单元格标记为可能错误的。
一个单元格的公式如果对更多的标记为错误的值有贡献,那么很可能就是错误的。相反,对更多正确的单元格值有贡献,那么很可能就是正确的。如果一个单元格对错误的单元格值有贡献,但是它本身只由正确的单元格值计算而来,那么它的错误可能性就会减小。该类方法,就是依据这种想法,进行量化排序。
后续,Hofer 等人\cite{hofer2013empirical}使用更加形式化的方法,以相似性系数来计算电子表格单元格的错误可能性。

另一条错误定位的思路是把电子表格转换成基于约束的形式,使得关于异常值的原因定位的复杂推导变得可能。
Jannach 等人\cite{jannach2010toward}提出将电子表格错误定位问题转换成约束可满足性问题(CSP)\cite{tsang2014foundations}。
基于用户给出的测试用例和关于某些单元格的异常值信息,该方法使用基于模型诊断的原则来判定哪些单元格原则上可能是所所所所所是异常计算结果的真正原因。
后来,Jannach 等人\cite{jannach2016model}又提出了新的算法改进策略,帮助提升原方法的可扩展性。

类似的方法也得到了 Abreu 等人\cite{abreu2012constraint,abreu2012debugging}的采用。
Abreu 的方法尽管整体上相似,但技术实现上有差异。他们没有使用 Hitting-Set 算法\cite{reiter1987theory},而是把单个公式的正确性的推导直接编码成约束表达。因此他们利用了额外的布尔变量来代表每个公式的正确性。同时他们的方法可以同时运用于多个测试用例。

Hofer 等人\cite{hofer2013empirical}提出把一个轻量级的基于模型的渐渐调试技术,结合到他们的基于频谱的错误定位方法中。
他们建议使用从统计错误定位技术(SFL)得到的系数作为基于模型的调试过程的初始可能性值。

\textbf{错误检测:}%类型推导00-10 年 + 缺陷检测 11 年至今

\textbf{错误修复:}
基于修复的方法不仅向用户指出有潜在问题的公式,也额外给出修复方案,比如应该把错误的公式修改成何种具体的正确公式。
Abraham 等人\cite{abraham2005goal}做出了第一份自动化给出修改建议的工作 GoalDebug(以目标为导向的调试技术)。
该方法中,需要用户为错误的单元格给出预期的结果,通过递归修改单个公式,根据电子表格特定的修改推导规则反向传播到之前的公式中,进而尝试自动给出合适的修改建议。
能够获取预期结果的修改结果再根据启发式方法进行排序。
另一个改进上述提到的 GoalDebug 的方法 \cite{abraham2007goaldebug,abraham2008mutation} 更适合处理多种的电子表格错误类型。

%CACheck 就带有修复属性


\section{电子表格的辅助开发工具}

这类方法在开发和维护过程中给予用户帮助,包括帮助用户避免发生引用错误,处理异常行为,对电子表格的长期使用提供支撑(监控表格变动的工具,自动化重构的插件,处理公式重用的方法),以及自动帮助用户完成数据提取任务的程序合成工具。

\textbf{电子表格演化:}
电子表格通常经历各种变化,不巧的是,变化常常会引入错误。
FormulaDataSleuth\cite{bekenn2008reducing}是一个旨在帮助电子表格开发者在表格变动时,立刻检测到这类错误的工具。
一旦开发者已经声明了哪些数据单元格和区域应当被工具监控,系统就会自动监测许多类型的潜在问题。
对于已经定义好的数据区域,工具能够检测出空单元格,或者输入值拥有错误的数据类型,或者超出了预定义的数据范围。
对于被监控的公式单元格,偶然的公式重写以及引入范围变化导致引用了错误的单元格也会被识别出来。

理解给定的电子表格如何随着时间演化,并观察到不同版本之间的差异,对于在不同项目中重用电子表格至关重要。
Chambers 等人\cite{chambers2010sheetdiff}提出了 SheetDiff 算法,能够检测并可视化特定类型的不同版本的电子表格之间的显著差异。
后来,Harutyunyan 等人\cite{harutyunyan2012planted}提出一种基于动态规划的算法 RowColAlign,能够检测版本差异,能够有效处理上述贪心算法 SheetDiff 中存在的问题。

%最后应该提到徐良的工作

\textbf{重构:}
重构被定义为修改程序内部结构,但不修改功能性的过程\cite{o2010spreadsheet}。
重构在多个方面有助于电子表格质量提升。
比如,通过简化公式,使得整体更易理解,通过移除冗余公式,使得维护更加轻松,更少犯错。
电子表格领域的重构通常和行列重新排列有关,也就是电子表格的设计和布局的转换。
手工进行这类转换常常是耗时且易错的。

相应地,Badame等人\cite{badame2012refactoring}识别出电子表格中的七种重构策略,并给 MS Excel 提供了一个相应的重构插件 RefBook。
该插件自动检测需要重构的位置,并给用户提供重构过程的建议。
例如可能提供的重构建议有:将单元格常量化、添加守卫单元格、以及替换尴尬的公式等。

Harris等人\cite{harris2011spreadsheet}提出了一种根据用户给定的样例进行复杂表格转换的方法。该方法基于一个描述表格转换的语言TableProg,以及一个 ProgFromEx 算法。 
该算法需要用户提供几组示例,描述表格变化前后,电子表格对应的具体案例。
ProgFromEx 能够自动推导出一组程序能够完成这样的表格转换过程。
%这里可以提到更多的 PBE 工作

\textbf{复用:}
通常,复用已有的已经验证过的软件制品能够节约开发时间,避免犯错的风险,提升整体项目的可维护性\cite{ye2005reuse}。
这种复用思路也可以应用于电子表格领域。
独立的电子表格或者其中的部分通常可以在其他项目中重用。
在微观层面,甚至是独立的公式也常常在一个电子表格中被复用多次。
对于公式复用的标准做法是简单的复制粘贴该公式。
然而,改变最初的公式并不会改变它的副本,如果忘记对公式副本进行更行很容易导致错误。

电子表格程序中的复用问题得到了 Djang 等人\cite{djang1998similarity}和 Montigel等人\cite{montigel2002portability}的关注。
Djang等人\cite{djang1998similarity}沿用了面向对象思想中对继承概念的使用,来实现电子表格中的复用功能。
原则上,它允许用户在单个单元格和更粗粒度的层面上,以多个继承或相互继承的形式,声明电子表格单元格之间的依赖关系。
而 Montigel等人\cite{montigel2002portability}提出了电子表格语言 Wizcell。
其中,Wizcell 通过是粘贴/复制,拖/拽操作的语义过程更加明显,来缓解复用问题。
特别地,提出了四种这类操作的可能输出:要么被复制的公式再被复制一次,要么被复制的公式存在对原公式的引用;要么复制后的单元格中的公式引用了之前原始单元格集合中的某部分单元格,要么其引用随着副本和原来单元格之间的相对距离而对应改变。
Wizcell 语言相应地允许用户声明潜在语义,因此减少了由于复用引入错误的可能。
\chapter{研究问题的形式化定义}

在本章中,为了后续讨论的方便和统一,我们给出电子表格的编程模型,并解释比如单元格类和单元格缺陷这类核心概念。
为了简化表达,除额外说明之外,我们用\textit{数值单元格}专指那些单元格里的数值是直接给出而不是计算得来的单元格,用\textit{公式单元格}专指那些单元格里的数值是根据自身公式计算得来的单元格。

\section{电子表格的编程模型}

一个电子表格(更准确地说,一个电子表格中的一个工作表)可以被建模成一个集合,其中包含带有表达式的单元格,这些单元格使用二维地址来索引(一个行索引和一个列索引,比如 $B1$ 或者 $C2$)。
数值单元格和公式单元格的表达式分别通过纯粹的数值和公式来刻画。
一个公式通过\textit{单元格引用}来引用另一个单元格,单元格引用也是通过被引用的单元格地址来索引。
用$R$来表示单元格引用的集合,$EXP$来表示表达式的集合,$V$表示纯粹数值的集合。
一个单元格的表达式 $exp$ 要么是一个纯粹数值($v \in V$),一个单元格引用($r \in R$),或者一个引用一个或多个表达式的公式($\varphi $)。
电子表格中使用的函数包括基本的运算符(例如,“+”,“-”,“*”和“/”),以及大量电子表格软件中的内置函数(例如,$SUM$,$AVERAGE$和$MAX$)。
形式化地表达,一个单元格的表达式$exp$为:
\begin{definition}
    $ exp =\quad v\quad |\quad r\quad |\quad \varphi (exp_1,\dots,exp_n). $
\end{definition}

更进一步,我们定义一个获取引用函数$\sigma(exp)$,该函数返回一个集合,其中包含在一个单元格的表达式中使用到的所有单元格引用。形式化地表达如下:
\begin{definition}
$
\sigma(exp) = 
\left\{
    \begin{aligned}
       & \emptyset & exp \in V; \\
       & \{exp\}     & exp \in R; \\
       & \sigma(exp_1) \cup \dots \cup \sigma(exp_n) & exp = \varphi(exp_1, \dots , exp_n).
    \end{aligned}
\right.
$
\end{definition}


\begin{figure}[tp]    
    \centering
    \includegraphics[width=\textwidth]{figure/style-A1.png}
    \caption{截取自EUSES 数据库中的电子表格 summ0602.xls 的工作表summary1201(A1 表示法)}
    \label{figure-A1}
\end{figure}
\begin{figure}[tp]   
    \centering
    \includegraphics[width=\textwidth]{figure/style-R1C1.png}
    \caption{截取自EUSES 数据库中的电子表格 summ0602.xls 的工作表summary1201(R1C1 表示法)}
    \label{figure-R1C1}
\end{figure}

\begin{definition}
    绝大多数电子表格软件有两种内置的表达公式应用的风格,即\textit{A1表示法} 和 \textit{R1C1表示法} \cite{tan2014bug} ,另一种划分方式是\textit{绝对引用}和\textit{相对引用}。
\end{definition}

绝对引用指向特定的单元格,当它在某个表达式中存在,并且该表达式被复制到其他单元格,仍然引用相同的单元格。
相对引用表示单元格地址在当前单元格和被引用单元格之间的偏移量,当该引用被复制到其他单元格时,偏移量保持不变,但实际引用的单元格地址发生了变化。

在 A1 表示法中,一个在第 $X$ 列第 $y$ 行的单元格在相对引用中表示为 $Xy$(例如$B5$),在绝对引用中表示为 $\$X\$y$(例如$\$B\$5$)。
如图\ref{figure-A1}所示,该电子表格使用的是 A1 表示法,且所有单元格引用都是相对引用。

另一方面,在 R1C1表示法中,一个在当前单元格的下方第$n$行和右侧第$m$列的单元格在相对引用中表示为$R[n]C[m]$(其中,$n$/$m$在$n=0$/$m=0$时可以省略,而一个处在第$n$行,第$m$列的单元格在绝对引用下表示为$RnCm$。
如图\ref{figure-R1C1}所示,该电子表格使用 R1C1表示法,且所有单元格引用都是相对引用。

一个有趣的观察是:含有相似计算模式的公式单元格通常具有语义等价的 R1C1 表示的公式形式。
例如,图\ref{figure-A1}中的单元格 $D11$ 中的公式$(C11/C\$21)*100$ 在 R1C1表示法中是图\ref{figure-R1C1}中的$(RC[-1]/R21C[-1])*100$。
后一个公式表示对两个单元格的值进行除法运算再乘以一百。
第一个值由同一行向左一列的单元格给出,第二个值由第 21 行向左一列的单元格给出。
图\ref{figure-R1C1}也给出了图\ref{figure-A1}中其他所有公式对应的 R1C1 表示结果。
我们不难观察出:其中一些单元格在计算意义上是等价的,或者是相似的。
我们正式利用这一出发点,从中提出特征来讲单元格进行区分,以便形成不同的单元格类,进而检测出每个类中包含的有缺陷的单元格。

在接下来的讨论中,除额外说明,我们假设表达式$exp$和函数$\sigma(exp)$都使用 R1C1表示法。

\section{单元格类}

单元格类的存在通常是为了后续进行单元格缺陷的检测。
类似地,软件测试为了发现代码中的错误,通常要对代码进行静态分析、动态执行,收集抽象或者具体的代码执行记录,通过对记录的分析来寻找代码中的错误根源。
我们对整个工作表进行分类,找出其中包含的不同的单元格类,本质上就是在试图找到该工作表中包含的具有各自计算意义的单元格集合。
进而,在这种单元格集合中,进行细致分析找出潜在的有缺陷的单元格。
这里我们给出单元格类,在本文中的定义:

\begin{definition}
    \textit{单元格类}是单元格的集合,其中每个单元格都具有\textit{相似}的\textit{计算目标}。
\end{definition}

要理解单元格类,需要在各类技术中明确其中的两个关键概念:
\begin{itemize}
    \item \textit{计算目标}的概念:能够直接或者间接表明该单元格的计算目标的特征,都可以用来表征该单元格的计算目标。比如,直接的特征就是该单元格拥有的公式(但该公式可能有缺陷,未必完全准确地反映该单元格的计算目标);间接的特征比如该单元格周围的其他单元格,或者该单元格的表头文字信息等,都可以间接地暗示出该单元格可能拥有的部分计算目标。
    \item \textit{相似}的概念:通常定义相似的方式是量化该单元格在一个单元格集合中,与整个集合中其他单元格的计算目标的相似度。类似地,可以采用比较直接特征和间接特征相似度的方式,来权衡最终两个单元格的相似度。
\end{itemize}
在缺乏终端用户的先验知识的前提下,如何定义计算目标,以及如何量化单元格之间的计算目标相似度,正是各类电子表格缺陷检测技术的设计挑战。

\section{单元格的公式缺陷}

% 这里似乎可以加一句,跟 Chapter 2 里的相关工作相呼应。
在理解单元格类的基础上,我们来进一步解释何为单元格的公式缺陷。
为了解释的方便,我们结合 EUSES 语料库中的一个工作表上的分析结果(如图\ref{figure-sg5}和\ref{figure-sg6})来辅助理解。
这个例子中包含 7 个有缺陷的单元格,其中的 6 个数值单元格极有可能是错误的表达方式。

例如,如果单元格$F17$的值与根据同类里的公式$(RC[-1]/R20C[-1])*100$计算出的值不一致,并且单元格区域$[F11:F19]$的值求和之后并不等于一百,这和单元格$F21$想要得到的计算结果也不一致。
在某些单元格(如$F17$)中的错误可能进一步导致其他单元格的错误(如$F21$)。
通常,有缺陷的单元格中一定存在\textit{替换}导致的异常,通常是用新的子表达式替换了旧的子表达式。
这里,我们根据用于替换的新的子表达式类型来对单元格缺陷进行分类,主要可以归为三类:

\begin{definition}
   \textit{用数值替换的缺陷}:我们把一个单元格用数值(常量)替换一个子表达式(如单元格引用),然而其它的同类单元格并没有这么替换时,这类单元格的缺陷称为用数值替换的缺陷。
\end{definition}
如图\ref{figure-sg6}所示,黄色标记的单元格类包含$[D11:D18]$和$[F11:F18]$这两部分,其中单元格$D16$,$F14$和$F17$都是数值单元格,然而类中的其他单元格都是具有公式$(RC[-1]/R20C[-1])*100$形式的,即这三个有缺陷的单元格用数值替换了整个表达式,进而在后续维护过程中丢失了其本该具有的计算目的,将来可能导致更为严重的错误。

\begin{definition}
    \textit{用不一致的单元格引用替换的缺陷}:类似地,我们把一个单元格用一个错误的单元格或者单元格范围替换一个子表达式,然而其它的同类单元格并没有这么替换时,这类单元格的缺陷称为用不一致的单元格引用替换的缺陷。
\end{definition}

如图\ref{figure-sg6}所示,浅蓝色标记的单元格类包含$[B20:F20]$这五个单元格,其中单元格$F20$引用了一个单元格区域$[F11:F18]$,这和该类中的另外两个公式的引用区域不同,丢失了对单元格$F19$的引用,按照我们对单元格类和单元格计算目标的理解,单元格$F20$包含用不一致的单元格引用替换的缺陷,如果将来某个终端用户在第 19 行填充了相关数据,这会导致$F20$产生错误的计算结果。

\begin{definition}
    \textit{用不一致的操作符/函数替换的缺陷}:类似地,我们把一个单元格用一个错误的操作符/函数替换一个子表达式,然而其它的同类单元格并没有这么替换时,这类单元格的缺陷称为用不一致的操作符/函数替换的缺陷。
\end{definition}
如图\ref{figure-sg6}所示,暗粉色标记的单元格类包含$[B26:F26]$这五个单元格,目前单元格$E26$含有正确且一致的公式形式$SUM(E23:E24)$,但如果用户使用不一致的操作符进行修改,比如改写成$E23 + E24$,那么将来某个终端用户对这个类中的公式进行重构时,可能未必能发现单元格$E26$也属于这个类的一员,进而忽视了对它的修改,可能导致最终的计算结果错误。


上述三类单元格缺陷类型已经在真实商业场景下频繁发现\cite{panko2006facing,powell2008critical}。
学界也存在类似的案例调研,关注电子表格缺陷如何被引入到表格中\cite{dou2014spreadsheet}。
由于缺陷的类型和变体多种多样,普通终端用户在选用电子表格测试技术时,由于并没有预先知道自己的表格中潜在的缺陷类型属于哪一类,并不好针对性的选择特定技术。那么基于学习、单元格聚类的技术因为天生具有对多种缺陷的适应性,能够把上述各种类型的缺陷单元格都尽可能不遗漏地加入到单元格类中,最后再做统一的缺陷检测分析。这对于终端用户来说,是一个更容易接受并采纳的电子表格缺陷检测技术方案。

\chapter{系统架构与方法设计}
在本章中,我们提出并详细描述我们在 \cu 的基础上构建的单元格聚类和缺陷检测技术\wa 。
我们首先介绍 \wa 的工作流程,以及它和 \cu 的结构关系。
之后,我们详细介绍 \wa 的三个基于有效性属性的单元格聚类检验方法\footnote{在本文中,检验方法常常和检验方法换用,表达同一个含义,即基于某个属性规则,对单元格聚类结果进行优化,剔除不合格的单元格和单元格类}。


\section{系统架构}
如图\ref{figure1}所示,\wa 和 \cu 集成在一起,包含 4 个阶段:两阶段的单元格聚类、单元格类检验和公式缺陷检测。

首先,\wa 使用 \cu 的第一阶段来生成一个包含若干个种子单元格类的集合,每个种子类中的任意一个单元格都包含明显相似的计算特征,即\textit{强特征}(例如公式表达式的语法树结构,以及公式表达式的单元格引用结构)。
这个阶段使得每个种子类都包含一个高度相似的计算语义。

第二,\wa 使用 \cu 来扩充每个种子类,将还没被吸纳进任何类中的数值单元格和公式单元格作为对象,只要这些单元格和在第一阶段已经在种子类中的单元格拥有相似的单元格布局或潜在的计算特征,即\textit{弱特征}(例如单元格位置、单元格表头信息、以及是否隶属于相同的单元格模板等)。
这个阶段是将在第一阶段被忽视的单元格重新吸纳起来,这些被忽视的单元格通常由于自身含有某种公式缺陷,因而在第一阶段中被忽略。

第三,\wa 检验第二阶段得到的单元格类,通过将那些违反有效性属性的单元格(后面三节会详述这三种属性)从它所属的类中排除,或者将整个违反有效性属性的单元格类直接移除。
这个阶段通过识别与对应类的计算语义不相关的单元格和不合格的单元格类,来提升单元格聚类的准确性。

第四,\wa 使用 \cu 从每个单元格类中识别出含有公式缺陷的单元格,并把这些单元格汇报给终端用户。
这个阶段是在含有共同计算语义的单元格类中检测第三章定义的三类公式缺陷,即用数值替换的公式缺陷、用单元格引用替换的公式缺陷和用操作符/函数替换的公式缺陷。

\cu 的优点在于能够将许多零散分布的单元格吸纳到种子类中,从而提升了单元格聚类和缺陷检测的召回率\cite{cheung2016custodes}。
然而,这一吸纳过程比较激进,因为它同时也将相当数量的与单元格类的计算语义不相关的单元格吸纳进来,影响了单元格聚类和缺陷检测的精度。
正因如此,这一不足之处正是我们的技术 \wa 提出的三个检验方法想要解决的痛点,接下来我们详细描述检验单元格和单元格类的方法细节。
\begin{figure}[tbp]    
    \centering
    \includegraphics[width=0.8\textwidth]{figure/figure1-copy.pdf}
    \caption{\wa 的工作流程(阶段 3 是相对于它的前身\cu 的主要贡献点)}
    \label{figure1}
\end{figure}


\section{针对单元格自身的有效性检验}
\begin{figure}[tp]
    \centering
    \includegraphics[width=\columnwidth]{figure/figure2.png}
    \caption{用于展现\wa 的单个单元格有效性检验方法的工作表``Summary1201''。\cu 检测出一个单元格类(用紫色标记),导致四个误报的缺陷(用红色三角标注),不过这四个单元格会被\wa 的单个单元格有效性检验方法排除出去,也就不再被错误标记为缺陷。}
    \label{figure2}
\end{figure}
第一个有效性检验方法考虑的是,当 \wa 准备将数值单元格扩充到种子类中的时候,这些被扩充的数值单元格自身是否可能拥有和种子类的计算语义相似的可能性,简称为\textit{有效性}。

在实际情况下,并没有直接的方法可以判断这些数值单元格的有效性,换言之,因为它们仅包含纯数值,无法从它自身找出和其它单元格之间的任何明显的计算语义关系。
不过,因为这些被扩充的单元格和种子类中的公式单元格即将被纳入同类,那么根据第三章中对单元格类的定义,它们应该享有一个相似的计算目标。
那么,这些被扩充的单元格的数值应当可以用某个公式单元格中含有的表达式计算出来,这是一种对数值单元格较强的约束。

为了验证这一约束,\wa 会枚举种子类中所有单元格的公式,“复制并粘贴”该公式到数值单元格中,来判断是否至少存在一个公式的计算结果和被扩充的这个单元格的值保持一致,如果保持一致,则称该数值单元格满足\textit{强有效性}。
但同时又考虑到如果数值单元格本身是含有缺陷的,那复制的公式计算的结果可能和原来的数值不同。因此,我们放宽了约束条件,当我们用一个公式来替换当前数值单元格里的内容时,只要该单元格是可计算的,即称为满足\textit{弱有效性}。
相反地,如果所有复制的公式都是不可计算的,例如复制的公式引用了错误的单元格类型,或本身包含了无效的单元格引用(比如越界),这个被扩充的数值单元格就是不满足弱有效性的,会被阻止扩充进该类中。
我们把这种方法就称为\textit{针对单元格自身的有效性检验}。
% 这里可以解释一下什么是 “用公式替换数值” 的思路

接下来,我们形式化地描述一下该检验方法。在前两个阶段得到的某以个种子类记为$SC$,对应的拓展类记为$EC$,种子类本身是个集合,包含若干个公式单元格,记为$SC= \{c_{1}, c_{2}, \dots, c_{n}\}$, 所有种子类中的公式合集记为$F$,用$eval(f)$表示计算公式$f$得到的数值,当前等待检验的数值单元格记为$c_v$,它存储的数值为$v$。

那么上述自然语言描述的强有效性定义如下:
\begin{definition}
    如果 $\exists f \in F,  eval(f_{c_v}) = v$,那么就称数值单元格$c_v$满足被吸纳进$SC$中的强有效性。   
\end{definition}

对应地,弱有效性定义如下:
\begin{definition}
    如果 $\exists f \in F, f_{c_v}$的计算过程不会触发异常,即是可计算的,那么就称数值单元格$c_v$满足被吸纳进$SC$中的弱有效性。
\end{definition}

这里我们要额外解释一下$f_{cc_v}$的含义。
该过程指用$SC$中的某个单元格$c_i$含有的公式$f$(用 R1C1 表示法的公式)复制粘贴到单元格$c_v$中,结合第三章对公式表达式的定义,复制粘贴过程中常量、操作符、函数名、单元格引用都保持不变。
我们在实际 \wa 实现中,使用弱有效性定义来检验单元格自身。
因为即便不能在聚类时直接确认该数值单元格具有公式缺陷,在后续的缺陷检测过程中,数值单元格仍旧会被标注为含有用常量替换的公式缺陷,并不影响最终的检测结果。

\subsection{案例分析}
如图\ref{figure2}所示,工作表“Summary1201”给出了一个案例,其中 25 个单元格(B11,B13,B14,B16,D11-17,F11-17和 H11-17)被\cu 划分为同类(用紫色标注)。
\cu 检测出了 6 个缺陷(用红色三角标注),其中 2 个(F17 和 H15)是真阳性,另外四个(B11,B13,B14 和 B16)是假阳性。
后四个数值单元格被扩充进这个类,是因为它们具有和种子类中单元格类似的弱特征(例如相似的表头和布局)。

然而,根据 \wa 的单个单元格有效性检验方法,这样的扩充是有问题的。
事实上,如果这四个单元格中的任意一个被扩充到这个类中,这个单元格把它的值和任意一个类中包含的公式计算出来的值相一致。
例如,考虑单元格 B11,按照类中包含的公式来看,对于它来说最好的潜在公式应该是“=(A11$/$A\$21)$*$100”。
然而,这个公式是不可计算的,因为单元格 A11 和 A\$21指向字符串单元格,无法参与这类除法算数运算中。
相似的问题也出现在单元格 B13、B14 和 B16。 
因此,\wa 会防止这类数值单元格被扩充到类中。


\section{针对单元格之间关系的有效性检验}
第二个有效性检验方法考虑的是,当 \wa 用额外的数值单元格扩充种子类时,这类被扩充进来的单元格不会破坏种子类中已有单元格之间的属性。

我们拿单元格的引用举例,因为这是电子表格单元格的重要特征。
假设种子类中已有单元格的引用从不会彼此重叠,那么我们会预期一个被扩充进来的数值单元格当它被加入到这个类中,并且它的值被类中某个其他单元格的公式替换时,也不会违反这个属性。
这个预期也可以用一种相反的方式表达出来,即一个单元格类中的已有公式之间的引用已经彼此重叠了,那么对于被新扩充进来的数值单元格也不能发生不相交的情况。
也就是说,这个属性应当对类中所有的单元格保持一致(不管是原有的单元格,还是被扩充进来的单元格),这可以被看作电子表格中标的编辑风格。
另外,如果检测到违反该一致性的数值单元格,它就应当被禁止扩充到这个类中。
我们把这种方法称为\textit{针对单元格之间关系的有效性检验}。

接下来,我们形式化地描述一下该检验方法。

\subsection{案例分析}
\begin{figure}[tbp]
    \centering
    \includegraphics[width=\columnwidth]{figure/figure3.png}
    \caption{用于展现\wa 的多个单元格有效性精化方法的工作表``Detail for the College of A\&S''。\cu 检测出一个单元格类(用黄色标记),并进而导致六个假阳性的缺陷被报告出来(用红色三角形标注),不过这六个单元格会被\wa 的多单元格有效性精化方法从类中排除出去,进而不会再被错误的标记为有缺陷的单元格。}
    \label{figure3}
\end{figure}
如图\ref{figure3}所示,工作表“Detail for the College of A\&S” 给出了一个案例。
其中,9 个单元格(AA8, AB8, AC8, AE8, AF8, AG8, AL8, AM8和AN8)被\cu 划分到同一个类中(用黄色标记)。
接着,\cu 检测出 6 个单元格缺陷(AA8, AB8, AE8,AF8,AL8和AM8),因为它们只包含纯数值,但这六个缺陷都是假阳性。

不过,\wa 能够禁止这六个单元格(AA8, AB8, AE8,AF8,AL8和AM8)被扩充进来,从而避免了这 6 个误报的情况。
事实上,这 6 个数值单元格并不和其它 3 个公式单元格(AC8,AG8和AN8)共享相同的计算语义。
前面的数值单元格代表了用户直接给出的具体的数值,而另外三个公式则是计算它们各自左侧的若干个单元格的和。

\wa 通过它的多个单元格有效性检验方法区分出这两类:后三个公式单元格的引用范围是互不重叠的,但如果把前面 6 个数值单元格扩充进来,并用任意一个公式来替换它们的值,这个属性会遭到破坏。
例如,当把单元格 AF8 中的数值用单元格 AG8 里的公式模板替换成公式“=SUM(AD8:AE8)”时,它的引用单元格(AD8 和 AE8)会和单元格 AG8 的引用单元格(AE8 和 AF8)发生重叠。
相似的问题也会出现在单元格 AA8,AB8,AE8,AL8 和 AM8 上。
因此,\wa 会禁止这类数值单元格被扩充到类中。


\section{针对整个类的有效性检验}
第三种有效性检验方法考虑的是最终形成单元格类的整体有效性,即它关注于类层面而不是单元格层面的有效性属性。

我们预期在这个类中应该存在一个统一的公式能够覆盖大多数单元格,也就是说大多数单元格应当遵循一个共同的计算语义。
\wa 会测试类中所有现存可获得的公式,如果没有任何一个能够满足这个目的,\wa 就会认定这个类不是有效的,并将该类从所有类的集合中删除,以避免后续在缺陷检测过程中将其中的过半数量的单元格标记为含有公式缺陷,即产生大量缺陷误报。
我们把这种方法称为\textit{针对整个类的有效性检验}。

接下来,我们形式化地描述一下该检验方法。

\subsection{案例分析}
\begin{figure}[tp]
    \centering
    \includegraphics[width=\columnwidth]{figure/figure4.png}
    \caption{用于展现\wa 的整个类有效性检验方法的工作表``World 1996''。\cu 识别出了一个类(用花色标记出来,但整个类是不合理的),进而导致所有相关的单元格都被错误地标记为有缺陷的(用红色三角形标记),不过这个类会被\wa 的整个类的有效性检验方法移除掉,进而相关的所有单元格也不会被错误地标记为有缺陷的。}
    \label{figure4}
\end{figure}
如图\ref{figure4}所示,工作表“World 1996”给出了一个案例。
其中,\cu 把 10 个单元格划分为同一个类(用黄色标注)。
进而 \cu 把其中的 7 个检测为有缺陷的单元格,但这 7 个都是假阳性。

事实上,这10 个单元格包含几乎完全不同的公式(5 种计算模式),这强烈地表明它们本质上遵循不同的计算语义。
通过我们的类级别的有效性检验方法,\wa 会将该类完整地从类集合中删除。
这里需要注意的是,\wa 需要一个阈值来控制“覆盖大多数单元格”这个判断的程度。
安全起见,\wa 选择一个保守的值,即 50\%,来尽可能保护单元格类不被排除在外(作为对比,\ca 选择了相对激进的值70\%)。


\section{有效性检验方法的扩展空间}


\section{有效性检验方法的综合应用}
根据上述设计的三个有效性检验方法之间的层次关系,\wa 将三个方法依次应用在 \cu 的第二阶段单元格聚类之后,即依次使用针对单元格自身、针对单元格之间关系和针对整个类的有效性检验方法。
它们三者彼此互补,构成了针对单元格聚类的有效性检验方法的有机整体。
它们的目标是提升单元格聚类有效性,使得聚类结果更加具有稳定和可靠。

稳定可靠的单元格聚类结果,最终也会对缺陷检测起到积极作用,如极大地减少公式缺陷的误报数量和误报率,这部分也会在第六章的实验评估部分进行验证。
\chapter{系统实现和工具演示}

\section{\wa 系统实现}

\wa 在实验评估中使用 Java 语言实现,并使用 Apache POI 库 \footnote{https://poi.apache.org} 来读写 Excel 格式的电子表格,由于包含一些历史遗留的 DLL 文件,目前只能在 Windows 操作系统下执行。
相应源码发布在 GitHub 网站上 \footnote{https://github.com/dlee992/QRS19-Code} 。

\wa 工具总共包含约 7,300 行 Java 代码,除了包含 \cu 的源代码以外,额外增改了约 2,500 行代码。
类似于 \cu 的方式,\wa 标记它分析过的电子表格的方法是使用不同的背景色来标注检测到的同一张工作表内的所有单元格类,并通过在单元格右上角添加注释的方式标注该单元格的缺陷类型,例如丢失公式的缺陷或者包含不一致公式的缺陷。

\section{SGuard 集成化测试工具}

\sg 工具也是使用 Java 实现的独立于 Excel 的可视化集成工具,也是用 Apache POI 库来读写 Excel 格式的电子表格。
和\wa 类似,目前只能在 Windows 操作系统下执行。
\sg 的初始核心代码是由张瑞青开发的,我们额外添加了两个新技术模块,即 \cu-OPT 和 \wa ,并进行了对应的集成编码工作。
相应的工具介绍主页发布在 GitHub Pages 上 \footnote{https://sheetguard.github.io/sguard/},其中包含了工具源码链接和介绍性的视频链接,发布在 YouTube \footnote{https://www.youtube.com/watch?v=gNPmMvQVf5Q} 和 Bilibili 网站上。

\sg 的完整实现包含约 10,500 行 Java 代码,其中包含 7,300 行核心代码和 3,200 行图形界面代码。

\subsection{界面设计}

\begin{figure}[tp]   
    \centering
    \includegraphics[width=\textwidth]{figure/figure11.png}
    \caption{\sg 的使用截图}
    \label{figure11}
\end{figure}

图\ref{figure11} 展示了\sg 使用交互界面来检测电子表格并展示相应结果的截图。
整个界面包含四大区域:

\begin{itemize}
    \item 上方的文件选取区域包含 Open Excel file(打开 Excel 文件)和Switch to values/formulas(切换到数值/公式形式)两个按钮,中间部分显示当前打开的 Excel 文件路径,如 A:$\backslash$illustrative\_exmaple.xls;
    \item 左侧的工作表选择区域列出了当前选定的 Excel 文件的所有工作表;
    \item 中间的工作表内容展示区域展示了类似于电子表格软件的核心界面,以字母标记的列号(A,B,\dots)和以数字标记的行号(1,2,\dots),每个对应的表格位置显示该工作表的具体内容,即公式,数值或文本,目前不能显示除此以外的模块,比如工作表中插入的图,以及不会保留原工作表的排版风格,只显示每个单元格里的具体内容(特别地,还不支持显示多单元格合并为一个单元格的显示效果);
    \item 右侧的核心操作区域包含如下几块内容:
        \begin{enumerate}
            \item 上方罗列了 \sg 工具内部囊括的所有检测技术,共有 6 个检测技术,分别是 \am\cite{dou2014spreadsheet},\ca\cite{dou2017cacheck},\cu\cite{cheung2016custodes},\cu-OPT\footnote{http://sccpu2.cse.ust.hk/custodes/cc2s.html},\wa 和 TableCheck\cite{dou2016detecting}。每个技术名称后面标记了对应论文发表的时间或者该技术更新的最新时间;
            \item 紧接着向下是 Start analysis 和 Cancel 两个按钮,顾名思义,分别用来开始执行技术对应的代码和取消当前分析过程;
            \item 再往下有两个显示标签,Detective cells 和 Marked clusters,每个运行的技术分别对应一个选项;
            \item 再往下是 Info 栏,展示一些执行信息以及每个电子表格缺陷检测结果汇总,包含每个技术检测到的单元格类数量和有缺陷的单元格数量;
            \item 紧接着是一个 Save to files 按钮,可以用来保存当前选定技术的检测结果到新的 Excel 文件中;
            \item 最后是 Logs 栏,用来显示每个完成执行的技术和对应的时间戳。
        \end{enumerate}
\end{itemize}

\subsection{使用展示}

接下来,我们结合一个简单的工作表例子来展示\sg 的完整使用流程。

\begin{figure}[tp]   
    \centering
    \includegraphics[width=\textwidth]{figure/sg/sguard-1.png}
    \caption{\sg 的使用流程演示截图 1}
    \label{figure-sg1}
\end{figure}
\begin{figure}[tbp]    
    \centering
    \includegraphics[width=\textwidth]{figure/sg/sguard-2.png}
    \caption{\sg 的使用流程演示截图 2}
    \label{figure-sg2}
\end{figure}
\begin{figure}[tbp]    
    \centering
    \includegraphics[width=\textwidth]{figure/sg/sguard-3.png}
    \caption{\sg 的使用流程演示截图 3}
    \label{figure-sg3}
\end{figure}
\begin{figure}[tbp]    
    \centering
    \includegraphics[width=\textwidth]{figure/sg/sguard-4.png}
    \caption{\sg 的使用流程演示截图 4}
    \label{figure-sg4}
\end{figure}
\begin{figure}[tbp]    
    \centering
    \includegraphics[width=\textwidth]{figure/sg/sguard-5.png}
    \caption{查看 \wa 技术的缺陷检测结果}
    \label{figure-sg5}
\end{figure}
\begin{figure}[tbp]    
    \centering
    \includegraphics[width=\textwidth]{figure/sg/sguard-6.png}
    \caption{\sg 的使用流程演示截图 6}
    \label{figure-sg6}
\end{figure}
\begin{figure}[tbp]    
    \centering
    \includegraphics[width=\textwidth]{figure/sg/sguard-7.png}
    \caption{\sg 的使用流程演示截图 7}
    \label{figure-sg7}
\end{figure}
\begin{figure}[tbp]    
    \centering
    \includegraphics[width=\textwidth]{figure/sg/sguard-8.png}
    \caption{\sg 的使用流程演示截图 8}
    \label{figure-sg8}
\end{figure}

\begin{enumerate}
    \item 如图\ref{figure-sg1}和\ref{figure-sg2}所示,我们点击 Open Excel file 按钮,从文件浏览器中选择一个要测试的电子表格,这里我们选择预先写好的 illustrative\_example.xls 文件;
    \item 如图\ref{figure-sg3}所示,从左侧列出的工作表中,我们选择名为 summary1201 的工作表,对应的单元格内容就自动显示在正中间;
    \item 如图\ref{figure-sg4}所示,我们在右侧功能区选择准备使用的电子表格测试技术,这里我们勾选上出TableCheck 之外的所有技术,并点击 Start analysis 开始执行;
    \item 如图\ref{figure-sg5}所示,我们会在 Logs 区域看到各个技术开始和结束的标志,在 Defective cells 选项卡里可以勾选上多个技术,查看对应检测到的缺陷单元格全集,这里我们仅选择\wa 技术,可以看到\wa 检测到了 7 个缺陷单元格,与 Info 区域显示的检测结果保持一致;
    \item 类似地,如图\ref{figure-sg6}所示,我们在 Marked clusters 选项卡里可以勾选多个技术,这里我们仅选择\wa 技术,中间区域就显示出\wa 检测到的 7 个单元格类,与 Info 区域显示的检测结果保持一致;
    \item 如图\ref{figure-sg7}所示,我们点击 Save to files 按钮之后,每个技术的检测结果单独生成一个 Excel 文件,和源文件放在同一个目录下;
    \item 如图\ref{figure-sg8}所示,我们点开任意一个工具标注的 Excel 文件,这里我们打开 illustrave\_example\_WARDER.xls 文件,该文件中标注了和 \sg 工具中显示的相同的单元格类,以及在右上角用注释的方式标注出有缺陷的单元格。
\end{enumerate}


\section{\eg 插件测试工具}

\eg 插件测试工具使用 JavaScript 语言实现,可在 Excel 软件中直接加载并使用的第三方插件,采用 Microsoft Office-js \footnote{https://github.com/OfficeDev/office-js} 框架来异步读写和操作 Excel 文件。
比\sg 更好地是,只要是 Excel 可以运行的平台,如 Windows,MacOS 或者浏览器中,都可以加载我们的\eg 插件,接近于跨平台兼容(暂不支持 Linux 操作系统,因为 Excel 本身就没有 Linux 发行版)。
% 相应的工具源码发布在 GitHub 上\footnote{https://github.com/dlee992/EGuard}。

目前,\eg 测试工具的当前实现包含约 3100 行 JavaScript、HTML 和 CSS 代码,其中包含约 2400 行核心功能代码和约 700 行图形界面代码,后续还会继续更新到一个更加完善的版本,然后正式发布上线。

\begin{figure}[tbp]    
    \centering
    \includegraphics[width=\textwidth]{figure/eg/eguard-1.png}
    \caption{\eg 的插件布局}
    \label{figure-eg1}
\end{figure}
\begin{figure}[tbp]    
    \centering
    \includegraphics[width=\textwidth]{figure/eg/eguard-2.png}
    \caption{\eg 执行单元格聚类后的电子表格标记和执行信息输出}
    \label{figure-eg2}
\end{figure}
\begin{figure}[tbp]    
    \centering
    \includegraphics[width=\textwidth]{figure/eg/eguard-3.png}
    \caption{\eg 执行三个单元格精化方法后的电子表格标记和执行信息输出}
    \label{figure-eg3}
\end{figure}
\begin{figure}[tp]   
    \centering
    \includegraphics[width=\textwidth]{figure/eg/eguard-4.png}
    \caption{\eg 执行缺陷检测后的电子表格标记和执行信息输出}
    \label{figure-eg4}
\end{figure}

\subsection{界面设计}

如图所示,\eg 使用 Excel 中嵌入一个网页的方式来和用户交互,进行电子表格单元格聚类和缺陷检测的测试过程。
整个交互界面分为如下部分:
\begin{enumerate}
    \item 上方 
    \item 中间 
    \item 下方 
\end{enumerate}

\subsection{使用展示}

接下来,我们结合一个简单的工作表例子来展示\eg 的完整使用流程。 
\chapter{实验评估}

本章中,我们评估我们的技术WARDER,并和现有的电子表格缺陷检测技术作对比。

\section{研究问题}

我们旨在回答下列三个研究问题:

\begin{itemize}
    \item \textbf{研究问题1(有效性)}:与现有电子表格缺陷检测技术相比,\wa 在单元格聚类和缺陷检测方面有效性如何?
    \item \textbf{研究问题2(相关性)}:\wa 优化后的单元格聚类对最终的缺陷检测精度提升有多大贡献?
    \item \textbf{研究问题3(独立性)}:\wa 的三个有效性精化方法对缺陷检测的效果提升分别有多大?
\end{itemize}

\section{实验设计和设置}

\begin{table}[tbp]
    \centering
    \caption{基准测试集的统计数据}
    \label{table1}
    %\large
    \resizebox{\columnwidth}{!}{
    \begin{tabular}{|m{.15\columnwidth}<{\centering}|m{.12\columnwidth}<{\centering}|m{.15\columnwidth}<{\centering}|m{.2\columnwidth}<{\centering}|m{.15\columnwidth}<{\centering}|m{.25\columnwidth}<{\centering}|}
    \hline
    \textbf{\# 电子表格} &  \textbf{\# 工作表} &  \textbf{\# 单元格} &  \textbf{\# 公式单元格} &  \textbf{\# 单元格类} &  \textbf{\# 有缺陷的单元格} \\ 
    \hline
    70 & 291 & 189,027 & 26,716 & 1,610 & 1,974 \\
    \hline
    \end{tabular}}
\end{table}

\subsection{基准测试集} 

为了便于\wa 和它的前身\cu 的比较,我们选择了 \cu 采用的从EUSES语料库中采样的测试集,作为我们的实验基准测试集。如表~\ref{table1}所示,该测试集包含70个电子表格和291个工作表。这291个工作表包含189,027个单元格,其中包含26,716个公式单元格。出于实验评估的目的,该测试集包含人工标注的数据(标记方法详见~\cite{cheung2016custodes}),其中包含1,610个单元格类和1,974个有缺陷的单元格(丢失公式或含有不一致的公式)。

\subsection{测试技术} 

在实验中,\wa 将和五个之前提到的电子表格缺陷检测技术进行对比,即\uc,\di,\am,\ca 和 \cu。我们从它们各自的原作者那里获取了对应的可执行工具或源码。主要在缺陷检测的有效性方面进行比较。对于\ca,我们额外比较了它们的单元格聚类的有效性。

为了评估的三个有效性精化的独立性(研究问题3),我们采用不同的配置来测试各自的实验效果,三种配置依次分别标记为\wasc (带有单单元格的有效性精化),\wamc (带有多单元格的有效性精化),\wawc (带有整个类的有效性精化)。最后,带有全部三种精化的配置被称为\wa-full,或简记为\wa。

\subsection{评价指标} 

针对缺陷检测的有效性,我们首先统计每个技术报告的缺陷数量,以及其中的真阳性(TP),假阳性(FP)和假阴性(FN)数量。基于此,我们进一步根据如下三组公式计算精度$precision_d$,召回率$recall_d$和$F\text{-}measure_d$值,来衡量电子表格缺陷检测上各技术的有效性。

\begin{gather*}
    precision_d=\frac{TP}{TP + FP}\qquad recall_d = \frac{TP}{TP + FN}\\
    f\text{-}measure_d = \frac{2 \times precision_d \times recall_d}{precision_d + recall_d}
\end{gather*}

针对电子表格缺陷的有效性(适用于\wa 和\cu ),我们采用和\cu 类似的方式来统计真阳性(TP),假阳性(FP)和假阴性(FN)数量。类似地,我们也统计这两个技术在单元格聚类上的精度 $precision_c$,召回率 $recall_c$和\fmc 值。

\subsection{测试环境} 

所有实验在一台台式机上进行,配有 Intel$^\circledR$ Core\texttrademark\ i7-6700 CPU @3.41GHz 处理器和 64GB 内存。该机器上安装了微软Windows 10专业版操作系统和Oracle Java 8执行环境。


\section{实验结果和分析}

本节中,我们会依次分析实验结果并回答上述提出的三个研究问题。

\subsection{有效性}

我们首先实验评估\wa 在单元格聚类和缺陷检测方面的有效性,然后和其他五个电子表格缺陷检测技术的结果进行比较。

\begin{table}[tbp]
    \centering
    \caption{6个电子表格缺陷检测技术的检测结果}
    \label{table2}
    \large
    \resizebox{\columnwidth}{!}{
    \begin{tabular}{|c|c|c|c|c|c|c|}
    \hline
    \textbf{技术} & \textbf{检测出} & \textbf{TP} & \textbf{FP} & \textbf{$precision_d$} & \textbf{$recall_d$} & \textbf{$F\text{-}measure_d$} \\
    \hline
        UCheck & 204 & 1 & 203 & 0.5\% & 0.1\% & 0.00 \\
    \hline
        Dimension & 1,824 & 14 & 1,828 & 0.8\% & 0.7\% & 0.01 \\
    \hline
        AmCheck & 2,372 & 1,316 & 1,030 & 56.1\% & 66.7\% & 0.61 \\
    \hline
        CACheck & 1,866 & 1,350 & 516 & 72.3\% & 68.4\% & 0.70 \\
    \hline
        CUSTODES & 2,380 & 1,539 & 841 & 64.7\% & 78.2\% & 0.71 \\
    \hline
        WARDER & 1,612 & 1,415 & 197 & 87.8\% & 71.9\% & 0.79 \\
    \hline
    \end{tabular}}
\end{table}

就缺陷检测而言,表~\ref{table2}对比了所有6个检测技术的结果,包括精度,召回率,\fmd,检测出的缺陷数量,以及其中包含的真阳性和假阳性。从表中,我们观察到:

\begin{enumerate}
    \item 由于他们有限的分析范围,\uc 和\di 获得了较低的分数(精度和召回率都低于1\%,\fmd 值也仅有0.01),这点也与之前的实证研究结果~\cite{zhang2017effectively}保持一致;
    \item 由于他们有效的分析模式(比如,单元格元组),\am 和 \ca 获得了较好的分数(56.1-72.3\%的精度,66.7-68.4\%的召回率,以及0.61-0.70的\fmd 值);
    \item \cu 比 \ca 获得了略微更好地分数(0.71的\fmd 值),以及对召回率的较大提升(72.8\%,对应的\ca 召回率为68.4\%);
    \item 如同预期的那样,\wa 关注于提升检测精度,对比于\cu 获得了大幅度提升,从64.7\% 提升到 87.8\%,最终总体上提升了\fmd 的值,从0.71到0.79,尽管对召回率有6\%的牺牲。 从数据中,可能产生一定疑惑:\cu 比 \wa 多检测出124个真阳性,但同时伴随着更多的假阳性(达644个),这对于终端用户的人工验证来说是很大的负担。
\end{enumerate}

\begin{figure}%
    \centering
    \subfloat[精度比较(仅展示被影响的工作表)]{{
        \includegraphics[width=0.40\textwidth]{figure/figure5a.pdf} 
        \label{figure5a}
        }}%
    \subfloat[召回率比较(整体)]{{
        \includegraphics[width=0.55\textwidth]{figure/figure5b.pdf} 
        \label{figure5b}
        }}%
    \caption{CUSTODES 和 WARDER 的单元格聚类结果}%
    \label{figure5}%
\end{figure}

就单元格聚类而言,图\ref{figure5}比较了\cu 和 \wa 在精度和召回率两方面的结果。从精度比较来看(如图\ref{figure5}(a)),我们把包含至少有一个单元格类的282个工作表分成两类:
\begin{enumerate}
    \item \wa 提升了其中31个工作表的精度,降低了7个;
    \item 对于余下精度保持不变的244个工作表,其中194个已达到100\%(即上限)。
\end{enumerate}
换言之,\wa 提升了225个工作表的单元格聚类精度,要么是实际提升精度,要么已经达到上限。如图\ref{figure6}所示,我们进一步观察这38个精度产生变化的工作表。我们不难发现,\wa 在不同程度上提升了单元格聚类的精度(0.3\%-94.6\%,平均20.7\%),这样的精度提升是显著的,远多于精度受损的部分。另外从图\ref{figure5}(b)中,我们也注意到,\wa 在精度提升上的有效性也仅导致召回率的轻微下降(2.4\%)。

\begin{figure*}
    \centering
    \includegraphics[width = \columnwidth]{figure/figure6.pdf}
    \caption{CUSTODES 和 WARDER 的单元格聚类结果(精度变化方面)}
    \label{figure6}
\end{figure*}
\begin{figure}[tp]
    \centering
    \includegraphics[width = \columnwidth]{figure/figure7.jpeg}
    \caption{工作表“Detail for College of Education” (包含一个绿色标记的单元格类)}
    \label{figure7}
\end{figure}

如图~\ref{figure6}所示,我们也观察到,\wa 对大多数精度受影响的工作表是正面作用,但也有极少量的例外情况,如名为“Detail for College ...”的工作表,它的聚类精度从100\%降低到了0。因此,我们进一步检查了该工作表的聚类情况。如图\ref{figure7}所示,人工标记的结果是,{O11,W11,Z11,AD11,AR11}应该被安排在同一个类中。\cu 看似正确地对这些单元格进行了分类,而\wa 分类得很差。然而,我们发现这五个单元格实际上包含不同的公式,因此它们违反了\wa 的 \wcvp (在这些单元格中不存在一个公共的计算目标能够覆盖过半数量)。事实上,这个单元格类的确不包含任何缺陷。因此,这个单元格聚类精度下降的案例并不影响\wa 的缺陷检测能力。

因此,针对研究问题1,我们能够得出如下结论:\textit{\wa 在单元格聚类和缺陷检测方面是有效的。它极大地提升了精度,达15.5-87.3\%,并且在所有被比较的电子表格缺陷检测技术中,取得了最佳的 \fmd 值(0.79)。}

\subsection{相关性}

\begin{table}[tbp]
    \centering
    \caption{\wa 相对于 \cu 在单元格聚类和缺陷检测上的的精度变化 ($\uparrow$: 精度提升, $\downarrow$: 精度下降, $\to$: 精度保持不变)}
    \label{table3}
    %\Large
    \begin{tabular}{|m{.16\columnwidth}<{\centering}|m{.16\columnwidth}<{\centering}|m{.16\columnwidth}<{\centering}|m{.08\columnwidth}<{\centering}|m{.18\columnwidth}<{\centering}|}
    \hline
    \textbf{相关性类型} & \textbf{聚类的精度变化} & \textbf{缺陷检测的精度变化}  & \textbf{\# 工作表} & \textbf{总计} \\
    \hline
        \multirow{3}{1.2cm}{正相关} & $\uparrow$  & $\uparrow$ & 8 & \multirow{3}*{126 (90.0\%)}\\
    \cline{2-4}
        ~ & $\downarrow$ &  $\downarrow$  & 2 & ~\\
    \cline{2-4}
        ~ & $\to$ &  $\to$  & 116 & ~\\
    \hline
            \multirow{4}{1.2cm}{负相关}& $\uparrow$ &  $\to$  & 5  & ~\\
    \cline{2-4}
         ~ & $\uparrow$  & $\downarrow$ & 2 & \multirow{4}*{9 (6.4\%)}\\
     \cline{2-4}

        ~ & $\downarrow$ &  $\to$  & 2 & ~ \\
    \cline{2-4}
        ~ & $\downarrow$ &  $\uparrow$  & 0 & ~ \\
    \hline
    \multirow{2}*{不相关} & $\to$  & $\uparrow$ & 5 & \multirow{2}*{5 (3.6\%)}\\
    \cline{2-4}
        ~ & $\to$ &  $\downarrow$  & 0 & ~ \\
    \hline
    合计 & - & - & 140 & 140 (100.0\%) \\
    \hline
    \end{tabular}
\end{table}

在这一小节里,我们研究\wa 相对于 \cu 在单元格聚类和缺陷检测上的的精度提升。

如表\ref{table3}所示,我们用三个符号 $\uparrow$,$\downarrow$和$\to$分别表示精度提升,精度下降,以及精度保持不变这三种情况。按照相关性类型,我们把人工标记中至少含有1个缺陷的工作表分成3个类型:
\begin{enumerate}
    \item \textit{正相关}:当相对于\cu 的单元格聚类,\wa 的精度有提升,下降或不变时,在缺陷检测上精度变化表现一致;
    \item \textit{负相关}:当相对于\cu 的单元格聚类,\wa 的精度有提升时,在缺陷检测上精度下降或保持不变;或者相反地,\wa 的精度下降时,在缺陷检测上精度提升或保持不变;
    \item \textit{不相关}:其他的组合情况,均划归此类,既不表现出正相关,也不表现出负相关。
\end{enumerate}
总得来说,我们能够观察到:第一类占据主导(90\%),因此表明 \wa 对单元格聚类的精度提升方法的确能够带来缺陷检测精度的提升,并且九成比例成明显正相关。

\begin{figure}%
    \centering
    \subfloat[CUSTODES 和 WARDER 在单元格聚类上的精度]{{
        \includegraphics[width=.95\columnwidth]{figure/figure8a.pdf} 
        \label{fig8a}
        }}%
    \qquad
    \subfloat[CUSTODES 和 WARDER 在缺陷检测上的精度]{{
        \includegraphics[width=.95\columnwidth]{figure/figure8b.pdf} 
        \label{fig8b}
        }}%
    \caption{\cu 和 \wa 中有精度变化的工作表}%
    \label{figure8}%
\end{figure}
\begin{figure}[tp]
    \centering
    \includegraphics[width = .95\columnwidth]{figure/figure9.jpg}
    \caption{工作表 ``CO'' (绿色标注的两个单元格统计特定引用区域的最大值,橙色标注的两个单元格应该用来统计次大值)}
    \label{figure9}
\end{figure}

图\ref{figure8}展示了在单元格聚类和缺陷检测上\wa 和 \cu 的更细致的精度对比。为了展示的清晰性和简洁性,我们移除了精度前后无变化的 116 个工作表,仅罗列出剩下的 24 个。细致的观察可以发现:在单元格聚类和缺陷检测的精度变化上大部分情况下是正相关的。不过,其中一个例外是名为“CO”的工作表,在这张表上,\wa 的缺陷检测精度从100\%跌至 0,然而单元格聚类精度却有提升。我们进一步检查了这个情况。如图\ref{figure9}所示,人工认为单元格{B11,E11}(绿色标记) 和{C11,F11}(橙色标记)应该构成两个类,并且单元格 C11 和 F11 应该是两个缺陷(用红色三角标记)。\cu 因为意外地将这四个单元格分成同类,而偶然地检测到了这两个缺陷。这是偶然的,因为这四个单元格本质上并不共享相同的计算目标(两个绿色的单元格用来计算最大值,而两个橙色的单元格用来计算第二大值)。\cu 只是简单地将两个橙色的单元格标记为缺陷,因为他们仅仅包含纯数值(丢失公式的缺陷)。另一方面,\wa 仅仅将{B11,E11}划分为同一类,因此无法检测到其中的任何缺陷。它漏掉了两个橙色单元格,因为它们不包含任何公式,且没有和它们相同计算目标的公式存在,因而无法被识别到任何类中。当没有额外的证据时(例如,更多的计算第二大值的单元格出现,并且相应的公式也存在于某些单元格中),\wa 选择不把这些单元格放到任何类中(否则,可能导致更多的假阳性)。

因此,针对研究问题2,我们能够得出如下结论:\textit{\wa 在单元格聚类上的改进的确进一步提升了缺陷检测的结果,这一相关性得到90.0\%的工作表的数据支撑。}


\subsection{独立性}

\begin{table}[tbp]
    \centering
    \caption{\cu 和 \wa 在不同配置下的缺陷检测结果}
    \label{table4}
    \large
    \resizebox{\columnwidth}{!}{
    \begin{tabular}{|c|c|c|c|c|c|c|}
    \hline
    \textbf{技术} & \textbf{检测出} & \textbf{TP} & \textbf{FP} & \textbf{$precision_d$} & \textbf{$recall_d$} & \textbf{$F\text{-}measure_d$} \\
    \hline
        CUSTODES & 2,380 & 1,539 & 841 & 64.7\% & 78.2\% & 0.71 \\
    \hline
        WARDER-sc &2,083 & 1,506 & 577 & 72.3\% & 76.3\% & 0.74\\
    \hline
        WARDER-mc &2,164 & 1,507 & 657 & 69.6\% & 76.3\% & 0.73\\
    \hline
        WARDER-wc &2,071 & 1,498 & 573 & 72.3\% & 75.9\% & 0.74\\
    \hline
        WARDER-full & 1,612 & 1,415 & 197 & 87.8\% & 71.9\% & 0.79 \\
    \hline
    \end{tabular}}
\end{table}

最后,我们研究\wa 的三个有效性精化在检测电子表格缺陷上各自的表现。\wa 被配置成单独使用每一种精化方法(正如前面提到的,命名为\wasc,\wamc 和\wawc ),并和同时使用三种方法的\wa 作比较(\wa -full)。

表\ref{table4}对比了\cu 和 \wa 的四种配置的缺陷检测结果。我们可以观察到:
\begin{enumerate}
    \item \wa 的每个有效性精化是有用的,并且各自能够比\cu 在缺陷检测精度上提升 4.9-7.6\%,在召回率上有较小的牺牲(1.9-2.3\%),最终对\fmd 值有小幅提升(从 0.71 提升到 0.73-0.74);
    \item 组合所有三种有效性精化(即\wa -full) 可以获得最高的精度(87.8\%)和\fmd 值(0.79),这也和之前的表\ref{table1}数据相吻合。
\end{enumerate}

因此,针对研究问题3,我们能够得出如下结论:\textit{\wa 的三个有效性属性能够各自独立展现出对缺陷检测的正面效果;同时在三者组合运用时,也能够获得整体的最佳效果。}
\section{案例研究}

\begin{table}[tbp]
    \centering
    \caption{在 VEnron2 数据集上四个电子表格缺陷检测技术的检测结果}
    \label{table5}
    %\large
    \resizebox{\columnwidth}{!}{
    \begin{tabular}{|m{.15\columnwidth}<{\centering}|m{.16\columnwidth}<{\centering}|m{.16\columnwidth}<{\centering}|m{.14\columnwidth}<{\centering}|m{.12\columnwidth}<{\centering}|m{.1\columnwidth}<{\centering}|m{.12\columnwidth}<{\centering}|}
    \hline
    ~& \multicolumn{3}{c|}{\textbf{全部数据集包含 6,258 个工作表}} & \multicolumn{3}{c|}{\textbf{采样 300 个工作表}} \\
    \hline
    \textbf{技术} & \textbf{\# 工作表} & \textbf{\# 缺陷} & \textbf{耗时 (分钟)} & \textbf{\# 缺陷} & \textbf{\# TP} & \textbf{Precision} \\
    \hline
        AmCheck  & 859  & 20,280 & 21 & 3,316 & 540 & 16.3\% \\
    \hline
        CACheck  & 953   & 12,953 & 372 & 1,559 & 534 & 34.3\% \\
    \hline
        CUSTODES & 1,284 & 14,102 & 537  & 2,334 & 629 & 26.9\% \\
    \hline
        WARDER   & 1,136 & 9,462  & 518  & 1,240 & 512 & 41.3\% \\
    \hline
    \end{tabular}}
\end{table}
\begin{figure}[tp]
    \centering
    \includegraphics[width = 0.9\columnwidth]{figure/figure10.png}
    \caption{四个电子表格缺陷检测技术报告出的真阳性交集的韦恩图}
    \label{figure10}
\end{figure}

除了前面的受控实验,我们也在更大规模的语料库 \ven \cite{xu2017spreadcluster} 上对 \wa 的检测有效性进行了实验。
\ven 包含 1,609 个版本化小组,是从 Enron 语料库 \cite{hermans2015enron} 中 79,983 个工作表中提炼而来。
我们从每个版本化的小组中选取罪行的电子表格文件,总共包含 1,609 个电子表格,其中含有 7,140 个工作表,作为我们案例研究的数据集。
我们把不同的电子表格缺陷检测技术拿来检测这些工作表,进而比较分析它们的检测效果。
考虑到\uc 和 \di 检测出的缺陷数量很少(低于总量的1\%),在案例研究中我们选择了另外四种技术,即\am ,\ca  ,\cu 和 \wa 。
为了加快比较和对比的公平性,我们移除了那些至少有一个技术无法正常执行的工作表(比如:导致异常崩溃,或者超出了我们设定的每张表的五分钟执行时间上限,设置时限的目的是防止程序陷入死锁或者其他未知的错误而影响整体实验进度)。
这一过滤方式给我们的案例研究最终留下了总计 6,258 个工作表。

我们注意到 \ven 并不包含用于评估电子表格缺陷检测技术有效性的人工标注结果(即,每张工作表里实际含有哪些有缺陷的单元格)。
因此,我们主要关注与四个技术的检测精度比较。
另外,由于整体数据量依然庞大,虽然我们对每个工作表都运行了检测程序(用于衡量他们的时间效率),但我们不得不使用采样的方法来选择少量工作表,进行人工参与的检测精度对比。
采样过程遵循了 \am 和 \cu 建议的评估方式。
在所有 6,258 个工作表中, 1,525 个工作表被至少一个技术检测出含有缺陷。
基于此,我们随机采样了约 20\% 的工作表(即 300 个)来进行人工审查,该审查的目的是标记每一个被工具报告出来的缺陷是否是真阳性还是假阳性。
表\ref{table5}对比了四种技术在人工审查后的缺陷检测结果。

从表\ref{table5}的右半部分,我们可以观察到:
\begin{enumerate}
    \item 在四个技术中,\wa 取得了最高的缺陷检测精度(41.3\%),超出其他技术7.0-25.0\%,这和之前我们验证过的 \wa 更加关注于在 \cu 的基础上提升精度的结论(41.3\% vs. 26.9\%);
    \item 尽管\wa 报告除更少的真阳性数量(512),但同时也伴随着少得多的假阳性数量(728),这比其他技术的假阳性数量少了 297-2,048个,这一特征在现实场景可能相当实用,因为所有的电子表格缺陷必须经过人工的审核验证,节省了大量人力。
\end{enumerate}

从表\ref{table5}的左半部分,我们可以观察到:
类似于采样的 300 个工作表,\wa 检测出了最少的缺陷数(9,462),对比于\am 的 20,280,\ca 的 12,953 和 \cu 的 14,102。
考虑到\wa 获得了最高的精度,他的检测按质量预期是最高的(比如,在\wa 检测出的 1,240 个缺陷中有 512 个真阳性,相比较而言,在\am 检测出的 3,316(2.6倍) 个缺陷中仅有 540 个真阳性(仅1.05倍))。
考虑到时间效率,\am 处理所有的 6,258个工作表所花费时间最短(仅 21 分钟),其他三个技术都花了相对长的时间(从 351 到 516分钟)。
这结果表明 \am 适合用于快速识别潜在的电子表格缺陷,但因为他的检测质量相对较低(精度是16.3\%),它更适合作为其他技术的互补验证来使用(比如用于过滤大多数的假阳性缺陷)。
另外,我们注意到\wa 在 \cu 的基础上,仅提升了它的单元格聚类过程。
因此,\wa 也没有那么高效(花费 518 分钟),对比于\cu 的时间消耗(537分钟)。
不过,\wa 移除了不相关的单元格和不合格的单元格蕾,减少了不必要的后续分析,这一努力带来了运行时间的略微缩短(19分钟)。

除了整体的检测精度和时间消耗对比,我们也研究了四个技术检测出的真阳性缺陷的交集,如图\ref{figure10}中的韦恩图所示。
在韦恩图中,黄色椭圆代表\am 检测出的真阳性缺陷,依次地,绿色椭圆代表\ca ,粉色椭圆代表\cu ,紫色代表\wa 。
每个子区域代表两个或多个技术检测出的真阳性缺陷的交集。
从图中,我们可以观察出:
\begin{enumerate}
    \item 基于模板的技术(\am 和 \ca)和基于聚类学习的技术(\cu 和 \wa)明显彼此互补。前者总共检测出 243(9+216+18) 个无法被后者检测出的缺陷,而后者检测出 270(211+59) 个前者无法检测的缺陷。此结果表明两类技术都很有用处;
    \item 在基于模板的技术中,\ca 继承自 \am 技术。相应地,它们检测出的缺陷有一大部分是相同的(78.4\%,602 个中的 472 个是相同的)。因此,\am 单独检测出进 68 个缺陷,而 \ca 单独检测出也仅 62 个。不过,\ca 仍是受欢迎的技术,考虑到它显著减少了假阳性(从\am 的 2776 个减少到 1025 个);
    \item 在基于学习的技术中心,\wa 精化了 \cu 的检测结果。相应地,\wa 仅检测出了 512 个真阳性,是\cu 的子集(81.4\%)。不过,\wa 本身就是更加关注与过滤掉不相关的单元格和不合格的类,并且这一努力带来了显著的假阳性的减少(从\cu 的 1,705 个减少到 728 个);
    \item \ca 和 \wa 作为各技术流派的代表,仍然是相互补充的。它们各自能够检测出 292 个和 270 个对方无法检测出的缺陷。这一结果很明确地表明它们的互补性很强。
\end{enumerate}

因此,\textit{\wa 在实际使用的电子表格上的缺陷检测表现也令人满意。它获得了最高的精度(41.3\%),超出其他技术多达 7.0-25.0\%。它的时间消耗有点高,但也和\ca 相当(在同一个量级),并且少于它的前身\cu。就检测到的缺陷而言,所有相关技术都有它们各自的优势,并且细致调研表明他们是彼此互补的。}

\section{威胁性分析和讨论}

一个主要的内在威胁性来源是单元格聚类指标的计算方式,即\prc ,\rec 和 \fmc 。
它们是基于真阳性,假阳性和假阴性概念计算而来,而后三者又是根据\cu 的结对相似度比较计算而来。
我们注意到这样的比较会统计是否两个单元格属于同一个类或者属于不同的类的单元格对的数量。
这样的计算方式不同于衡量缺陷检测效果的方式。
因此,研究\wa 的单元格聚类和它的缺陷检测相关性结论可能在一定程度上受到影响。
然而,我们依然观察到 90\% 的工作表在我们的分析下是正相关的。
这表明\wa 对单元格聚类的提升的确有助于最终的缺陷检测。

另外,我们也注意到\wa 啊仍然有提升的空间,考虑到它不能检测到某些特定的电子表格缺陷,正如我们之前分析的那样。
\begin{enumerate}
    \item \wa 关注于识别不相关的单元格(将其移除)和无效的单元格类(取消该类的分析)。它并没有挥手哪些相关的单元格,但被\cu 遗漏掉的部分;
    \item 即便所有相关的单元格都能被正确的聚类,\cu 本身仍然不能检测出某些缺陷,由于其自身在缺陷检测上有限的分析能力。因为\wa 仅仅关注于改善单元格聚类过程,并没有涉及到缺陷检测部分的优化。因此,两种技术可能都无法检测到这样的电子表格缺陷。
\end{enumerate}
不过,我们在实验中观察到:\wa 已经极大地提升了 \cu 的适应性。这表明\wa 关注到了对优化起主导作用的因素。不过上述分析也的确指出了进一步优化的新方向。

最后,一个主要的外部威胁性来源是我们尝试了但没能和另外两个基于机器学习的电子表格缺陷检测技术,Melford\cite{singh2017melford} 和 ExceLint\cite{Barowy2018excelint} 进行对比实验和案例研究。
前者,我们没有找到可获得的工具。后者,我们找到了对应的工具但在实验评估时遇到了问题。
首先,ExceLint 的检测范围和其他六类我们研究的电子表格缺陷检测技术很不一样,它仅仅关注与检测由于错误引用导致的公式不一致性缺陷。
第二,ExceLint 认为公式丢失的缺陷不太重要,因为它们不会立刻触发错误。
然而,其他所有技术都认为这样的缺陷是有害的,并检测这些缺陷,因为这类缺陷在未来的维护过程中,可能导致意料之外的错误。
事实上,公式丢失缺陷在实际的电子表格中很常见(例如,不同技术在 VEnron2 数据集上检测出的此类缺陷占比 64-78\%)。
因此,直接对比\wa 和 ExceLint 可能不太公平,并严重低估 ExceLint 的有效性。另外,我们在实际运行 ExceLint 的过程中遇到了别的问题,比如它缺少人工标记的基准测试集。
因此,我们把对它的比较留给其他研究人员,在将来进行综合的比较和实验。

\chapter{总结与展望}

本章将对本文的工作进行总结,然后展望与电子表格错误检测相关的研究中需要解决的问题和可能存在的解决方案。

\section{工作总结}
电子表格在终端用户开发工具中变得越来越流行,帮助用户迅速高效完成数值存储和公式计算等即时任务,在各个领域都获得广泛应用。
然而电子表格容易产生错误的特性也给许多终端用户带来了麻烦,给众多公司带来了经济损失,以及随着表格复杂化带来的较高的维护成本。

针对电子表格中的公式缺陷问题,近二十年来,研究者们提出了许多不同思路的缺陷检测技术,致力于提升电子表格的可靠性。
根据检测方法的设计思路可以分成三类,分别是利用类型推导技术,利用基于规则的模式匹配技术,以及基于学习/聚类的机器学习技术。
其中,后两类方法逐渐占据主流,前者的特点在于能够相对精准地识别特定子类型的公式缺陷,但往往整体的识别召回率不高;而后者的特点恰恰相反,整体的识别召回率相对较高,但由于聚类或学习技术的自动化特征抽取不够精准,往往导致精度不高,使得终端用户在进行人工检查时耗费较多精力。

本文提出的技术方案立足于电子表格中的聚类技术,在聚类的过程中,通过设计和规则类似的有效性属性检测方法,来对所得的单元格类进行精化,将仅仅因为特征相似但语义上无关的单元格从类中剔除出去,实现整体检测效果的提升。
这些有效性属性关注三个层面的类有效性,单个单元格层面,多个单元格层面,以及整个类的层面,能够极大地提升检测精度。
在基准测试集和以更大语料库为测试数据的案例研究中证实\wa 在检测电子表格缺陷方面的有效性。

\section{研究展望}

就本文采用的优化技术而言,\wa 从整体分析来看依然有提升改进的空间。
\begin{itemize}
    \item 有效性属性框架:本文提供了一个从下到上相对完整的有效性属性的思考框架,立足于每个单元格自身,到单元格之间的关系,再到单元格类(即单元格集合)的有效性,在整个框架中仍可以添加新的或者改进当前的有效性属性,进而更好地提升检测精度;
    \item 捞回丢失但相关的单元格:本文只关注于如何提升检测的精度,但就其聚类而言,依然会丢掉一些缺乏共性特征,但也的确属于相同语义的公式,如何将这些公式单元格捞回到对应的类中,也是一个值得进一步思考和探究的角度;
    \item 缺陷检测阶段:另外,本文主要关注于在聚类阶段进行优化,不过,在缺陷检测阶段,也有优化和提升的空间,比如对检测出的缺陷提供额外的修复建议。
\end{itemize}

另外,从终端用户使用电子表格的立场来看,如何能够将电子表格缺陷检测技术集成到电子表格开发环境中,类似编程的集成开发环境一样,即时给用户反馈当前表格,乃至当前正在编辑的公式是否存在缺陷的提示,能够更加有效地辅助终端用户在开发过程中就同时进行提升可靠性的工作。


\begin{acknowledgement}

首先,感谢南京大学和南京大学计算机系提供良好的校园环境和学习、科研环境,能够享受到各种校内便利和福利,身为南大学子,感到非常幸福。
在我的整个五年研究生过程中,聆听了许多老师开设的一流的计算机方向课程,帮助我建立了对整个计算机科学(尤其是程序设计)的世界观和模型观。其中,印象最为深刻的有,蒋炎岩老师的操作系统课,李越等老师的静态分析课,冯新宇老师的形式化语义课,喻良老师的可计算课,这些课程深入浅出,让我被计算机科学与理论的精深所震撼。在未来的人生历程中,我也会继续对这方面内容的学习和理解,希望有一天,能够帮助其他年轻人进入计算机领域———这个有趣的小世界。

其次,感谢许畅老师,蒋炎岩老师和王慧妍师姐,在具体科研问题上对我的帮助是巨大的。最开始接手的电子表格测试问题,直到入学三年后的第一次投稿,虽然论文整体来讲,对本科研领域的贡献极小,但对我个人而言,是第一次完整参与了一篇论文从最开始的各种编程,试错,到最后的方法确定,执行各种类型的受控实验,再到最后的论文写作,整个过程中有很多值得反思和深思的地方。到最近两年进行的程序修复方向的工作和学习,使我对计算机系统有了更加深入的了解,虽然比起那些在程序修复领域的专业学者还存在很大的差距,但也勉强算是给自己入了个门,知道山外有山,人外有人的道理在任何领域都是适用的。

再次,感谢软件所的各位老师和同学,尤其是在饭点和我一起吃饭的赵泽林同学,李文杰同学等,我们常常相互调侃,相互安慰。没有他们,也就没有每天最快乐的时光。感谢你们!

最后,感谢我的爸妈和沈加慧同学,是他们陪伴着我走过了每一个艰难的时刻,并总是选择无条件地支持我的每一个决定,尊重我的每一个决定。希望在今后的日子里,依然有你们的陪伴,让我们一起迈向下一段人生旅程。

最后的最后,感谢这个时代,希望往后的我能够“面朝大海,春暖花开”,做一个能够与自己的心灵坦诚相待的人!

\end{acknowledgement}

\bibliography{sample}

\backmatter
%%%%%%%%%%%%%%%%%%%%%%%%%%%%%%%%%%%%%%%%%%%%%%%%%%%%%%%%%%%%%%%%%%%%%%%%%%%%%%%
% 作者简历与科研成果页,应放在backmatter之后
\begin{resume}
  % 论文作者身份简介,一句话即可。
  \begin{authorinfo}
  \noindent 李达,男,汉族,1992年12月出生,江苏省泗洪人。
  \end{authorinfo}
  % 论文作者教育经历列表,按日期从近到远排列,不包括将要申请的学位。
  \begin{education}
  \item[2016年9月 --- 2021年6月] 南京大学计算机科学与技术系 \hfill 硕士
  \item[2012年9月 --- 2016年6月] 南京航空航天大学计算机科学与技术学院 \hfill 本科
  \end{education}
  % 论文作者在攻读学位期间所发表的文章的列表,按发表日期从近到远排列。
  \begin{publications}
    \item \textbf{Da Li}, Huiyan Wang, Chang Xu, Fengming Shi, Xiaoxing Ma, Jian Lu, "WARDER: Refining cell clustering for effective spreadsheet defect detection via validity properties," in \textsl{Proc. IEEE International Conference on Software Quality, Reliability and Security (QRS) 2019}, Jul. 2019. [CCF-C]
    \item \textbf{Da Li}, Huiyan Wang, Chang Xu, Ruiqing Zhang, Shing-Chi Cheung, Xiaoxing Ma, "SGUARD: A Feature-based Clustering Tool for Effective Spreadsheet Defect Detection," in \textsl{Proc. IEEE/ACM International Conference on Automated Software Engineering (ASE)}, Nov. 2019. [CCF-A, Tool Demo Track]
    \item Yicheng Huang, Chang Xu, Yanyan Jiang, Huiyan Wang, \textbf{Da Li}, "WARDER: Towards Effective Spreadsheet Defect Detection by validity-based Cell Cluster Refinements," in \textsl{Journal of Systems and Software, 2020}, Sept. 2020. [CCF-B]
    \item 那个专利信息
  
  \end{publications}
  % 论文作者在攻读学位期间参与的科研课题的列表,按照日期从近到远排列。
  \begin{projects}
  \item xx``yy''
  (课题年限~xx年yy月 --- xx年yy月),负责zz。
  \end{projects}
 \end{resume}


\input{chapters/rights.tex}

\end{document}