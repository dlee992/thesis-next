\chapter{chapter}

\lipsum[1]

\section{section}\label{sec:rate}
\lipsum[1]
\subsection{subsection}
\lipsum[1]

\begin{figure}[htbp]
  \centering
  \includegraphics[width=0.6\linewidth]{./figure/github.jpg}
  \caption{单图示例}
  \label{fig:system}
\end{figure}

\chapter{算法}

\begin{algorithm}[htbp]
  \caption{算法名字}
  \label{alg:alg1}
  \begin{algorithmic}[1]
        \REQUIRE 这是输入
        \ENSURE 这是输出
        \WHILE {flag}
		      \STATE 这是语句
        \ENDWHILE
  \end{algorithmic}
\end{algorithm}

\chapter{实验验证}

实验硬件设备如图\ref{img:1}所示。
\begin{figure}[htbp]
\begin{minipage}[t]{0.5\textwidth}
\centering
\includegraphics[width=0.8\textwidth]{./figure/github.jpg}
\caption{实验硬件设备总览}
\label{img:1}
\end{minipage}
\begin{minipage}[t]{0.5\textwidth}
\centering
\includegraphics[width=0.8\textwidth]{./figure/github.jpg}
\caption{实验测量示意图}
\label{img:2}
\end{minipage}
\end{figure}

图\ref{fig:sub}所示子图\ref{subfig:a}和子图\ref{subfig:b}。
\begin{figure}[H]
	\begin{subfigure}{.5\textwidth}
		\centering
		\includegraphics[width=0.8\textwidth]{./figure/github.jpg}
		\caption{子图}
		\label{subfig:a}
	\end{subfigure}
	\begin{subfigure}{.5\textwidth}
		\centering
		\includegraphics[width=0.8\textwidth]{./figure/github.jpg}
		\caption{子图}
		\label{subfig:b}
	\end{subfigure}
\caption{子图样例}
\label{fig:sub}
\end{figure}

\chapter{总结与展望}
\lipsum[1]



%%%%%%%%%%%%%%%%%%%%%%%%%%%%%%%%%%%%%%%%%%%%%%%%%%%%%%%%%%%%%%%%%%%%%%%%%%%%%%%
% 致谢,应放在《结论》之后
\begin{acknowledgement}
%thanks
\lipsum[1]

\end{acknowledgement}

%%%%%%%%%%%%%%%%%%%%%%%%%%%%%%%%%%%%%%%%%%%%%%%%%%%%%%%%%%%%%%%%%%%%%%%%%%%%%%%




% 参考文献。应放在\backmatter之前。
% 推荐使用BibTeX,若不使用BibTeX时注释掉下面一句。
%\nocite{*}
\bibliography{sample}


% 附录,必须放在参考文献后,backmatter前
\appendix
\chapter{附录代码}\label{app:1}
\section{main函数}
\begin{lstlisting}[language=C]
int main()
{
	return 0;
}
\end{lstlisting}
%%%%%%%%%%%%%%%%%%%%%%%%%%%%%%%%%%%%%%%%%%%%%%%%%%%%%%%%%%%%%%%%%%%%%%%%%%%%%%%
% 书籍附件
\backmatter