\abstracttitlea{电子表格的单元格聚类与缺陷检测优化技术研究}

\begin{abstract}

终端用户编程获得了广泛关注,电子表格无疑是终端用户编程中最流行的编程范式之一。
电子表格通常被用来存储、计算和分析用户数据,帮助用户进行数据计算和决策计划。
由于电子表格中可能存在大量的数据,其组织方式也多种多样,因此其中可能潜藏着各种类型的缺陷。
近年来,多家金融投资公司因为电子表格中潜藏的错误,造成了巨大的经济损失。

利用自动化技术帮助终端用户检测电子表格是否存在错误,尤为重要。
其中,如何自动化检测与公式相关的电子表格缺陷得到了最多的关注。
近二十年来,研究者们提出了各种各样的方法来检测公式缺陷。
目前检测效果较好的主流方法可以依据它们的技术思路分成两类:基于规则匹配和基于聚类/学习算法的缺陷检测技术。
前者通常能够依据预先设计的精准规则,以较高的准确率检测到特定类型的单元格缺陷,但相对而言检测的召回率不高;
而后者能够根据公式特征和单元格的布局特征对单元格进行聚类,大幅提高检测的召回率,但相对而言,后者由于电子表格本身的复杂性,对公式的结构和特征提取不够准确,导致最终的检测精度不高。

考虑到上述不足,文本基于 \cu 提出了一个新颖的技术\wa 。
虽然 \cu 具有自适应性的能力,可以进行跨表格、跨布局风格的学习,但同时它存在将相关和不相关的单元格吸纳进同类中的不足。
\wa 通过对单元格类的构建过程进行自上而下的有效性检验和优化,能够准确地过滤掉和单元格类无关的单元格或不符合要求的整个单元格类,有效提升了单元格聚类和缺陷检测的效果。

综上所述,本文的主要贡献有如下三点:
\begin{enumerate}
    \item 提出了基于有效性检验的电子表格单元格聚类和缺陷检测技术\wa ,聚焦于单元格自身、单元格之间关系和整个单元格类的有效性检验和优化,提升 \cu 的单元格聚类效果,进而提升电子表格缺陷检测的效果;

    \item 利用一个被相关工作广泛应用的电子表格基准测试集和一个大规模的电子表格语料库,对 \wa 进行了充分评估验证,并对比了当下最流行的其它电子表格缺陷检测技术;

    \item 实现了两个可视化工具:一个 Excel 插件 EGuard,可在操作 Excel 工作表的同时,看到 \wa 对当前打开的工作表的检测结果和执行信息,使得终端用户能够迅速检查和修复工具报告出的缺陷;另一个集成检测工具 \sg ,可对比多种主流技术的检测结果并统一展示。
\end{enumerate}

\keywords{电子表格检测;单元格聚类;缺陷检测;有效性检验;软件工程}

\end{abstract}