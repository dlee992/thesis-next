\begin{acknowledgement}

首先,感谢南京大学和南京大学计算机系提供良好的校园环境和学习、科研环境,能够享受到各种校内便利和福利,身为南大学子,感到非常幸福。
在我的整个五年研究生过程中,聆听了许多老师开设的一流的计算机方向课程,帮助我建立了对整个计算机科学(尤其是程序设计)的世界观和模型观。其中,印象最为深刻的有,蒋炎岩老师的操作系统课,李越等老师的静态分析课,冯新宇老师的形式化语义课,喻良老师的可计算课,这些课程深入浅出,让我被计算机科学与理论的精深所震撼。在未来的人生历程中,我也会继续对这方面内容的学习和理解,希望有一天,能够帮助其他年轻人进入计算机领域———这个有趣的小世界。

其次,感谢许畅老师,蒋炎岩老师和王慧妍师姐,在具体科研问题上对我的帮助是巨大的。最开始接手的电子表格测试问题,直到入学三年后的第一次投稿,虽然论文整体来讲,对本科研领域的贡献极小,但对我个人而言,是第一次完整参与了一篇论文从最开始的各种编程,试错,到最后的方法确定,执行各种类型的受控实验,再到最后的论文写作,整个过程中有很多值得反思和深思的地方。到最近两年进行的程序修复方向的工作和学习,使我对计算机系统有了更加深入的了解,虽然比起那些在程序修复领域的专业学者还存在很大的差距,但也勉强算是给自己入了个门,知道山外有山,人外有人的道理在任何领域都是适用的。

再次,感谢软件所的各位老师和同学,尤其是在饭点和我一起吃饭的赵泽林同学,李文杰同学等,我们常常相互调侃,相互安慰。没有他们,也就没有每天最快乐的时光。感谢你们!

最后,感谢我的爸妈和沈加慧同学,是他们陪伴着我走过了每一个艰难的时刻,并总是选择无条件地支持我的每一个决定,尊重我的每一个决定。希望在今后的日子里,依然有你们的陪伴,让我们一起迈向下一段人生旅程。

最后的最后,感谢这个时代,希望往后的我能够“面朝大海,春暖花开”,做一个能够与自己的心灵坦诚相待的人!

\end{acknowledgement}