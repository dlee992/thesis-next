\chapter{绪论}\label{introduction}

本章是对本文内容的简要说明,将先从电子表格,及其公式错误两方面介绍本文的研究背景,然后从研究问题、研究思路、研究贡献三方面介绍本文工作,最后对全文的组织结构进行说明。

\section{研究背景}

\subsection{电子表格及其用法}

电子表格\cite{BHR12}是用于组织、分析和存储表格形式的数据的计算机应用程序。
尽管它们最初是为会计或簿记任务而开发的,但现在已广泛用于构建,排序和共享表格列表的任何环境。

电子表格对在表格的单元格中输入的数据进行操作。
每个单元格可以包含数字或文本数据,也可以包含基于其他单元格的内容自动计算并显示值的公式的结果。
电子表格也可以引用一个这样的电子文档。电子表格用户可以调整任何存储的值,并观察对计算值的影响。
这使得电子表格可用于“假设分析”,因为无需手动重新计算即可快速调查许多情况。
现代电子表格软件可以具有多个交互工作表,并且可以以文本和数字或图形形式显示数据。

除了执行基本的算术和数学功能外,现代电子表格还提供了用于常见财务会计和统计操作的内置功能。
这样的计算如净现值或标准偏差可以应用于表格数据与在公式中的预编程的功能。
电子表格程序还提供条件表达式,在文本和数字之间转换的函数以及对文本字符串进行操作的函数。

\begin{figure}[t!]
    \centering
    \includegraphics[width=0.6\textwidth]{figure/github.jpg}
    \caption{Microsoft Excel Spreadsheet Software}   
    \label{fig:1} 
\end{figure}

电子表格已在整个商业环境中取代了基于纸张的系统。
如图\ref{fig:1}所示,Microsoft Excel现在在Windows和Macintosh平台上拥有最大的市场份额。
电子表格程序是办公室生产力套件的标准功能; 自从Web应用程序问世以来,办公套件现在也以Web应用程序形式存在。
基于Web的电子表格是一个相对较新的类别。

% 介绍电子表格的使用
以Excel为例,简要介绍一下电子表格的基本功能。

\subsection{电子表格中的缺陷}

然而,现实生活中时常发生因为电子表格的使用不当,导致较大经济损失和名誉受损的情况。比如,2012年伦敦奥运会,售票人员使用电子表格来统计已售票数和总座位数,结果数据录入错误,导致游泳比赛发生门票超售事件,损失不小。另外,20xx年,两位金融领域的学者利用电子表格进行货币相关的数据统计分析,得出下一年度应当收紧货币发行,事后第三方人员发现,在该电子表格中,由于公式引用不当,致使最终结论完全相反,造成了很大的负面影响。

对于日常使用电子表格的财会人员,随着表格中数据量的日益增大(比如,整个学校的教职工数据录入和分析),对整张表格的数据准确性和完整性,常常难以把控。有研究表明,大约xx\%的电子表格中,都不可避免地含有一些计算错误,即本文关注的重点,公式使用上产生的错误或漏洞。

\section{本文工作}

\subsection{研究问题}


\subsection{研究思路}


\subsection{研究贡献}

本文的主要贡献有如下三点:

\begin{enumerate}
    \item 我们提出了电子表格缺陷检测技术WARDER,聚焦于单元格和多个单元格形成的类的有效性属性来精华CUSTODES方法的单元格聚类过程,整体上有效提升了电子表格缺陷检测的有效性。
    \item 我们使用了两个现有的电子表格基准测试集合一个大规模的电子表格数据库,对WARDER进行了充分测试和评估,并对比了现在最流行的其他电子表格缺陷检测技术。
    \item 我们实现了两个可视化工具:一个基于Java的可视化工具SGUARD,可对比多种主流技术的检测结果并直观展示;另一个基于JavaScript的Excel插件,通过运行本插件,可即时、分步看到WARDER针对当前表格的检测结果,方便用户立即确认和修复表格缺陷。
\end{enumerate}

\section{论文组织结构}

本文的剩余部分组织结构如下:
% 第1章为绪论,将对本文的工作的研究背景、研究问题和研究贡献进行简单介绍;
第2章为相关工作综述,将对和本文相关的工作按照各自类别总结并进行介绍;
第3章为研究问题与分析,将对本文所研究的问题进行定义,并对本文的研究思路进行详细的分析和解说;第4章为系统设计,将对解决本文研究问题所用的方法进行详细介绍;
第5章为系统实现和工具展示,将介绍本文的WARDER检测方法实现为系统的过程中的实现细节,以及对两个可视化工具的使用进行必要的说明;
第6章为实验评估,将阐述本文设计实验的思路和运行实验的细节,并对实验结果进行分析,验证本文系统的有效性和高效性;
最后,本文在第7章总结了本文的工作,并论述了对该方向研究的展望。