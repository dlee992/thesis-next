% \chapter{研究问题的核心概念}

\section{电子表格的单元格类}

我们首先给出在本文中电子表格单元格类的定义:

\begin{definition}
    \textit{单元格类}是单元格的集合,即$cluster = \{cell_1, cell_2, \dots\}$,其中每个单元格 $cell_i$ 都具有\textit{相似}的\textit{计算语义}。
\end{definition}

要理解单元格类,需要在各类相关技术中明确其中的两个关键概念:

\begin{itemize}
    \item \textit{计算语义}的概念:能够直接或者间接表明该单元格的计算语义的单元格特征,都可以用来一定程度上表征该单元格的计算语义。例如,直接的特征就是该单元格拥有的公式(但该公式可能有缺陷,未必完全准确地反映该单元格的真实应该表达的计算语义)。间接的特征也存在很多形式,比如,该单元格周围的其他单元格的计算特征,或者该单元格所在的表头字符串信息等,都可以暗示该单元格可能拥有的计算语义。
    \item \textit{相似}的概念:通常刻画相似的方法是量化该单元格与一个单元格集合中其他单元格的计算特征的相似度。比如,可以比较从单元格中抽取出的直接特征和间接特征的相似度的方法,来权衡最终两个单元格的相似度。
\end{itemize}

如图\ref{figure-3-3}所示,从单元格 B20 向左延伸到 F20 的五个单元格都是统计每个单元格上方的若干个单元格的数量和,这五个单元格就天然构成了一个单元格类。他们都具有在相似位置进行求和运算的计算语义,首先单元格 B20、C20 和 E20 拥有完全相同的 R1C1 表示法下的公式表达式,因此可以归为同类;而考虑到单元格 D20 和 F20 的位置,以及 F20 具有的相似公式表达式(区别仅在于引用单元格不尽相同),D20 具有数值 100,这恰好就是单元格 D11 到 D19 的总和,因此也可以归为同类。

\begin{figure}[tbp]    
    \centering
    \includegraphics[width=1\textwidth]{figure/figure-3-3.png}
    \caption{电子表格的单元格类和公式缺陷示意图}
    \label{figure-3-3}
\end{figure}

在缺乏终端用户的先验知识的前提下,如何定义计算语义,以及如何量化单元格之间的计算语义相似度,正是各类电子表格缺陷检测技术的核心挑战之一\cite{Barowy2018excelint}。
使用识别单元格类作为中间环节的缺陷检测技术,通常会将电子表格中的所有公式单元格和数值单元格进行分类,形成单元格类的集合。最后,利用这些单元格类,在每个类中,进行缺陷检测识别出有问题的单元格。


\section{电子表格的公式缺陷}
在定义“单元格类”概念后,我们来解释何为电子表格的公式缺陷。
同样参考图\ref{figure-3-3},其中总共包含 4 个单元格类(单元格类用字体颜色区分,分别为红、黄、蓝和紫)和 10 个公式缺陷,下面详细分析。
通常,有缺陷的单元格中一定存在\textit{替换}导致的异常,通常是用不合法的子表达式替换了原有的合法子表达式。
这里,我们根据定义 3-1 的表达式结构刻画对公式缺陷进行如下分类。

\subsection{用数值替换的公式缺陷}
在公式单元格中用某个数值常量替换一个子表达式(如单元格引用),但它的同类单元格并没有进行这番替换时,该公式单元格具有的公式缺陷就称为\textit{用数值替换的公式缺陷}。
现实情况下,存在大量用常量替换整个公式的情况,此时公式单元格退化成了数值单元格,使得检测该类缺陷更为困难。
在图\ref{figure-3-3}中,有 8 个单元格具有此类公式缺陷,虽然它们本身都是数值单元格,但它们的确具有计算语义,如单元格 D16,显然它的同列几个公式(从 D11 到 D18)都是用来计算左侧单元格的占比,而 D16 只有一个常量 1.15。
如果将来改动了单元格 C16 的值,那么 D16 显示的数值就不再准确,因此该类公式缺陷应该被修复。

\subsection{用单元格引用替换的公式缺陷}
类似地,我们把一个单元格用一个错误的单元格或者单元格范围替换一个子表达式,然而其它的同类单元格并没有这么替换时,这类单元格的缺陷称为\textit{用单元格引用替换的公式缺陷}。
在图\ref{figure-3-3}中, 单元格 F20 的引用集合$\sigma$(SUM(F11:F18))比起同一行中的其他几个公式缺少了对单元格 F19 的引用。
如果将来在第 19 行添加了新的数据,那么 F20 的计算结果不再准确,因此该类公式缺陷应该被修复。

\subsection{用操作符/函数替换的公式缺陷}
类似地,我们把一个单元格用一个错误的操作符/函数替换一个子表达式,然而其它的同类单元格并没有这么替换时,这类单元格的缺陷称为\textit{用操作符/函数替换的公式缺陷}。
在图\ref{figure-3-3}中,单元格 E26 具有与 B26 和 C26 不同的操作符,E26 使用加法操作,而另外两个公式采用内置函数 SUM 来求和。
如果将来求和的范围扩大了,由于公式形式不同,可能忘记对 E26 的更新,因此该类公式缺陷也该被修复。

\subsection{小结}
上述三类单元格缺陷类型已在真实商业场景下频繁发现\cite{panko2006facing,powell2008critical}。
由于缺陷的类型和变体多种多样,普通终端用户在选用电子表格测试技术时,由于并不预先知道自己的表格中潜在的缺陷类型,难以针对性地选择特定检测技术。
那么基于学习的单元格聚类技术因为天生具有对多种缺陷的适应性,能够把上述各种类型的缺陷单元格都尽可能不遗漏地加入到单元格类中,最后再统一地进行缺陷检测分析。
这对于终端用户来说,是一个容易接受并采纳的电子表格缺陷检测技术方案。