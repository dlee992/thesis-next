\chapter{实验评估}

本章中,我们对WARDER进行实验评估,并和现有的电子表格缺陷检测技术作对比。
首先给出研究问题并简要介绍系统实现,接着描述实验设计和设置,然后展示实验结果和案例研究结果,并进行详尽的分析和讨论。


\section{研究问题}
我们旨在回答下列三个研究问题:
\begin{itemize}
    \item \textbf{研究问题1(有效性)}:与现有电子表格缺陷检测技术相比,\wa 在单元格聚类和缺陷检测方面有效性如何?
    \item \textbf{研究问题2(相关性)}:\wa 优化后的单元格聚类对最终的缺陷检测精度提升有多大贡献?
    \item \textbf{研究问题3(独立性)}:\wa 的三个有效性检验方法对缺陷检测的效果提升分别有多大?
\end{itemize}

\section{系统实现}
\wa 技术在实验评估中使用 Java 语言实现,并使用 Apache POI 库 \footnote{https://poi.apache.org} 来读写 Excel 格式的电子表格。
相应源码发布在 GitHub 网站上 \footnote{https://github.com/dlee992/QRS19-Code} 。

\wa 工具总共包含约 7,300 行 Java 代码,除了包含 \cu 的源代码以外,额外增改了约 2,500 行代码。
类似于 \cu 的方式,\wa 标记它分析过的电子表格的方法是使用不同的背景色来标注检测到的同一张工作表内的所有单元格类,并通过在单元格右上角添加注释的方式标注含有公式缺陷的单元格的备注信息,包括所在类的信息和具体的公式缺陷种类。


\section{实验设计和设置}

\begin{table}[tbp]
    \centering
    \caption{基准测试集的统计数据}
    \label{table1}
    %\large
    \resizebox{\columnwidth}{!}{
    \begin{tabular}{|m{.15\columnwidth}<{\centering}|m{.12\columnwidth}<{\centering}|m{.15\columnwidth}<{\centering}|m{.2\columnwidth}<{\centering}|m{.15\columnwidth}<{\centering}|m{.25\columnwidth}<{\centering}|}
    \hline
    \textbf{\# 电子表格} &  \textbf{\# 工作表} &  \textbf{\# 单元格} &  \textbf{\# 公式单元格} &  \textbf{\# 单元格类} &  \textbf{\# 有缺陷的单元格} \\ 
    \hline
    70 & 291 & 189,027 & 26,716 & 1,610 & 1,974 \\
    \hline
    \end{tabular}}
\end{table}

\subsection{基准测试集} 

为了便于\wa 和它的前身\cu 的比较,我们选择了 \cu 采用的从EUSES语料库中采样的测试集,作为我们的实验基准测试集。如表~\ref{table1}所示,该测试集包含70个电子表格和291个工作表。这291个工作表包含189,027个单元格,其中包含26,716个公式单元格。出于实验评估的目的,该测试集包含人工标注的数据(标记方法详见~\cite{cheung2016custodes}),其中包含1,610个单元格类和1,974个有缺陷的单元格(丢失公式或含有不一致的公式)。

\subsection{测试技术} 

在实验中,\wa 将和五个之前提到的电子表格缺陷检测技术进行对比,即\uc,\di,\am,\ca 和 \cu。我们从它们各自的原作者那里获取了对应的可执行工具或源码。主要在缺陷检测的有效性方面进行比较。对于\ca,我们额外比较了它们的单元格聚类的有效性。

为了评估的三个有效性精化的独立性(研究问题3),我们采用不同的配置来测试各自的实验效果,三种配置依次分别标记为\wasc (带有单单元格的有效性精化),\wamc (带有多单元格的有效性精化),\wawc (带有整个类的有效性精化)。最后,带有全部三种精化的配置被称为\wa-full,或简记为\wa。

\subsection{评价指标} 

针对缺陷检测的有效性,我们首先统计每个技术报告的缺陷数量,以及其中的真阳性(TP),假阳性(FP)和假阴性(FN)数量。基于此,我们进一步根据如下三组公式计算精度$precision_d$,召回率$recall_d$和$F\text{-}measure_d$值,来衡量电子表格缺陷检测上各技术的有效性。

\begin{gather*}
    precision_d=\frac{TP}{TP + FP}\qquad recall_d = \frac{TP}{TP + FN}\\
    f\text{-}measure_d = \frac{2 \times precision_d \times recall_d}{precision_d + recall_d}
\end{gather*}

针对电子表格缺陷的有效性(适用于\wa 和\cu ),我们采用和\cu 类似的方式来统计真阳性(TP),假阳性(FP)和假阴性(FN)数量。类似地,我们也统计这两个技术在单元格聚类上的精度 $precision_c$,召回率 $recall_c$和\fmc 值。

\subsection{测试环境} 

所有实验在一台台式机上进行,配有 Intel$^\circledR$ Core\texttrademark\ i7-6700 CPU @3.41GHz 处理器和 64GB 内存。该机器上安装了微软Windows 10专业版操作系统和Oracle Java 8执行环境。


\section{实验结果和分析}

本节中,我们会依次分析实验结果并回答上述提出的三个研究问题。

\subsection{有效性}

我们首先实验评估\wa 在单元格聚类和缺陷检测方面的有效性,然后和其他五个电子表格缺陷检测技术的结果进行比较。

\begin{table}[tbp]
    \centering
    \caption{6个电子表格缺陷检测技术的检测结果}
    \label{table2}
    \large
    \resizebox{\columnwidth}{!}{
    \begin{tabular}{|c|c|c|c|c|c|c|}
    \hline
    \textbf{技术} & \textbf{检测出} & \textbf{TP} & \textbf{FP} & \textbf{$precision_d$} & \textbf{$recall_d$} & \textbf{$F\text{-}measure_d$} \\
    \hline
        UCheck & 204 & 1 & 203 & 0.5\% & 0.1\% & 0.00 \\
    \hline
        Dimension & 1,824 & 14 & 1,828 & 0.8\% & 0.7\% & 0.01 \\
    \hline
        AmCheck & 2,372 & 1,316 & 1,030 & 56.1\% & 66.7\% & 0.61 \\
    \hline
        CACheck & 1,866 & 1,350 & 516 & 72.3\% & 68.4\% & 0.70 \\
    \hline
        CUSTODES & 2,380 & 1,539 & 841 & 64.7\% & 78.2\% & 0.71 \\
    \hline
        WARDER & 1,612 & 1,415 & 197 & 87.8\% & 71.9\% & 0.79 \\
    \hline
    \end{tabular}}
\end{table}

就缺陷检测而言,表~\ref{table2}对比了所有6个检测技术的结果,包括精度,召回率,\fmd,检测出的缺陷数量,以及其中包含的真阳性和假阳性。从表中,我们观察到:

\begin{enumerate}
    \item 由于他们有限的分析范围,\uc 和\di 获得了较低的分数(精度和召回率都低于1\%,\fmd 值也仅有0.01),这点也与之前的实证研究结果~\cite{zhang2017effectively}保持一致;
    \item 由于他们有效的分析模式(比如,单元格元组),\am 和 \ca 获得了较好的分数(56.1-72.3\%的精度,66.7-68.4\%的召回率,以及0.61-0.70的\fmd 值);
    \item \cu 比 \ca 获得了略微更好地分数(0.71的\fmd 值),以及对召回率的较大提升(72.8\%,对应的\ca 召回率为68.4\%);
    \item 如同预期的那样,\wa 关注于提升检测精度,对比于\cu 获得了大幅度提升,从64.7\% 提升到 87.8\%,最终总体上提升了\fmd 的值,从0.71到0.79,尽管对召回率有6\%的牺牲。 从数据中,可能产生一定疑惑:\cu 比 \wa 多检测出124个真阳性,但同时伴随着更多的假阳性(达644个),这对于终端用户的人工验证来说是很大的负担。
\end{enumerate}

\begin{figure}%
    \centering
    \subfloat[精度比较(仅展示被影响的工作表)]{{
        \includegraphics[width=0.40\textwidth]{figure/figure5a.pdf} 
        \label{figure5a}
        }}%
    \subfloat[召回率比较(整体)]{{
        \includegraphics[width=0.55\textwidth]{figure/figure5b.pdf} 
        \label{figure5b}
        }}%
    \caption{CUSTODES 和 WARDER 的单元格聚类结果}%
    \label{figure5}%
\end{figure}

就单元格聚类而言,图\ref{figure5}比较了\cu 和 \wa 在精度和召回率两方面的结果。从精度比较来看(如图\ref{figure5}(a)),我们把包含至少有一个单元格类的282个工作表分成两类:
\begin{enumerate}
    \item \wa 提升了其中31个工作表的精度,降低了7个;
    \item 对于余下精度保持不变的244个工作表,其中194个已达到100\%(即上限)。
\end{enumerate}
换言之,\wa 提升了225个工作表的单元格聚类精度,要么是实际提升精度,要么已经达到上限。如图\ref{figure6}所示,我们进一步观察这38个精度产生变化的工作表。我们不难发现,\wa 在不同程度上提升了单元格聚类的精度(0.3\%-94.6\%,平均20.7\%),这样的精度提升是显著的,远多于精度受损的部分。另外从图\ref{figure5}(b)中,我们也注意到,\wa 在精度提升上的有效性也仅导致召回率的轻微下降(2.4\%)。

\begin{figure*}
    \centering
    \includegraphics[width = \columnwidth]{figure/figure6.pdf}
    \caption{CUSTODES 和 WARDER 的单元格聚类结果(精度变化方面)}
    \label{figure6}
\end{figure*}
\begin{figure}[tp]
    \centering
    \includegraphics[width = \columnwidth]{figure/figure7.jpeg}
    \caption{工作表“Detail for College of Education” (包含一个绿色标记的单元格类)}
    \label{figure7}
\end{figure}

如图~\ref{figure6}所示,我们也观察到,\wa 对大多数精度受影响的工作表是正面作用,但也有极少量的例外情况,如名为“Detail for College ...”的工作表,它的聚类精度从100\%降低到了0。因此,我们进一步检查了该工作表的聚类情况。如图\ref{figure7}所示,人工标记的结果是,{O11,W11,Z11,AD11,AR11}应该被安排在同一个类中。\cu 看似正确地对这些单元格进行了分类,而\wa 分类得很差。然而,我们发现这五个单元格实际上包含不同的公式,因此它们违反了\wa 的 \wcvp (在这些单元格中不存在一个公共的计算目标能够覆盖过半数量)。事实上,这个单元格类的确不包含任何缺陷。因此,这个单元格聚类精度下降的案例并不影响\wa 的缺陷检测能力。

因此,针对研究问题1,我们能够得出如下结论:\textit{\wa 在单元格聚类和缺陷检测方面是有效的。它极大地提升了精度,达15.5-87.3\%,并且在所有被比较的电子表格缺陷检测技术中,取得了最佳的 \fmd 值(0.79)。}

\subsection{相关性}

\begin{table}[tbp]
    \centering
    \caption{\wa 相对于 \cu 在单元格聚类和缺陷检测上的的精度变化 ($\uparrow$: 精度提升, $\downarrow$: 精度下降, $\to$: 精度保持不变)}
    \label{table3}
    %\Large
    \begin{tabular}{|m{.16\columnwidth}<{\centering}|m{.16\columnwidth}<{\centering}|m{.16\columnwidth}<{\centering}|m{.08\columnwidth}<{\centering}|m{.18\columnwidth}<{\centering}|}
    \hline
    \textbf{相关性类型} & \textbf{聚类的精度变化} & \textbf{缺陷检测的精度变化}  & \textbf{\# 工作表} & \textbf{总计} \\
    \hline
        \multirow{3}{1.2cm}{正相关} & $\uparrow$  & $\uparrow$ & 8 & \multirow{3}*{126 (90.0\%)}\\
    \cline{2-4}
        ~ & $\downarrow$ &  $\downarrow$  & 2 & ~\\
    \cline{2-4}
        ~ & $\to$ &  $\to$  & 116 & ~\\
    \hline
            \multirow{4}{1.2cm}{负相关}& $\uparrow$ &  $\to$  & 5  & ~\\
    \cline{2-4}
         ~ & $\uparrow$  & $\downarrow$ & 2 & \multirow{4}*{9 (6.4\%)}\\
     \cline{2-4}

        ~ & $\downarrow$ &  $\to$  & 2 & ~ \\
    \cline{2-4}
        ~ & $\downarrow$ &  $\uparrow$  & 0 & ~ \\
    \hline
    \multirow{2}*{不相关} & $\to$  & $\uparrow$ & 5 & \multirow{2}*{5 (3.6\%)}\\
    \cline{2-4}
        ~ & $\to$ &  $\downarrow$  & 0 & ~ \\
    \hline
    合计 & - & - & 140 & 140 (100.0\%) \\
    \hline
    \end{tabular}
\end{table}

在这一小节里,我们研究\wa 相对于 \cu 在单元格聚类和缺陷检测上的的精度提升。

如表\ref{table3}所示,我们用三个符号 $\uparrow$,$\downarrow$和$\to$分别表示精度提升,精度下降,以及精度保持不变这三种情况。按照相关性类型,我们把人工标记中至少含有1个缺陷的工作表分成3个类型:
\begin{enumerate}
    \item \textit{正相关}:当相对于\cu 的单元格聚类,\wa 的精度有提升,下降或不变时,在缺陷检测上精度变化表现一致;
    \item \textit{负相关}:当相对于\cu 的单元格聚类,\wa 的精度有提升时,在缺陷检测上精度下降或保持不变;或者相反地,\wa 的精度下降时,在缺陷检测上精度提升或保持不变;
    \item \textit{不相关}:其他的组合情况,均划归此类,既不表现出正相关,也不表现出负相关。
\end{enumerate}
总得来说,我们能够观察到:第一类占据主导(90\%),因此表明 \wa 对单元格聚类的精度提升方法的确能够带来缺陷检测精度的提升,并且九成比例成明显正相关。

\begin{figure}%
    \centering
    \subfloat[CUSTODES 和 WARDER 在单元格聚类上的精度]{{
        \includegraphics[width=.95\columnwidth]{figure/figure8a.pdf} 
        \label{fig8a}
        }}%
    \qquad
    \subfloat[CUSTODES 和 WARDER 在缺陷检测上的精度]{{
        \includegraphics[width=.95\columnwidth]{figure/figure8b.pdf} 
        \label{fig8b}
        }}%
    \caption{\cu 和 \wa 中有精度变化的工作表}%
    \label{figure8}%
\end{figure}
\begin{figure}[tp]
    \centering
    \includegraphics[width = .95\columnwidth]{figure/figure9.jpg}
    \caption{工作表 ``CO'' (绿色标注的两个单元格统计特定引用区域的最大值,橙色标注的两个单元格应该用来统计次大值)}
    \label{figure9}
\end{figure}

图\ref{figure8}展示了在单元格聚类和缺陷检测上\wa 和 \cu 的更细致的精度对比。为了展示的清晰性和简洁性,我们移除了精度前后无变化的 116 个工作表,仅罗列出剩下的 24 个。细致的观察可以发现:在单元格聚类和缺陷检测的精度变化上大部分情况下是正相关的。不过,其中一个例外是名为“CO”的工作表,在这张表上,\wa 的缺陷检测精度从100\%跌至 0,然而单元格聚类精度却有提升。我们进一步检查了这个情况。如图\ref{figure9}所示,人工认为单元格{B11,E11}(绿色标记) 和{C11,F11}(橙色标记)应该构成两个类,并且单元格 C11 和 F11 应该是两个缺陷(用红色三角标记)。\cu 因为意外地将这四个单元格分成同类,而偶然地检测到了这两个缺陷。这是偶然的,因为这四个单元格本质上并不共享相同的计算目标(两个绿色的单元格用来计算最大值,而两个橙色的单元格用来计算第二大值)。\cu 只是简单地将两个橙色的单元格标记为缺陷,因为他们仅仅包含纯数值(丢失公式的缺陷)。另一方面,\wa 仅仅将{B11,E11}划分为同一类,因此无法检测到其中的任何缺陷。它漏掉了两个橙色单元格,因为它们不包含任何公式,且没有和它们相同计算目标的公式存在,因而无法被识别到任何类中。当没有额外的证据时(例如,更多的计算第二大值的单元格出现,并且相应的公式也存在于某些单元格中),\wa 选择不把这些单元格放到任何类中(否则,可能导致更多的假阳性)。

因此,针对研究问题2,我们能够得出如下结论:\textit{\wa 在单元格聚类上的改进的确进一步提升了缺陷检测的结果,这一相关性得到90.0\%的工作表的数据支撑。}


\subsection{独立性}

\begin{table}[tbp]
    \centering
    \caption{\cu 和 \wa 在不同配置下的缺陷检测结果}
    \label{table4}
    \large
    \resizebox{\columnwidth}{!}{
    \begin{tabular}{|c|c|c|c|c|c|c|}
    \hline
    \textbf{技术} & \textbf{检测出} & \textbf{TP} & \textbf{FP} & \textbf{$precision_d$} & \textbf{$recall_d$} & \textbf{$F\text{-}measure_d$} \\
    \hline
        CUSTODES & 2,380 & 1,539 & 841 & 64.7\% & 78.2\% & 0.71 \\
    \hline
        WARDER-sc &2,083 & 1,506 & 577 & 72.3\% & 76.3\% & 0.74\\
    \hline
        WARDER-mc &2,164 & 1,507 & 657 & 69.6\% & 76.3\% & 0.73\\
    \hline
        WARDER-wc &2,071 & 1,498 & 573 & 72.3\% & 75.9\% & 0.74\\
    \hline
        WARDER-full & 1,612 & 1,415 & 197 & 87.8\% & 71.9\% & 0.79 \\
    \hline
    \end{tabular}}
\end{table}

最后,我们研究\wa 的三个有效性精化在检测电子表格缺陷上各自的表现。\wa 被配置成单独使用每一种精化方法(正如前面提到的,命名为\wasc,\wamc 和\wawc ),并和同时使用三种方法的\wa 作比较(\wa -full)。

表\ref{table4}对比了\cu 和 \wa 的四种配置的缺陷检测结果。我们可以观察到:
\begin{enumerate}
    \item \wa 的每个有效性精化是有用的,并且各自能够比\cu 在缺陷检测精度上提升 4.9-7.6\%,在召回率上有较小的牺牲(1.9-2.3\%),最终对\fmd 值有小幅提升(从 0.71 提升到 0.73-0.74);
    \item 组合所有三种有效性精化(即\wa -full) 可以获得最高的精度(87.8\%)和\fmd 值(0.79),这也和之前的表\ref{table1}数据相吻合。
\end{enumerate}

因此,针对研究问题3,我们能够得出如下结论:\textit{\wa 的三个有效性属性能够各自独立展现出对缺陷检测的正面效果;同时在三者组合运用时,也能够获得整体的最佳效果。}
\section{案例研究}

\begin{table}[tbp]
    \centering
    \caption{在 VEnron2 数据集上四个电子表格缺陷检测技术的检测结果}
    \label{table5}
    %\large
    \resizebox{\columnwidth}{!}{
    \begin{tabular}{|m{.15\columnwidth}<{\centering}|m{.16\columnwidth}<{\centering}|m{.16\columnwidth}<{\centering}|m{.14\columnwidth}<{\centering}|m{.12\columnwidth}<{\centering}|m{.1\columnwidth}<{\centering}|m{.12\columnwidth}<{\centering}|}
    \hline
    ~& \multicolumn{3}{c|}{\textbf{全部数据集包含 6,258 个工作表}} & \multicolumn{3}{c|}{\textbf{采样 300 个工作表}} \\
    \hline
    \textbf{技术} & \textbf{\# 工作表} & \textbf{\# 缺陷} & \textbf{耗时 (分钟)} & \textbf{\# 缺陷} & \textbf{\# TP} & \textbf{Precision} \\
    \hline
        AmCheck  & 859  & 20,280 & 21 & 3,316 & 540 & 16.3\% \\
    \hline
        CACheck  & 953   & 12,953 & 372 & 1,559 & 534 & 34.3\% \\
    \hline
        CUSTODES & 1,284 & 14,102 & 537  & 2,334 & 629 & 26.9\% \\
    \hline
        WARDER   & 1,136 & 9,462  & 518  & 1,240 & 512 & 41.3\% \\
    \hline
    \end{tabular}}
\end{table}
\begin{figure}[tp]
    \centering
    \includegraphics[width = 0.9\columnwidth]{figure/figure10.png}
    \caption{四个电子表格缺陷检测技术报告出的真阳性交集的韦恩图}
    \label{figure10}
\end{figure}

除了前面的受控实验,我们也在更大规模的语料库 \ven \cite{xu2017spreadcluster} 上对 \wa 的检测有效性进行了实验。
\ven 包含 1,609 个版本化小组,是从 Enron 语料库 \cite{hermans2015enron} 中 79,983 个工作表中提炼而来。
我们从每个版本化的小组中选取罪行的电子表格文件,总共包含 1,609 个电子表格,其中含有 7,140 个工作表,作为我们案例研究的数据集。
我们把不同的电子表格缺陷检测技术拿来检测这些工作表,进而比较分析它们的检测效果。
考虑到\uc 和 \di 检测出的缺陷数量很少(低于总量的1\%),在案例研究中我们选择了另外四种技术,即\am ,\ca  ,\cu 和 \wa 。
为了加快比较和对比的公平性,我们移除了那些至少有一个技术无法正常执行的工作表(比如:导致异常崩溃,或者超出了我们设定的每张表的五分钟执行时间上限,设置时限的目的是防止程序陷入死锁或者其他未知的错误而影响整体实验进度)。
这一过滤方式给我们的案例研究最终留下了总计 6,258 个工作表。

我们注意到 \ven 并不包含用于评估电子表格缺陷检测技术有效性的人工标注结果(即,每张工作表里实际含有哪些有缺陷的单元格)。
因此,我们主要关注与四个技术的检测精度比较。
另外,由于整体数据量依然庞大,虽然我们对每个工作表都运行了检测程序(用于衡量他们的时间效率),但我们不得不使用采样的方法来选择少量工作表,进行人工参与的检测精度对比。
采样过程遵循了 \am 和 \cu 建议的评估方式。
在所有 6,258 个工作表中, 1,525 个工作表被至少一个技术检测出含有缺陷。
基于此,我们随机采样了约 20\% 的工作表(即 300 个)来进行人工审查,该审查的目的是标记每一个被工具报告出来的缺陷是否是真阳性还是假阳性。
表\ref{table5}对比了四种技术在人工审查后的缺陷检测结果。

从表\ref{table5}的右半部分,我们可以观察到:
\begin{enumerate}
    \item 在四个技术中,\wa 取得了最高的缺陷检测精度(41.3\%),超出其他技术7.0-25.0\%,这和之前我们验证过的 \wa 更加关注于在 \cu 的基础上提升精度的结论(41.3\% vs. 26.9\%);
    \item 尽管\wa 报告除更少的真阳性数量(512),但同时也伴随着少得多的假阳性数量(728),这比其他技术的假阳性数量少了 297-2,048个,这一特征在现实场景可能相当实用,因为所有的电子表格缺陷必须经过人工的审核验证,节省了大量人力。
\end{enumerate}

从表\ref{table5}的左半部分,我们可以观察到:
类似于采样的 300 个工作表,\wa 检测出了最少的缺陷数(9,462),对比于\am 的 20,280,\ca 的 12,953 和 \cu 的 14,102。
考虑到\wa 获得了最高的精度,他的检测按质量预期是最高的(比如,在\wa 检测出的 1,240 个缺陷中有 512 个真阳性,相比较而言,在\am 检测出的 3,316(2.6倍) 个缺陷中仅有 540 个真阳性(仅1.05倍))。
考虑到时间效率,\am 处理所有的 6,258个工作表所花费时间最短(仅 21 分钟),其他三个技术都花了相对长的时间(从 351 到 516分钟)。
这结果表明 \am 适合用于快速识别潜在的电子表格缺陷,但因为他的检测质量相对较低(精度是16.3\%),它更适合作为其他技术的互补验证来使用(比如用于过滤大多数的假阳性缺陷)。
另外,我们注意到\wa 在 \cu 的基础上,仅提升了它的单元格聚类过程。
因此,\wa 也没有那么高效(花费 518 分钟),对比于\cu 的时间消耗(537分钟)。
不过,\wa 移除了不相关的单元格和不合格的单元格蕾,减少了不必要的后续分析,这一努力带来了运行时间的略微缩短(19分钟)。

除了整体的检测精度和时间消耗对比,我们也研究了四个技术检测出的真阳性缺陷的交集,如图\ref{figure10}中的韦恩图所示。
在韦恩图中,黄色椭圆代表\am 检测出的真阳性缺陷,依次地,绿色椭圆代表\ca ,粉色椭圆代表\cu ,紫色代表\wa 。
每个子区域代表两个或多个技术检测出的真阳性缺陷的交集。
从图中,我们可以观察出:
\begin{enumerate}
    \item 基于模板的技术(\am 和 \ca)和基于聚类学习的技术(\cu 和 \wa)明显彼此互补。前者总共检测出 243(9+216+18) 个无法被后者检测出的缺陷,而后者检测出 270(211+59) 个前者无法检测的缺陷。此结果表明两类技术都很有用处;
    \item 在基于模板的技术中,\ca 继承自 \am 技术。相应地,它们检测出的缺陷有一大部分是相同的(78.4\%,602 个中的 472 个是相同的)。因此,\am 单独检测出进 68 个缺陷,而 \ca 单独检测出也仅 62 个。不过,\ca 仍是受欢迎的技术,考虑到它显著减少了假阳性(从\am 的 2776 个减少到 1025 个);
    \item 在基于学习的技术中心,\wa 精化了 \cu 的检测结果。相应地,\wa 仅检测出了 512 个真阳性,是\cu 的子集(81.4\%)。不过,\wa 本身就是更加关注与过滤掉不相关的单元格和不合格的类,并且这一努力带来了显著的假阳性的减少(从\cu 的 1,705 个减少到 728 个);
    \item \ca 和 \wa 作为各技术流派的代表,仍然是相互补充的。它们各自能够检测出 292 个和 270 个对方无法检测出的缺陷。这一结果很明确地表明它们的互补性很强。
\end{enumerate}

因此,\textit{\wa 在实际使用的电子表格上的缺陷检测表现也令人满意。它获得了最高的精度(41.3\%),超出其他技术多达 7.0-25.0\%。它的时间消耗有点高,但也和\ca 相当(在同一个量级),并且少于它的前身\cu。就检测到的缺陷而言,所有相关技术都有它们各自的优势,并且细致调研表明他们是彼此互补的。}

\section{威胁性分析和讨论}

一个主要的内在威胁性来源是单元格聚类指标的计算方式,即\prc ,\rec 和 \fmc 。
它们是基于真阳性,假阳性和假阴性概念计算而来,而后三者又是根据\cu 的结对相似度比较计算而来。
我们注意到这样的比较会统计是否两个单元格属于同一个类或者属于不同的类的单元格对的数量。
这样的计算方式不同于衡量缺陷检测效果的方式。
因此,研究\wa 的单元格聚类和它的缺陷检测相关性结论可能在一定程度上受到影响。
然而,我们依然观察到 90\% 的工作表在我们的分析下是正相关的。
这表明\wa 对单元格聚类的提升的确有助于最终的缺陷检测。

另外,我们也注意到\wa 啊仍然有提升的空间,考虑到它不能检测到某些特定的电子表格缺陷,正如我们之前分析的那样。
\begin{enumerate}
    \item \wa 关注于识别不相关的单元格(将其移除)和无效的单元格类(取消该类的分析)。它并没有挥手哪些相关的单元格,但被\cu 遗漏掉的部分;
    \item 即便所有相关的单元格都能被正确的聚类,\cu 本身仍然不能检测出某些缺陷,由于其自身在缺陷检测上有限的分析能力。因为\wa 仅仅关注于改善单元格聚类过程,并没有涉及到缺陷检测部分的优化。因此,两种技术可能都无法检测到这样的电子表格缺陷。
\end{enumerate}
不过,我们在实验中观察到:\wa 已经极大地提升了 \cu 的适应性。这表明\wa 关注到了对优化起主导作用的因素。不过上述分析也的确指出了进一步优化的新方向。

最后,一个主要的外部威胁性来源是我们尝试了但没能和另外两个基于机器学习的电子表格缺陷检测技术,Melford\cite{singh2017melford} 和 ExceLint\cite{Barowy2018excelint} 进行对比实验和案例研究。
前者,我们没有找到可获得的工具。后者,我们找到了对应的工具但在实验评估时遇到了问题。
首先,ExceLint 的检测范围和其他六类我们研究的电子表格缺陷检测技术很不一样,它仅仅关注与检测由于错误引用导致的公式不一致性缺陷。
第二,ExceLint 认为公式丢失的缺陷不太重要,因为它们不会立刻触发错误。
然而,其他所有技术都认为这样的缺陷是有害的,并检测这些缺陷,因为这类缺陷在未来的维护过程中,可能导致意料之外的错误。
事实上,公式丢失缺陷在实际的电子表格中很常见(例如,不同技术在 VEnron2 数据集上检测出的此类缺陷占比 64-78\%)。
因此,直接对比\wa 和 ExceLint 可能不太公平,并严重低估 ExceLint 的有效性。另外,我们在实际运行 ExceLint 的过程中遇到了别的问题,比如它缺少人工标记的基准测试集。
因此,我们把对它的比较留给其他研究人员,在将来进行综合的比较和实验。