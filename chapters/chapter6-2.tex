\section{实验设计和设置}

\begin{table}[tbp]
    \centering
    \caption{基准测试集的统计数据}
    \label{table1}
    %\large
    \resizebox{\columnwidth}{!}{
    \begin{tabular}{|m{.15\columnwidth}<{\centering}|m{.12\columnwidth}<{\centering}|m{.15\columnwidth}<{\centering}|m{.2\columnwidth}<{\centering}|m{.15\columnwidth}<{\centering}|m{.25\columnwidth}<{\centering}|}
    \hline
    \textbf{\# 电子表格} &  \textbf{\# 工作表} &  \textbf{\# 单元格} &  \textbf{\# 公式单元格} &  \textbf{\# 单元格类} &  \textbf{\# 有缺陷的单元格} \\ 
    \hline
    70 & 291 & 189,027 & 26,716 & 1,610 & 1,974 \\
    \hline
    \end{tabular}}
\end{table}

\subsection{基准测试集} 

为了便于\wa 和它的前身\cu 的比较,我们选择了 \cu 采用的从EUSES语料库中采样的测试集,作为我们的实验基准测试集。如表~\ref{table1}所示,该测试集包含70个电子表格和291个工作表。这291个工作表包含189,027个单元格,其中包含26,716个公式单元格。出于实验评估的目的,该测试集包含人工标注的数据(标记方法详见~\cite{cheung2016custodes}),其中包含1,610个单元格类和1,974个有缺陷的单元格(丢失公式或含有不一致的公式)。

\subsection{测试技术} 

在实验中,\wa 将和五个之前提到的电子表格缺陷检测技术进行对比,即\uc,\di,\am,\ca 和 \cu。我们从它们各自的原作者那里获取了对应的可执行工具或源码。主要在缺陷检测的有效性方面进行比较。对于\ca,我们额外比较了它们的单元格聚类的有效性。

为了评估的三个有效性精化的独立性(研究问题3),我们采用不同的配置来测试各自的实验效果,三种配置依次分别标记为\wasc (带有单单元格的有效性精化),\wamc (带有多单元格的有效性精化),\wawc (带有整个类的有效性精化)。最后,带有全部三种精化的配置被称为\wa-full,或简记为\wa。

\subsection{评价指标} 

针对缺陷检测的有效性,我们首先统计每个技术报告的缺陷数量,以及其中的真阳性(TP),假阳性(FP)和假阴性(FN)数量。基于此,我们进一步根据如下三组公式计算精度$precision_d$,召回率$recall_d$和$F\text{-}measure_d$值,来衡量电子表格缺陷检测上各技术的有效性。

\begin{gather*}
    precision_d=\frac{TP}{TP + FP}\qquad recall_d = \frac{TP}{TP + FN}\\
    f\text{-}measure_d = \frac{2 \times precision_d \times recall_d}{precision_d + recall_d}
\end{gather*}

针对电子表格缺陷的有效性(适用于\wa 和\cu ),我们采用和\cu 类似的方式来统计真阳性(TP),假阳性(FP)和假阴性(FN)数量。类似地,我们也统计这两个技术在单元格聚类上的精度 $precision_c$,召回率 $recall_c$和\fmc 值。

\subsection{测试环境} 

所有实验在一台台式机上进行,配有 Intel$^\circledR$ Core\texttrademark\ i7-6700 CPU @3.41GHz 处理器和 64GB 内存。该机器上安装了微软Windows 10专业版操作系统和Oracle Java 8执行环境。

