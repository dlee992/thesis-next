\chapter{绪论}\label{introduction}

本章是对本文内容的简要说明,将先从电子表格,及其公式错误两方面介绍本文的研究背景,然后从研究问题、研究思路、研究贡献三方面介绍本文工作,最后对全文的组织结构进行说明。

\section{研究背景}

\subsection{电子表格的流行}

终端用户编程已经获得了广泛关注和应用。
毫无置疑,电子表格是终端用户的程序中最流行的应用范式。
电子表格被用户广泛接受,在世界范围内被广泛使用。
几乎每个使用计算机办公的人都会接触到电子表格环境,比如 Microsoft Excel,Google Sheets 和 Apple Numbers。
精算师,管理者,销售人员和行政人员都是电子表格的重度用户\cite{scaffidi2005estimating},通常基于这些电子表格数据来做出决策。
它的使用简易性(所见即所得的设计宗旨)是电子表格如此流行的主要原因之一。

\subsection{电子表格的使用风险}

但另一方面,人们对电子表格的数据可靠性也一直缺乏必要的重视。
电子表格的易用性使得人们不再需要通过课程训练来掌握基本用法,而是让人们通过边使用边学习的方式来掌握电子表格用法。
因此,很多人并不清楚伴随电子表格而来的使用风险。
电子表格错误导致严重的名誉损失或巨额财产损失的事实已经屡见不鲜\footnote{http://www.eusprig.org/horror-stories.htm}:

\begin{itemize}
    \item 2012 年,JP 摩根在价值-风险模型计算中使用电子表格,其中的一个电子表格错误导致损失 40 亿美元;
    \item 2016 年,US-village West Baraboo 因为一个电子表格错误,在一次追加投资中额外多投入了 40 万美元;
    \item 2016 年,加拿大电力公司 TransAlta 因为电子表格中的复制粘贴错误,导致额外购买了 2,400 万美元的美国电力传输保障合同。
\end{itemize}

即便如此,依然有人认为这些事例的发生是极其罕见的,但多个实证调研 \cite{panko2016we,powell2009impact} 证实了在商业应用的电子表格环境依然缺乏有效的质量保障环节。
更深层面的原因是,因为终端用户大多不是专业程序员,他们通常不遵循良好的编程原则。
相对地,终端用户通常更加关心如何把具体任务尽快完成。
这个过程中,缺少对计算过程的建模,测试,封装和模块化等等在程序员编程中已经广为人知和应用的保障技术和编程原则,这就导致了在现实场景下,高达 90\% 的电子表格包含或多或少的错误\cite{rajalingham2008classification},进而造成相关公司的经济损失。


\section{本文工作}

\subsection{研究问题和现状}

% 这里需要一段引入到我们的检测 scope
这类电子表格中的错误集中发生在数值单元格和公式单元格中,现有实证研究\cite{panko2010revising}表明后者通常是问题的根源。
再本文中,我们将公式单元格中的错误称为电子表格中的缺陷\footnote{在本文的剩余部分如无特殊说明,电子表格的错误(error)和缺陷(defect)是等价概念,指和公式有关的单元格错误,在某些研究工作中用潜在错误(smell)来指代同一概念},并且关注于针对这类缺陷进行检测的有效技术。

% 介绍现有检测技术

检测电子表格缺陷并不容易。
首先,电子表格通常是普通终端用户维护的,维护过程中可能包含不专业的操作,比如出于各种可能的现实目的,用纯数值改写整个公式或者公式中的子公式。
这会使得电子表格中的缺陷频繁出现。
第二,追踪电子表格的修改历史在普通电子表格场景下难以获得,因为普通终端用户并不习惯使用类似 Git 的版本控制软件。
这使得我们难以判断电子表格缺陷是在何时何处引入的。
第三,电子表格单元格之间的计算目标关系通常是隐晦的,这对检测电子表格缺陷进一步带来了难度。

为了应对这些挑战,学界已经提出了各种电子表格缺陷检测技术:

\begin{itemize}
    \item 有的技术依赖于表头信息来推导公式引用中的类型不一致(如\uc \cite{abraham2007ucheck} 和 \di \cite{chambers2009automatic})。
    \item 有的技术利用矩形布局的特性来识别丢失公式或者不一致的公式缺陷(如\am \cite{dou2014spreadsheet} 和 \ca \cite{dou2017cacheck})。
    \item 还有的技术使用自适应学习的方法来检测公式单元格中的缺陷(如\cu \cite{cheung2016custodes},Melford\cite{singh2017melford} 和 ExceLint \cite{Barowy2018excelint})。
\end{itemize}

% 及其不足

然而,这些电子表格缺陷检测技术仍有各自的不足之处:

\begin{itemize}
    \item 对第一类来说(基于类型分析的技术),它们的推导相对粗糙,并且关注于非常有限的缺陷种类,导致缺陷检测的精度和召回率都比较低\cite{zhang2017effectively};
    \item 对第二类来说(基于模式、规则匹配的技术),它们倚仗的模式或规则通常是严格且精确的,专注于电子表格中具有某种特性的缺陷类型,因而旺旺能够获得较高的检测精度(比如\am 的 71.9\% 和 \ca 的 86.8\% \cite{dou2017cacheck}),但又无法使用不同电子表格中多变的布局和排版,进而导致相对有限的召回率(比如\am 的 60.3\% 和 \ca 的 71.0\% \cite{dou2017cacheck});
    \item 对第三类来说(基于学习的技术),由于它们先天的适应性的学习能力,得到了广泛应用。我们以 \cu \cite{cheung2016custodes} 为例,因为它被认为是当前“最好的自动化缺陷检测工具”\cite{Barowy2018excelint}。\cu 根据电子表格中的公式语义对单元格进行分类,同时通过在不同的电子表格中学习各种单元格特征,来控制有缺陷的公式的不吻合度和不同电子表格中的多种布局风格对分类结果的影响。通过这种原理和方法,它极大地提升了检测的召回率(达到 80\% \cite{cheung2016custodes}),但同时由于把一些无关的单元格也分类到某些类中,进而牺牲了检测的精度(仅 65\% \cite{cheung2016custodes})。
\end{itemize}

\subsection{研究思路}

考虑到这些已知缺陷,在本篇工作中我们提出了一个新颖的技术\wa \footnote{\wa 的命名方式遵循了 \cu 的风格} \cite{li2019warder},基于\cu 的先天的适应性的跨表格、跨布局风格的学习能力,但同时改善它在将相关和不相关的单元格混进同类中的不足之处。
我们的核心观察是:在对单元格进行聚类时,如果偶然地将不相关的单元格混入到类中,必会极大地影响缺陷检测的有效性(例如牺牲检测的精度)。
因此,在\wa 中我们的主要突破点关注于精化电子表格的聚类过程,通过尽可能的排除那些不相关的单元格使得聚类结果可靠性更高,同时也要保留\cu 最初的自适应的学习能力。

精化方法分为从底向上的三层设计:
\begin{itemize}
    \item 单个单元格的检验:当把数值单元格放进类中时,我们考虑它自身自身是否是有效的。当用一个公式来替换当前数值单元格里的内容时,该单元格应该是是可计算的。否则,如果所有公式的测试都以失败告终(例如引用一个错误的单元格内容或者无效的单元格引用),这个单元格就是有问题的,应当被阻止放进该类;
    \item 多个单元格的检验:当把数值单元格放心类中时,它们不会破坏类中已有单元格之间的特定现有属性。例如,类中已有单元格的引用集合没有交集,如果新加入的单元格当它的值被类中某个其他单元格的公式替换时,单元格见的引用集合却出现了交集,这个单元格就是有问题的,应当被阻止放进该类;
    \item 整个类的检验:即它关注于类层面而不是单元格层面的检验。因为每个类是用来包含具有相似计算目标的单元格,那么每个类中应该存在一个统一的公式能够覆盖大多数单元格。如果某个类无法满足该属性,整个类就应当从缺陷检测的环节中排除。
\end{itemize}

有了这三个精细的检验方法,\wa 和它的前身 \cu,一起其他电子表格缺陷检测技术进行对比,展现出了明显的优势。
例如,在\cu 的来源于 EUSES \cite{fisher2005euses} 的包含 291 个工作表的基准测试集之上,\wa 与\cu 对比,取得了显著的提升,在单元格聚类方面提升了 79.8\% 的工作表,尤其在精度方面取得了 0.3-94.6\% 的提升(平均 20.7\%),在召回率方面有 2.4\% 的牺牲。
\wa 通过更加有效的单元格聚类,对比与\cu,在缺陷检测方面使得精度提升了 23.1\%。
结合召回率来看,总体上使得$F\text{-}measure$值从 0.71 提升至 0.79。
\wa 也比其他电子表格缺陷检测技术有了较大的提升,取得了平均 87.8\% 的精度表现和 71.9\% 的召回率表现,对比于其他技术的 0.5-72.4\% 和 0.1-68.4\%。
另外,在一个更大的数据集 VEnron2 \cite{xu2017spreadcluster} 上(从 Enron \cite{hermans2015enron} 里的 79,983 个工作表中提取出 1,609 个含有演化关系的电子表格集合),\wa 也展现了相对于其他技术的独特的优势(在检测精度方面达到 41.3\%,对比于其他技术的16.3-34.3\%)。

\subsection{研究贡献}

综上所述,本文的主要贡献有如下三点:

\begin{enumerate}
    \item 我们提出了电子表格缺陷检测技术WARDER,聚焦于单元格和多个单元格形成的类的有效性属性来精化CUSTODES方法的单元格聚类过程,整体上有效提升了电子表格缺陷检测的有效性。
    \item 我们使用了两个现有的电子表格基准测试集合一个大规模的电子表格数据库,对WARDER进行了充分测试和评估,并对比了现在最流行的其他电子表格缺陷检测技术。
    \item 我们实现了两个可视化工具:一个基于Java的可视化工具SGUARD\cite{li2019sguard},可对比多种主流技术的检测结果并直观展示;另一个基于JavaScript的Excel插件 EGuard \footnote{https://github.com/dlee992/EGuard},通过运行本插件,可即时、分步看到WARDER针对当前表格的检测结果,方便用户立即确认和修复表格缺陷。
\end{enumerate}

\section{组织结构}

本文的剩余部分组织结构如下:
% 第1章为绪论,将对本文的工作的研究背景、研究问题和研究贡献进行简单介绍;
第2章为相关工作综述,将对和本文相关的工作按照各自类别总结并进行介绍;
第3章为研究问题与分析,将对本文所研究的问题进行定义,并对本文的研究思路进行详细的分析和解说;第4章为系统设计,将对解决本文研究问题所用的方法进行详细介绍;
第5章为系统实现和工具展示,将介绍本文的WARDER检测方法实现为系统的过程中的实现细节,以及对两个可视化工具的使用进行必要的说明;
第6章为实验评估,将阐述本文设计实验的思路和运行实验的细节,并对实验结果进行分析,验证本文系统的有效性和高效性;
最后,本文在第7章总结了本文的工作,并论述了对该方向研究的展望。