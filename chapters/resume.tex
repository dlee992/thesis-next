%%%%%%%%%%%%%%%%%%%%%%%%%%%%%%%%%%%%%%%%%%%%%%%%%%%%%%%%%%%%%%%%%%%%%%%%%%%%%%%
% 作者简历与科研成果页,应放在backmatter之后
\begin{resume}
  % 论文作者身份简介,一句话即可。
%   \begin{authorinfo}
  % \noindent 李达,男,汉族,1992年12月出生,江苏省泗洪人。
  % \end{authorinfo}
  % % 论文作者教育经历列表,按日期从近到远排列,不包括将要申请的学位。
  % \begin{education}
  % \item[2016年9月 --- 2021年6月] 南京大学计算机科学与技术系 \hfill 硕士
  % \item[2012年9月 --- 2016年6月] 南京航空航天大学计算机科学与技术学院 \hfill 本科
  % \end{education}
  % % 论文作者在攻读学位期间所发表的文章的列表,按发表日期从近到远排列。
  % \begin{publications}
    % \item Yicheng Huang, Chang Xu, Yanyan Jiang, Huiyan Wang, \textbf{Da Li}, "WARDER: Towards Effective Spreadsheet Defect Detection by validity-based Cell Cluster Refinements," in \textsl{Journal of Systems and Software, 2020}, Sept. 2020. [CCF-B]
    % \item \textbf{Da Li}, Huiyan Wang, Chang Xu, Ruiqing Zhang, Shing-Chi Cheung, Xiaoxing Ma, "SGUARD: A Feature-based Clustering Tool for Effective Spreadsheet Defect Detection," in \textsl{Proc. IEEE/ACM International Conference on Automated Software Engineering (ASE)}, Nov. 2019. [CCF-A, Tool Demo Track]
    % \item \textbf{Da Li}, Huiyan Wang, Chang Xu, Fengming Shi, Xiaoxing Ma, Jian Lu, "WARDER: Refining cell clustering for effective spreadsheet defect detection via validity properties," in \textsl{Proc. IEEE International Conference on Software Quality, Reliability and Security (QRS) 2019}, Jul. 2019. [CCF-C]
    % \item 专利申请:一种基于检验单元格聚类的电子表格缺陷检测方法,“许畅、李达、王慧妍、马晓星”,申请日:2019.07.04  
% \end{publications}

  \begin{publications}
    \item CCF-A 类,ASE 软工会议工具展示一篇,一作,2019.11
    \item CCF-C 类,QRS 软工会议论文一篇,一作,2019.7
    \item CCF-B 类,JSS 期刊论文一篇,五作,2020.4
    \item 专利申请,二作,申请日:2019.07
  \end{publications}

% \begin{projects}
  % \item 国家自然科学基金重大课题:面向演化的群智化软件建模与构造方法(61690204),马晓星(2017.01-2021.12)
  % \item 国家自然科学基金面上项目:面向大数据环境的网构软件场景理解和功能及能耗保障研究(61472174),许畅(2015.01-2018.12)
% \end{projects}

\end{resume}

