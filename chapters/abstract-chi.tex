\begin{abstract}

终端用户编程获得了广泛关注和应用,其中,电子表格无疑是终端用户编程中最流行的编程范式之一。
电子表格得到用户的广泛接受,并在世界范围内广泛使用。
电子表格通常被用来存储,计算,和分析用户数据,比如 MS Excel 和 Google Sheet 等软件,提供给用户存储数据和计算公式的方便。
不过,在电子表格编程范式广泛应用的背后,每一个电子表格中也确实潜藏着各种类型的错误和缺陷。
最近十年来,各大商业公司因为使用电子表格时犯错,已经造成了数百万美元的经济损失。

因此,利用软件工程技术帮助终端用户检测相关的电子表格潜在缺陷,已经得到了广泛关注。
过去二十年间,多种不同技术思路的技术和工具被提出,用来检测电子表格中的各类显式错误和隐式缺陷。
随着电子表格软件的迭代更新,简单的显式错误已经能够得到有效的检验,如何检测更加复杂的隐式缺陷得到更加广泛的研究。
具有代表性的已有工作,大体可以依据技术思路分成两类:基于规则校验和基于聚类/学习算法的缺陷检测技术。
前者(代表性工作 \ca)通常能够依据设计的精准规则,以较高的准确率检测到特定类型的单元格缺陷,但相对后者而言检测的召回率不高;
而后者(代表性工作 \cu)根据公式的语法结构和语义特征对单元格进行聚类,进而大幅度提高检测的召回率,但相对前者而言,由于电子表格本身的复杂性,导致对公式的结构和特征提取不够准备,导致最终的检测精度不高。

考虑到上述不足,在本篇工作中我们提出了一个新颖的技术\wa ,基于\cu 的先天的适应性的跨表格、跨布局风格的学习能力,但同时改善它在将相关和不相关的单元格混进同类中的不足之处。
本文的主要贡献如下:
\begin{enumerate}
    \item 提出了电子表格缺陷检测技术\wa,聚焦于单元格和多个单元格形成的类的有效性属性来精化\cu 的单元格聚类过程,整体上有效提升了电子表格缺陷检测的有效性。
    \item 使用了一个现有的电子表格基准测试集和一个大规模的电子表格预料库,对\wa 进行了充分测试和评估,并对比了现在最流行的其他电子表格缺陷检测技术。
    \item 实现了两个可视化工具:一个基于Java的可视化工具\sg ,可对比多种主流技术的检测结果并直观展示;另一个基于 JavaScript 的 Excel 插件,通过运行本插件,可即时、分步看到 \wa 针对当前表格的检测结果,方便用户立即确认和修复表格缺陷。
\end{enumerate}

\keywords{电子表格测试;单元格聚类;缺陷检测;有效性检验;软件工程}

\end{abstract}